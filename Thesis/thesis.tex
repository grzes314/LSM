\documentclass[a4paper,11pt, twoside]{book}
\usepackage[utf8]{inputenc}
\usepackage{amsmath, amsthm, amssymb}
\usepackage[polish, english]{babel}
\usepackage[OT4]{fontenc}
\usepackage[left=2.5cm,right=2.5cm]{geometry}
\usepackage{fancyhdr}
\usepackage[section]{placeins} % keeps floats in their places
\usepackage{graphicx}
\usepackage{color}
\usepackage{titlesec}
%\usepackage{hyperref}
\usepackage{listings} % allows inclusion of code snippets
\usepackage[chapter]{algorithm} % allows to keep algorithms as floats
\usepackage{algpseudocode} % allows writing pseudocodes
\usepackage[font={small,sl}]{caption}
\usepackage{bbm}
\usepackage{subfig} % placing floats side by side
\usepackage{rotating}
\usepackage{sidecap}
\usepackage{wrapfig}


% pretty display of a chapter/part
\addto\captionsenglish{
  \renewcommand\chaptername{}}
\titleformat{\chapter}[display]
  {\normalfont\Large\filcenter\sffamily}
  {\titlerule[1pt]%
   \vspace{1pt}%
   \titlerule
   \vspace{1pc}%
   \LARGE\MakeUppercase{\chaptertitlename} \Roman{chapter}
  }
  {1pc}
  {\titlerule
  \vspace{1pc}%
  \Huge}
  
\definecolor{comment}{RGB}{96,96,192}
\definecolor{colorForKeyWord}{RGB}{165,42,42}

\lstdefinestyle{customJava}{
  belowcaptionskip=1\baselineskip,
  breaklines=true,
  frame=L,
  xleftmargin=\parindent,
  language=Java,
  showstringspaces=false,
  basicstyle=\footnotesize\ttfamily,
  keywordstyle=\bfseries\color{colorForKeyWord},
  commentstyle=\itshape\color{comment},
  identifierstyle=\color{black},
  stringstyle=\color{orange},
}
\lstset{escapechar=@,style=customJava}

\newenvironment{absolutelynopagebreak}
  {\par\nobreak\vfill\penalty0\vfilneg
   \vtop\bgroup}
  {\par\xdef\tpd{\the\prevdepth}\egroup
   \prevdepth=\tpd}


\newtheorem{thm}{Theorem}[chapter]
\newtheorem{prop}[thm]{Proposition}
\newtheorem{coro}[thm]{Corollary}
\newtheorem{lemma}[thm]{Lemma}

\theoremstyle{definition}
\newtheorem{mydef}{Definition}[chapter]
%\newtheorem{notation}[mydef]{Notation}
%\newtheorem{example}{Example}[chapter]

\theoremstyle{remark}
\newtheorem{remark}{Remark}[chapter]

\newcounter{example}[chapter]
\newenvironment{example}
   {\refstepcounter{example} \par\medskip\noindent \textbf{Example~\arabic{chapter}.\arabic{example}.}  }
   {\hfill $\diamondsuit$\par\noindent\ignorespacesafterend}
  
\def\Var{{\rm Var}}
\def\E{{\mathbb{E}}}
\def\P{{\mathbb{P}}\,}
\def\Q{{\mathbb{Q}}\,}
\def\Cov{{\hbox{Cov}}}
\def\Corr{{\hbox{Corr}}}
\def\Em{{\mathbb{E}^*}}
\def\Pm{{\mathbb{P}}^*\,}
\def\R{{\mathbb{R}}}
\def\conv{\xrightarrow[n \rightarrow \infty]{}}
\def\limn{\lim\limits_{n \rightarrow \infty} }
\def\Sa{\bar{S}}
\def\Xa{\bar{X}}
\def\xia{\bar{\xi}}
\def\CMC[#1]{\hat{Y}^{\hbox{\tiny CMC}}_{#1}}
\def\AV[#1]{\hat{Y}^{\hbox{\tiny AV}}_{#1}}
\def\CV[#1]{\hat{Y}^{\hbox{\tiny CV}}_{#1}}
\def\CMCa[#1, #2]{\hat{#1}^{\hbox{\tiny CMC}}_{#2}}
\def\AVa[#1, #2]{\hat{#1}^{\hbox{\tiny AV}}_{#2}}
\def\CVa[#1, #2]{\hat{#1}^{\hbox{\tiny CV}}_{#2}}
\def\bpipe{{\bigl|\bigr.}}
\def\Bpipe{{\Bigl|\Bigr.}}
\DeclareMathOperator*{\esssup}{ess\,sup}



%opening
\title{Pricing Exotic Options using Monte Carlo methods}
\author{Grzegorz Łoś}

\pagestyle{fancy}
\fancyhf{}
\fancyhead[RO]{\small\bfseries\thepage}
\fancyhead[LE,RO]{\small\bfseries\thepage} %do odkomentowania w wersji dwustronnej
\fancyhead[LO]{\small\bfseries\nouppercase\rightmark}
\fancyhead[RE]{\small\bfseries\nouppercase\leftmark} %do odkomentowania w wersji dwustronnej

%\lhead{\nouppercase{\bfseries \leftmark}}
%\rhead{\nouppercase \rightmark}
\setlength{\headheight}{15pt}

\begin{document}
 
\thispagestyle{empty}
\begin{center}
\textbf{\large Uniwersytet Wrocławski\\
Wydział Matematyki i Informatyki\\
Instytut Matematyczny}\\
\vspace{4cm}
\textbf{\textit{\large Grzegorz Łoś}\\
\vspace{0.5cm}
{\Large Pricing Exotic Options using Monte Carlo methods}}\\
\end{center}
\vspace{3cm}
{\hspace*{6.5cm}\large Master's thesis\\
\hspace*{6.5cm}\large  written under the supervision of\\
\hspace*{6.5cm}\large  Dr Paweł Kawa }
\vfill
\begin{center}
{\large Wrocław 2013}\\
\end{center}

\newpage
\thispagestyle{empty}
\vspace*{10cm}
\noindent {\large Oświadczam, że pracę magisterską wykonałem samodzielnie\\ i~zgłaszam ją do oceny.\\[1.5cm]
Data:....................\hfill Podpis autora pracy:.........................\\[1.5cm]
Oświadczam, że praca jest gotowa do oceny przez recenzenta.\\[1.5cm]
Data:.................... \hfill Podpis opiekuna pracy:.........................}

\newpage

\tableofcontents

\newpage


\chapter*{Abstract}
\addcontentsline{toc}{chapter}{\bfseries Abstract}

%\section*{\begin{center}\begin{normalsize} Abstract \end{normalsize}\end{center}}
\begin{quotation}
\noindent
  This thesis presents techniques of option pricing based on Monte Carlo simulations. Mathematical theories underlying the presented methods are recalled, however, the thesis is practical in nature, hence it not always gets deep into mathematical details.
  Instead, it provides many algorithms in the form of concise pseudocode, which outline how to use the theory in practice. The utility of depicted methods is affirmed by implementations in R and Java programming languages. This paper is filled with illustrations of the results obtained by the created application.
  
  The first chapter of the thesis gathers a group of definitions and facts from the probability theory which are essential for the thesis, like It\^{o}'s, lemma Girsanov theorem, a general description of Monte Carlo methods.
  In the second part, Reader may find an introduction to option pricing. It contains description of the market model, Black-Scholes paradigm, definition of the martingale measure. It also provides a detailed explanation how to calibrate parameters of the Black-Scholes model from the real-world market's history.
  In the next part, the Monte Carlo pricing procedure for European options is described. It starts with simple vanilla options, and gradually moves to more complicated exotic instruments, whose exercise may depend on multiple assets or on the whole history of the market scenario. 
  The fourth chapter introduces American contracts and collects several facts necessary to value instruments with American-style exercise. A technique known as Least Squares Monte Carlo or Longstaff-Schwartz method, which may be used to price American options, is described.
  The next part depicts an architecture of a financial Java library, which was created as a part of the thesis and bases on the presented algorithms.
  In the last part, illustrations of several pricing results are collected.
\end{quotation}

\chapter{Preliminaries}
There exists several approaches to the analysis of assets' prices, for example fundamental analysis, technical analysis and quantitative analysis. Mathematicians are interested in the last one, in which every asset is considered to be a stochastic process.
Despite the fact that nobody really believes that movements of asset prices are truly random, markets' stochastic models are commonly used, due to very satisfactory results. However, this approach requires non-trivial mathematical background whose essentials are collected in this part.

\section{Elements of stochastic analysis}
We expect from Reader some basic knowledge of the probability theory and stochastic processes. Purpose of this section is to recall definitions and facts essential for this work and to establish notation.

\begin{mydef}
 Let $(\Omega, \mathcal{F}, \P)$ be a probability space, $(E, \mathcal{E})$ be a measurable space and $T$ be a positive real number. A \textbf{stochastic process} with values in a measurable space $E$, indexed by an interval $[0,T]$, is a family of random variables $X = (X_t)_{t = 0}^T$, where each $X_t$ is $E$-valued.
 
 For given $\omega \in \Omega$ a \textbf{trajectory} of process $X$ is a function $t \mapsto X_t(\omega)$, with domain $T$ and codomain $E$.
\end{mydef}

\begin{remark}
 Stochastic process may be seen as a function $X: \Omega \rightarrow \mathbb{E}^T$. Then a trajectory is a value of such function, i.e. for given $\omega,\ X(\omega)$ is a trajectory. In the other words, a stochastic process is a random function and a trajectory is its concrete realization.
\end{remark}

\begin{remark}
 In our applications space $E$ always equals $\R$ or $\R^n$. 
\end{remark}

\begin{mydef}
 A \textbf{Brownian motion} (or a \textbf{Wiener process}) is a stochastic process $(W_t)_{t \geq 0}$ defined by following conditions:
 \begin{itemize}
  \item $W_0 = 0$ a.s.,
  \item for any $t,\ W_t \sim \mathcal{N}(0,t)$,
  \item increments of $W$ are independent (i.e. for any $t_0 \leq t_1 \leq \ldots \leq t_n$ random variables $W_{t_0}, W_{t_1} - W_{t_0},\ \ldots, W_{t_n} - W_{t_{n-1}}$ are independent,
  \item increments of $W$ are stationary (i.e. for every $0 \leq s < t,\ W_t-W_s$ is equal in distribution to $W_{t-s})$,
  \item trajectories of $W$ are continuous a.s.
 \end{itemize}
\end{mydef}
\noindent \textbf{In this thesis $W$ always denotes the Brownian motion.}

\begin{mydef}
 \textbf{Filtration} $(\mathcal{F}_t)_{t=0}^T$ on the probability space $(\Omega, \mathcal{F}, {P})$ is an increasing family of $\sigma$-algebras contained in $\mathcal{F}$, i.e. for all $s<t,\ \mathcal{F}_s \subseteq \mathcal{F}_t \subseteq \mathcal{F}$.
\end{mydef}
\noindent Sometimes $\mathcal{F}_t$ is interpreted as a set of all events observable up to time $t$.

\begin{mydef}
 Process $X=(X_t)_{t=0}^T$ is called \textbf{adapted} with respect to the filtration $\mathcal{F}_t$ if and only if for all $t~\in~[0,T],\ X_t$ is $\mathcal{F}_t$-measurable.
\end{mydef}
\noindent Minimal filtration to which $X$ is adapted is of course filtration generated by $X$, called \textbf{natural filtration}, defined as $\mathcal{F}_t^X = \sigma(X_s:\ s \leq t)$.

\begin{mydef}
 A \textbf{stopping time} with respect to the filtration $(\mathcal{F}_t)_{t=0}^T$ is a random variable $\tau:\ \Omega \rightarrow [0,T]\cup\{\infty\}$, such that $\{\tau \leq t\} \in \mathcal{F}_t$ for all $t \in [0,T]$.
\end{mydef}
If $X$ is a process corresponding to some risky game, then a stopping time may be seen as a strategy which tells us whether we should withdraw at time $t$, basing only on the information accessible at time $t$.
\begin{example}
 A typical example of a stopping time is first $t$, when $X_t$ reaches a fixed barrier, i.e.
 \[\tau = \inf\{t\in[0,T]: X_t \geq b\}\]
\end{example}
\noindent Stopping times play important role in the financial mathematics. Often we are interested in finding an optimal strategy for exercising an American option. Such strategy is a stopping time.

\begin{mydef}
 We call a stochastic process $M=(M_t)_{t=0}^T$ a \textbf{martingale} with respect to the filtration $(\mathcal{F}_t)_{t=0}^T$ if and only if it satisfies following conditions:
 \begin{itemize}
  \item for all $t \in [0,T],\ M_t$ is $\mathcal{F}_t$-measurable,
  \item for all $t \in [0,T],\ \E|M_t| < \infty$,
  \item for all $0 \leq s < t \leq T,\ \E[M_t|\mathcal{F}_s] = M_s$  a.s.
 \end{itemize}
\end{mydef}

\begin{example}
 Brownian motion $W$ is a martingale with respect to its natural filtration $\mathcal{F}$. Indeed, for all $t > s$ we have
 \begin{samepage}
  \begin{itemize}
  \item $W_t \sim \mathcal{F}_t$,
  \item $\E|W_t| \leq \sqrt{t} < \infty$ (because from Jensen's inequality: $(\E|W_t|)^2 \leq \E W_t^2 = t$),
  \item $\E[W_t|\mathcal{F}_s] = \E[W_t-W_s|\mathcal{F}_s] + \E[W_s|\mathcal{F}_s] = \E[W_t-W_s] + W_s = W_s$.
 \end{itemize}
 \end{samepage}

\end{example}
\begin{example}
 \label{ex:angleWt}
 Process $(W_t^2-t)_t$ is a martingale with respect to $W$'s natural filtration $\mathcal{F}$. Of course $(W_t^2-t) \sim \mathcal{F}_t$. From
 \[ \E|W_t^2-t| \leq \E(|W_t^2| + |t|) = \E(W_t^2 + t) = 2t\]
 we have $\E|W_t^2-t| < \infty$. Moreover, for any $t > s$
 \begin{equation*}
  \begin{split}
       & \E[W_t^2-t|\mathcal{F}_s] = \E[(W_t-W_s)^2 + 2W_tW_s - W_s^2|\mathcal{F}_s] -t =\\ 
    =\ & \E[(W_t-W_s)^2|\mathcal{F}_s] + \E[2W_tW_s|\mathcal{F}_s] - \E[W_s^2|\mathcal{F}_s] -t = \\
    =\ & \E[(W_t-W_s)^2] + 2W_s\E[W_t|\mathcal{F}_s] - W_s^2 -t = \\
    =\ & \E[(W_t-W_s)^2] + 2W_s^2 - W_s^2 -t = W_s^2 - s.
  \end{split}
 \end{equation*}
\end{example}

\begin{mydef}
 A stochastic process $M=(M_t)_{t=0}^T$ is called a \textbf{supermartingale} with respect to the filtration $(\mathcal{F}_t)_{t=0}^T$ if and only if it satisfies following conditions:
 \begin{itemize}
  \item for all $t \in [0,T],\ M_t$ is $\mathcal{F}_t$-measurable,
  \item for all $t \in [0,T],\ \E|M_t| < \infty$,
  \item for all $0 \leq s < t \leq T,\  M_s \geq \E[M_t|\mathcal{F}_s]$  a.s.
 \end{itemize}
 $M$ is called a \textbf{submartingale} if it is satisfies the above conditions, but with an inequality $ M_s \leq \E[M_t|\mathcal{F}_s]$.
\end{mydef}
\begin{example}
 For $0 \leq s < t \leq T$ we have
 \[ \E[W_t^2|\mathcal{F}_s] > \E[W_t^2|\mathcal{F}_s] - t + s = \E[W_t^2 - t|\mathcal{F}_s] + s = W_s^2, \]
 hence $W_t^2$ is a submartingale.
\end{example}


The breakthrough which initiated a rapid development of the financial mathematics was the discovery of It\^{o} integral, named after Japanese mathematician Kiyoshi It\^{o}. We do not discuss construction of such integral, what is done for example in \cite{latala}.
Instead we provide some intuition about its meaning. It\^{o} integral of $X=(X)_{t=0}^T$ with respect to Wiener process $W=(W)_{t=0}^T$ is a stochastic process
\[ \left( \int_0^{t} X_s dW_s \right)_{t = 0}^T, \]
where
\begin{equation}
 \label{eq:ito_integral}
  \int_0^{t} X_s dW_s = \lim_{n \rightarrow \infty} \sum\limits_{i=1}^n X_{t_{i-1}} (W_{t_i} - W_{t_{i-1}}).
\end{equation}
and for each $n$, $\{t_i\}_{i=0}^n$ is a partition of the interval $[0,t]$, whose diameter converges to 0 as $n$ tends to infinity.
As we can see It\^{o} integral is similar to the Stieltjes integral, which is defined for $g$ with locally bounded variation, and continuous $f$, as
\[  \int_0^{t} f(s) dg(s) = \lim_{n \rightarrow \infty} \sum\limits_{i=1}^n f(t_{i-1}) (g(t_i) - g(t_{i-1})). \]
However, on almost all trajectories $W$ has locally unbounded variation. In order to give sense to equation (\ref{eq:ito_integral}), the limit has to be taken in $L_2$.

\begin{remark} Note that
\begin{itemize}
 \item to integrate $\int\limits_0^t X_s ds$ we need only ``standard'' Riemann theory. The integrand is random, but $\int\limits_0^t X_s ds$ can be treated as a function $\omega \mapsto \int\limits_0^t X_s(\omega) ds$ and the last term is a Riemann integral of a function $X(\omega)$.
 \item $\int\limits_0^t X_s dW_s$ denotes a stochastic integral, thus its calculation requires It\^{o} theory.
\end{itemize}
\end{remark}


\begin{mydef}
\label{def:SDE}
 Let $\mu, \sigma \in C^1,\ \xi$ be a $\mathcal{F}_s$-measurable random variable. We say that process $X=(X_t)_{t=0}^T$ solves a \textbf{stochastic differential equation (SDE)}
 \begin{equation*}
 \begin{split}
   dX_t &= \mu(X_t)dt + \sigma(X_t) dW_t,\\
   X_0 &= \xi  
 \end{split}  
 \end{equation*}
 if and only if
 \[X_t = \xi + \int\limits_0^t \mu(X_s)ds + \int\limits_0^t\sigma(X_s) dW_s\]
for all $t \in [0,T)$.
\end{mydef}
\begin{remark}
 Sometimes indices are omitted and SDE is written in the form
\end{remark}
\[ dX = \mu(X)dt + \sigma(X) dW. \]

Next definition presents one of the most important type of processes, used to model asset movements in real-world markets.
\begin{mydef}
 A stochastic process $S$ given by SDE
 \begin{equation}
  dS_t = \mu S_t dt + \sigma S_t dW_t, 
  \label{eq:gbm}
 \end{equation}
where $\mu,\sigma \in \R$ is called a \textbf{geometric Brownian motion}.
\end{mydef}

Now we can formulate a version of It\^{o}'s lemma, which is widely used in financial mathematics.
\begin{thm}[It\^{o}'s lemma]
 \label{thm:ito}
  Let $S$ be a geometric Brownian motion as in (\ref{eq:gbm}), $F:\ \R^2 \rightarrow \R,\ F \in C^2$. Then 
  \begin{equation*}
   F(S_t, t) = F(S_0, 0) + \int\limits_0^t \frac{\partial F}{\partial S}(S_r,r)dS_r + \int\limits_0^t \frac{\partial F}{\partial t}(S_r,r)dr + \frac{1}{2}\sigma^2 S_t^2 \int\limits_0^t \frac{\partial^2 F}{\partial S^2}(S_r,r)dr
  \end{equation*}
  or equivalently in SDE form
  \begin{equation}
   \label{eq:ito}
   dF(S_t, t) = \frac{\partial F}{\partial S}(S_t,t)dS_t + \frac{\partial F}{\partial t}(S_t,t)dt + \frac{1}{2}\sigma^2 S_t^2 \frac{\partial^2 F}{\partial S^2}(S_t,t)dt   .
  \end{equation}  
\end{thm}

\noindent In the books on the stochastic processes It\^{o}'s lemma is proven in much greater generality. However, for our purposes, as in many other literature on financial mathematics, the formulated theorem is sufficient.
Equation (\ref{eq:ito}) is also called an It\^{o}'s formula.

It\^{o}'s lemma is a very powerful tool, indispensable in Black-Scholes theory. We show how it can be used to solve equation (\ref{eq:gbm}).
\begin{prop}
\label{prop:solution_dynamics}
 Let $S$ be a geometric Brownian motion, as in (\ref{eq:gbm}).
 Solution of its SDE is given by
 \begin{equation}
  \label{eq:gmb_sol}
  S_t = S_0 \exp\left\{ (\mu - \frac{1}{2}\sigma^2)t + \sigma W_t \right\}.
 \end{equation}
\end{prop}
\begin{proof}
We apply It\^{o}'s formula with $F(S,t) = \ln(S)$. We have
 \begin{equation*}
  \begin{split}
   dF &= \frac{\partial F}{\partial S}dS + \frac{\partial F}{\partial t}dt + \frac{1}{2}\sigma^2 S^2 \frac{\partial^2 F}{\partial S^2}dt \\
   &= \frac{\partial F}{\partial S}(\mu S dt + \sigma S dW) + \frac{\partial F}{\partial t}dt + \frac{1}{2}\sigma^2 S^2 \frac{\partial^2 F}{\partial S^2}dt \\    
   &= \frac{1}{S}(\mu S dt + \sigma S dW) + 0dt - \frac{1}{2}\sigma^2 S^2 \frac{1}{S^2}dt \\   
   &= (\mu - \frac{1}{2}\sigma^2) dt + \sigma dW
  \end{split}.
 \end{equation*}
 From Definition \ref{def:SDE}
 \begin{equation*}
  \begin{split}
  F(S_t,t) &= F_0 + \int\limits_0^t (\mu - \frac{1}{2}\sigma^2) ds + \int\limits_0^t \sigma dW_s\\
  &= F_0 + (\mu - \frac{1}{2}\sigma^2)t +  \sigma W_t. 
  \end{split}.
 \end{equation*}
 By substituting $S_0 = e^{F_0}$, we get
 \[ S = S_0 \exp\left\{ (\mu - \frac{1}{2}\sigma^2)t + \sigma W_t \right\}. \] 
\end{proof}

So far, we have discussed only one dimensional stochastic processes. However, often we have to take into account several assets at once, thus we need mathematical tools to describe multidimensional cases.
\begin{mydef}
 Process $W$ is called \textbf{$d$-dimensional standard Wiener process} if it is a vector process (its values are in $\mathbb{R}^d$) of the form
 \[ W = \left[ \begin{array}{c}
         W^{(1)}\\
         W^{(2)}\\
         \vdots\\
         W^{(d)}
        \end{array} \right],\]
where all components are independent Wiener processes.
\end{mydef}
Movements of asset prices usually cannot be regarded as independent. We need notation to describe dependency between processes.
\begin{mydef}
 We say that Wiener processes $W$ and $V$ are correlated and have correlation $\varrho$, what we denote by
 \[ \Corr(W, V) = \varrho, \]
 if and only if
 \[ \Corr(W_1, V_1) = \varrho. \]
\end{mydef}
\begin{remark}
 It is easy to notice that $\Corr(W_1, V_1) = \varrho$ if and only if for all $t > 0$, $\Corr(W_t, V_t) = \varrho$.
\end{remark}
\begin{mydef}
 The process $W$ is called \textbf{$d$-dimensional correlated Wiener process} with correlation matrix $\Sigma = (\varrho_{ij})_{i,j=1}^d$, if and only if $W = [W^{(1)},W^{(2)},\ldots,W^{(d)}]'$, where for each $i,j$, $\Corr(W^{(i)}, W^{(j)}) = \varrho_{ij}$.   
\end{mydef}

We end this section with the Girsanov theorem, whose significance in the financial mathematics arises from the fact, that it allows us to move from the real measure $\P$ to an equivalent martingale measure $\Pm$. In the literature exist many versions of this theorem. The one presented here comes from \cite{bjork}, there Reader may also find the proof.
\begin{thm}[\bfseries Girsanov theorem]
 \label{thm:girsanov}
 Let $\bar{W}$ be a $d$-dimensional standard Wiener process on the probability space $(\Omega, \mathcal{F}, (\mathcal{F}_t)_{t=0}^T, \P)$ and let $\varphi$ be any $d$-dimensional adapted column vector process.
 Choose a fixed $T$ and define the process $L$ on $[0,T]$ by
 \[ L_t = \exp\left\{ \int\limits_0^t \varphi_s \cdot d\bar{W}_s - \frac{1}{2}\int\limits_0^t ||\varphi_s||^2ds. \right\} \]
 Assume that 
 \[ \E^P[L_T] = 1, \]
 and define the new probability measure $\mathbb{Q}$ on $\mathcal{F}_T$ by
 \[ \frac{d\mathbb{Q}}{d\P} = L_T,\ \ \hbox{ on } \mathcal{F}_T. \]
 Then $W$ given by
 \[W_t = \bar{W}_t - \int\limits_0^t \varphi_s ds\]
 is a standard Wiener process under $\mathbb{Q}$.
\end{thm}
\begin{remark}
 Symbol $\cdot$ in definition of $L$ is the inner product of two vectors. 
\end{remark}

\section{Introduction to Monte Carlo methods}
\label{sec:introMC}
Monte Carlo methods are a class of algorithms designed for estimation of unknown values by simulation.
They do not refer to any particular algorithm, they are rather a general ``recipe'' for procedures, which obtain results by simulation.

\subsection{Crude Monte Carlo}
Suppose that we want to estimate an unknown value $I$, which can be written as expected value of some random variable, i.e.
\begin{equation}
 \label{eq:EY}
 I = \E Y. 
\end{equation}
The idea of Monte Carlo technique is to replicate $Y$ many times, and as an estimation of $I$ take an average. So
\[ I \approx \frac{1}{n} \sum\limits_{i=1}^n Y_i, \]
where $n$ is a big natural number and $Y_i$ are independent, with the same distribution as~$Y$.

This procedure is justified by the strong law of large numbers. Let 
\begin{equation}
 \label{eq:CMC}
 \CMC[n] = \frac{1}{n}\sum\limits_{i=1}^n Y_i.
\end{equation}
Of course $\E\CMC[n] = \E Y = I$, so $\CMC[n]$ is unbiased. Moreover, from the law of large numbers $\CMC[n] \conv I$ a.s., what explains why Monte Carlo method works. 

Value $\CMC[n]$ is called \textbf{crude Monte Carlo (CMC)} estimator. Simple calculation gives its variance
\begin{equation}
 \label{eq:VarCMC}
 \Var(\CMC[n]) = \frac{1}{n}\Var(Y).
\end{equation}
In this paragraph we simply write $\hat{Y}_n$ instead of $\CMC[n]$.

Here appears a natural question, how big should be the number $n$ to obtain the satisfying accuracy of the estimator?
To find an answer we have to specify the question a little more: for chosen numbers $b$ and $\alpha$, how big should be $n$, so we could tell, that an error of the estimation, with probability $1-\alpha$, is not greater than $b$?

Let $\sigma = \sqrt{\Var{Y}}$, $z_{1-\alpha/2} = \Phi^{-1}(1-\alpha/2)$, where $\Phi$ is the cumulative distribution function of the normal distribution.
Strong convergence of $\hat{Y}_n$ implies also weak convergence, what means that
\[ \P(|\hat{Y}_n - I| > b) \conv 0 \hbox{\ \ a.s.,} \]
for any $b > 0$. 
Hence we write
\begin{align*}
 \P(-b \leq \hat{Y}_n - I \leq b) &= 1 - \alpha\\
 \P(-b \leq \frac{\sum\limits_{i=1}^n Y_i - nI}{n}  \leq b) &= 1 - \alpha\\
 \P(-\frac{b\sqrt{n}}{\sigma} \leq \frac{\sum\limits_{i=1}^n Y_i - nI}{\sqrt{n}\sigma}  \leq \frac{b\sqrt{n}}{\sigma}) &= 1 - \alpha
\end{align*}
From Central Limit Theorem we know that
\[ \limn \P(-z_{1-\alpha/2} \leq \frac{\sum\limits_{i=1}^n Y_i - nI}{\sqrt{n}\sigma}  \leq z_{1-\alpha/2}) = 1 - \alpha, \]
hence for large $n$ we have
\[z_{1-\alpha/2} \approx \frac{b\sqrt{n}}{\sigma}.\]
In a typical situation we do not know the variation of $Y$ (we do not even know the expected value, after all we are using Monte Carlo method to find it!), so the above formula has rather theoretical meaning.
Instead, we have to use an unbiased estimator
\[ \hat{\sigma} = \frac{1}{n-1}\sum\limits_{i=1}^n (Y_i - \hat{Y}_n)^2 \]
and replace $\sigma$ by $\hat{\sigma}$ in above approximation. The foregoing discussion may be considered a sketch of a proof of following
\begin{thm}
 Dependency between a number of simulations $n$, an error $b$ and a confidence level $\alpha$ is given by following formulas
 \begin{equation}
   \label{eq:error}
   b = \frac{\hat{\sigma} z_{1-\alpha/2}}{\sqrt{n}}
 \end{equation}
 \begin{equation}
   \label{eq:sim}
   n = \frac{\hat{\sigma}^2 z_{1-\alpha/2}^2}{b^2}.
 \end{equation}
\end{thm}
\noindent Equation (\ref{eq:error}) tells us how big is an error of the estimation when we performed $n$ simulations. Equation (\ref{eq:sim}) inverses the situation, it allows us to plan the number of simulations necessary to obtain requested accuracy.

Unfortunately equation (\ref{eq:error}) tells us that convergence of the Monte Carlo method is slow. To improve accuracy by one more digit, one have to perform 100 times more simulations. The only way to decrease the number of necessary simulations is to choose $Y$ with the smallest possible variance. In next paragraphs, two methods of variance reduction are discussed.

The confidence level $\alpha$, which appears in (\ref{eq:error}) in the quantile function, is not essential when comparing two estimators. Hence we introduce 
\begin{mydef}
 Value
 \begin{equation}
  \label{eq:stderr}
  \frac{\hat{\sigma}}{\sqrt{n}}
 \end{equation}
is called \textbf{standard error} (abbreviated \textbf{s.e.}).
\end{mydef}
Suppose that in some fixed time we can take $n$ samples from the distribution of $Y$, and $m$ samples from the distribution of $Z$, where $EY = EZ = I$. In order to settle which estimator is better, $\hat{Y}_n$ or $\hat{Z}_m$, it is sufficient to compare their standard errors, that is $\dfrac{\hat{\sigma}_Y}{\sqrt{n}}$ and $\dfrac{\hat{\sigma}_Z}{\sqrt{m}}$.

Since $z_{0.975} \approx 2$, equation (\ref{eq:error}) tells us that with \textbf{95\% confidence the difference between the real value and the result of the estimation does not exceed two standard errors.}

\subsection{Antithetic variates}
Let us consider again $I$ and $Y$ as in (\ref{eq:EY}). Equation (\ref{eq:sim}) shows that the number of simulations required to obtain given accuracy is proportional to variance of $Y$. It explains the necessity of choosing $Y$ wisely. If we can find $Y'$ which has smaller variance than $Y$, then we can significantly decrease the number of needed simulations.
We describe two techniques of variance reduction: this paragraph introduces antithetic variates method, and the following presents control variates method.

In \textbf{antithetic variates method (AV method)} every sample is a pair of values, each from the same distribution as $Y$. Every of $n$ sample pairs is independent from each other, however random variables in a pair should be correlated. In the other words we consider pairs $(Y_{2i-1}, Y_{2i})$, $i=1,2,...,n$,
where each $(Y_{2i-1}, Y_{2i})$ is independent from $(Y_{2j-1}, Y_{2j})$, if $i \neq j$, and for some $\varrho$, $\Corr(Y_{2i-1}, Y_{2i}) = \varrho$. Let
\begin{equation*}
 \tilde{Y}_i = \frac{Y_{2i-1} + Y_{2i}}{2},\ \ \ i = 1,2,...,n.
\end{equation*}
It is clear that $(\tilde{Y}_i)_{i=1}^n$ are i.i.d., and their variance satisfies
\begin{equation*}
 \begin{split}
 \Var(\tilde{Y}_i) &= \dfrac{1}{4} \Var( Y_{2i-1} + Y_{2i} ) = \dfrac{1}{4} (\Var(Y_{2i-1}) + \Var(Y_{2i}) + 2\cdot\Cov(Y_{2i-1}, Y_{2i})) \\
 %&= \dfrac{1}{4} (\Var(Y_{2i-1}) + \Var(Y_{2i}) + 2\Corr(Y_{2i-1}, Y_{2i})\sqrt{\Var(Y_{2i-1})\Var(Y_{2i})})
 &= \dfrac{1}{4} (2\cdot\Var(Y) + 2\cdot\Var(Y)\varrho) \\
 &= \dfrac{\Var(Y)}{2} (1 + \varrho).
 \end{split}
\end{equation*}
We define \textbf{antithetic variates estimator} as
\begin{equation*}
 \AV[n] = \frac{1}{n}\sum\limits_{i=1}^n \tilde{Y}_i.
\end{equation*}
It is clear that such estimator is unbiased, i.e. $\E\AV[n] = I$. Let us calculate its variance
\begin{equation}
 \label{eq:VarAV}
 \begin{split}
 \Var(\AV[n]) &= \frac{1}{n^2} \Var(\sum\limits_{i=1}^n \tilde{Y}_i) \\
   &= \frac{1}{n} \Var(\tilde{Y}_i) = \dfrac{\Var(Y)}{2n} (1 + \varrho).
 \end{split}
\end{equation}
In order to calculate $\AV[n]$ we have to take $2n$ samples (or rather $n$ pairs of samples). From (\ref{eq:VarCMC}) we see that variance of the crude Monte Carlo estimator, which performs the same number of draws, equals $\frac{1}{2n}\Var(Y)$. Hence, if correlation of random variables in a pair is negative, then we reduce variance. In consequence the number of simulations necessary to keep an error smaller than $b$, at the confidence level $\alpha$ is also smaller.

\subsection{Control variates}
The \textbf{control variates method (CV method)} also involves drawing pairs of values, however in opposite to antithetic variates method, elements in pair do not come from the same distribution and expected value of the second distribution must be known. More precisely, we consider pairs $(Y_i, X_i)$, $i=1,2,...,n$, where each $(Y_i, X_i)$ is independent from $(Y_j, X_j)$,
if $i \neq j$, $\E X$ is known and $|\Cov(Y_i, X_i)| > 0$. Let $\hat{X}_n = \frac{1}{n}\sum_{i=1}^n X_i$. \textbf{Control variates estimator} is defined as
\begin{equation}
 \label{eq:CV}
 \CV[n] = \CMC[n] + c(\hat{X}_n - \E X)
\end{equation}
for some $c$. Clearly, $\CV[n]$ is unbiased. In order to reduce variance, $c$ must be chosen properly. We have
\begin{equation*}
 \begin{split}
 \Var( \CV[n] ) &= \Var( \CMC[n] + c\hat{X}_n ) = \frac{1}{n^2} \Var \left( \sum\limits_{i=1}^n( Y_i + c X_i) \right) \\
                &= \frac{1}{n}\Var(Y + c X) = \frac{1}{n}( \Var(Y) + 2c\Cov(Y,X) + c^2 \Var(X)).
 \end{split}
\end{equation*}
The last expression is a simple quadratic equation with respect to $c$, hence it is easy to determine for which argument it reaches its minimum value:
\[ c = -\frac{\Cov(Y,X)}{\Var(X)}. \]
Let $\varrho = \Corr(Y,X)$. By substituting $c$ to the last equation we obtain
\begin{equation}
 \begin{split}
 \Var( \CV[n] ) &=  \frac{1}{n} \left( \Var(Y) - 2\frac{\Cov(Y,X)^2}{\Var(X)} + \frac{\Cov(Y,X)^2}{\Var(X)^2} \Var(X) \right) \\
 &= \frac{1}{n} \left( \Var(Y) - \frac{\Cov(Y,X)^2}{\Var(X)}  \right) \\
 &= \frac{\Var(Y)}{n} \left( 1 - \varrho^2  \right) \label{eq:VarCV}
 \end{split}
\end{equation}
Crude Monte Carlo estimator, which takes the same number of random variables (i.e. 2 times the number of pairs in control variates method) has the variance equal to $\frac{\Var(Y)}{2n}$. Thus if $1 - \varrho^2 < \frac{1}{2}$ we reduce the variance.

In practice, however, we do not know values $\Var(X)$ (we only assumed we know expectation of $X$) and $\Cov(Y,X)$. Hence in the simulations we have to use 
\begin{equation}
 \label{eq:CVc}
 c = -\frac{\sigma_{XY}^2}{\sigma_{X}^2},
\end{equation}
where
\begin{equation*}
 \begin{split}
  \sigma_{XY}^2 &= \frac{1}{n-1} \sum\limits_{i=1}^{n} (X_i - \hat{X}_n)(Y_i - \CMC[n]),\\
  \sigma_{X}^2 &= \frac{1}{n-1} \sum\limits_{i=1}^{n} (X_i - \hat{X}_n)^2.
 \end{split}
\end{equation*}

\begin{example}
To compare presented Monte Carlo methods let us calculate value of $I = \int_0^1 e^x dx$ by simulation. Of course exact value equals $e - 1 \approx 1.71828183$. Let $g(x) = e^x$ and $U \sim \mathcal{U}(0,1)$. Then 
\[ I = \E[g(U)], \]
 hence we can use derived theory with $Y = g(U)$. We consider following estimators:
 \begin{equation*}
  \begin{split}
   \CMC[2n] &= \frac{1}{2n} \sum\limits_{i=1}^{2n} g(U_i), \\
   \AV[n] &= \frac{1}{n} \sum\limits_{i=1}^{n} \frac{g(U_i) + g(1-U_i)}{2}, \\ 
   \CV[n] &= \frac{1}{n} \sum\limits_{i=1}^{n} \left( g(U_i) + c(U_i - \frac{1}{2}) \right),
  \end{split}
 \end{equation*}
where $c$ is as in (\ref{eq:CVc}) with $Y = g(U)$ and $X = U$. Note that we are taking twice as much simulations in crude Monte Carlo method, since it does not use pairs of random variables. The results of comparison of all presented methods are gathered in Table \ref{tab:MCcompare} and Figure \ref{fig:boxMC}.

\begin{table}
\centering
 \caption{Results of calculating $\int_0^1 e^x dx$ by simulation.}
 \label{tab:MCcompare}
\begin{tabular} {||c | c | c | c | c |c | c ||}  
 \hline 
  & \multicolumn{2}{|c|}{ CMC } & \multicolumn{2}{|c|}{ AV } & \multicolumn{2}{|c|}{ CV } \\
  $\log_{10}(n)$ & \multicolumn{1}{c}{ $\CMC[2n]$ } & \multicolumn{1}{c|}{ s.e. } & \multicolumn{1}{c}{ $\AV[n]$ } & \multicolumn{1}{c|}{ s.e. } & \multicolumn{1}{c}{ $\CV[n]$ } & \multicolumn{1}{c|}{ s.e. } \\ \hline \hline 
  2 & 1.69825 & 0.03504 & 1.71717 & 0.00648 & 1.71962 & 0.00638 \\ \hline 
  3 & 1.72458 & 0.01094 & 1.72171 & 0.00205 & 1.72032 & 0.00203 \\ \hline 
  4 & 1.72116 & 0.00349 & 1.71805 & 0.00062 & 1.71918 & 0.00063 \\ \hline 
  5 & 1.71665 & 0.00110 & 1.71802 & 0.00020 & 1.71844 & 0.00020 \\ \hline 
  6 & 1.71796 & 0.00035 & 1.71831 & 0.00006 & 1.71829 & 0.00006 \\ \hline 
\end{tabular}  
\end{table}

\begin{figure}
\centering
 \includegraphics[scale=0.5]{images/Preliminaries/boxMonteCarlo.pdf}
 \includegraphics[scale=0.5]{images/Preliminaries/convergenceMC.pdf}
\caption{The upper picture presents a ``result cloud''. For each method 10000 simulations were run 100 times. Hence each method gave 100 estimations of $\int_0^1 e^x dx$. Small points indicate obtained values. As usually in box plots, the lower and upper edges of the boxes are first and third quartiles. In the lower picture we see the comparison of the speed of convergence. }
\label{fig:boxMC}
\end{figure}

% \begin{figure}
% \centering
%  \includegraphics[scale=0.6]{images/Preliminaries/convergenceMC.pdf}
% \caption{Comparison of the speed of convergence. }
% \label{fig:convergenceMC}
% \end{figure}

We see that in this case antithetic and control variates methods gave approximately equal results, while crude Monte Carlo is far behind them. When performing one million simulations (in case of CMC two millions) standard error turned out 50 times smaller in AV and CV than in CMC. It means that results from the first two methods are more than one digit more accurate.

The box plot in Figure \ref{fig:boxMC} shows that a variance of the crude Monte Carlo estimator is much more greater than in the two other methods. In practice it means, that if we ran simulations with AV estimator or CV estimator several time, then every time we would obtain more or less equal result. For CMC the discrepancy between the results would be much greater.
The lower illustration explains that fact -- convergence of CMC is definitely slower than two other methods.

In order to realize why AV and CV methods lead to so similar accuracy, look at (\ref{eq:VarAV}) and (\ref{eq:VarCV}). In AV method variance is reduced by a coefficient
\[ \frac{1 + \Corr(e^U, e^{1-U})}{2} \approx \frac{1 - 0.968}{2} = 0.0162, \]
while in CV method variance is reduced by a coefficient
\[ 1 - \Corr(e^U, U)^2 \approx 1 - 0.992^2 = 0.0163. \]
Such similarity of the results is pure coincidence.

Let us finish this example with a remark that this was a preconceived case, which showed that variance reduction techniques may be very useful. Actually, in many applications it may be difficult to find appropriate pairs of highly correlated random variables necessary to use AV or CV methods.
\end{example}


\subsection{Simulation}
Now, when we know how Monte Carlo methods work, we need to describe how to get random values.
\paragraph{Independent standard normal random variables.} Most computational environments and programming languages have built-in generator of values from uniform distribution. We show how to obtain (pseudo)random variables with the ubiquitous normal distribution.

The most popular way is to use the \textbf{Box-Muller transformation} (Algorithm \ref{alg:box-muller}) which uses two independent variates with the uniform distribution on $(0,1)$, and ``produces'' two independent variates with the standard normal distribution.
\begin{algorithm}
 \begin{algorithmic}[1]
  \Function{BoxMuller}{\null}
    \State Generate independent random variables $U_1, U_2 \sim \mathcal{U}(0,1)$,
    \State $Z_1 \gets \sqrt{-2\log(U_1)} \cos(2\pi U_2)$,
    \State $Z_2 \gets \sqrt{-2\log(U_1)} \sin(2\pi U_2)$,
    \State \Return ($Z_1$, $Z_2$).
  \EndFunction
 \end{algorithmic}
 \caption{The Box-Muller method.}
 \label{alg:box-muller}
\end{algorithm}

This method returns values of two independent random variables coming from the standard normal distribution. When we need more than two samples, then of course we repeat that algorithm as many times as necessary.

In \cite{london} we can find a remark that necessity of calculating sine and cosine may slow down the above algorithm. Another method, \textbf{polar rejection}, is proposed (Algorithm \ref{alg:polarRejection}).
\begin{algorithm}
 \begin{algorithmic}[1]
  \Function{PolarRejection}{$\mu$, $\sigma$}
    \State Generate independent random variables $U_1, U_2 \sim \mathcal{U}(0,1)$,
    \State $V_1 \gets 2U_1-1$,
    \State $V_2 \gets 2U_2-1$,
    \State $W \gets V_1^2 + V_2^2$,
    \State if $W > 1$ return to step 2.,
    \State $Z_1 \gets \sqrt{\frac{-2\log(W)}{W}} V_1$,
    \State $Z_2 \gets \sqrt{\frac{-2\log(W)}{W}} V_2$,
    \State \Return ($Z_1$, $Z_2$).
  \EndFunction
 \end{algorithmic}
 \caption{Polar rejection method.}
 \label{alg:polarRejection}
\end{algorithm}
However, conducted experiments do not show any significant difference in efficiency of both presented methods. It brings us to the conclusion that the efficiency of both algorithms is comparable and not worth bothering. Moreover, which method is faster may depend even on the choice of the programming language.

\paragraph{Independent non-standard normal random variables.} Sampling from an arbitrary normal distribution $\mathcal{N}(\mu, \sigma^2)$ is now straightforward. From the elementary probability theory comes the method presented in Algorithm \ref{alg:gaussDist}.
\begin{algorithm}
 \begin{algorithmic}[1]
  \Function{GaussDistr}{\null}
    \State $(Z_1, Z_2) \gets $ \Call{BoxMuller}{\null}
    \Comment{{\color{comment} or $(Z_1, Z_2) \gets $ \Call{PolarRejection}{\null}}}
    \State \Return $\mu$ + $\sigma \cdot Z_1$
  \EndFunction
 \end{algorithmic}
 \caption{Drawing a random variable from the normal distribution $\mathcal{N}(\mu, \sigma^2)$.}
 \label{alg:gaussDist}
\end{algorithm}

\paragraph{Correlated normal random values.} The above methods allow us to generate independent random variables. In the real world, however, we notice dependencies between observed phenomena. For example movements of prices of the market assets are usually correlated. When the market is in a boom cycle, then all the prices are increasing; on the other hand, if the recession comes, all of the prices are sinking down.
In order to model such values we must be able to generate correlated random variables.

\begin{figure}
\centering
 \includegraphics[scale=0.5]{images/Preliminaries/normalCorrelated.pdf}
\caption{Graphical illustration of the dependency between two normal correlated random variables $Z1$ and $Z2$. In the upper left corner $\varrho = -0.9$, upper right: $\varrho = -0.5$, lower left: $\varrho = 0$ (independent variables), lower right: $\varrho = 0.5$.}
\end{figure}

Let us begin with two standard normal random variables with correlation $\varrho$. We want to obtain $N_1, N_3 \sim \mathcal{N}(0,1)$ such that $\Corr(N_1, N_3) = \varrho$.  For now we can only generate independent $N_1$ and $N_2$, i.e. $\Corr(N_1, N_2) = 0$. On the other hand $\Corr(N_1, N_1) = 1$. Here rises the idea that there exists $N_3$ of the form $N_3 = a N_1 + b N_2$, such that $\Corr(N_1, N_3) = \varrho$.
It should come from the standard normal distribution, thus
\[ 1 = \Var(N_3) = \Var( a N_1) +  \Var(b N_2) = a^2 + b^2 \]
Moreover
\[ \varrho = \Corr(N_1, N_3) = \Cov(N_1, N_3) = a\Cov(N_1, N_1) + b\Cov(N_1, N_2) = a \]
Hence
\[ a = \varrho,\ \ \ \ b = \sqrt{1 - \varrho^2}. \]
We have proven following
\begin{prop}
 \label{prop:corrNorm}
 If $Z_1$ and $Z_2$ are independent random variables with the same distribution $\mathcal{N}(0,1)$, then for any $\varrho \in [-1,1]$ random variables
 \begin{align*}
 N_1 &= Z_1 \\
 N_2 &= \varrho Z_1 + \sqrt{1 - \varrho^2} Z_2
 \end{align*}
 have distribution $\mathcal{N}(0,1)$ and $\Corr(N_1, N_2) = \varrho$.
\end{prop}

\begin{remark}
 Let $\Sigma = \left( \begin{array}{cc}
                    1    & \varrho \\
                    \varrho & 1
                    \end{array} \right),$
  and $L = \left( \begin{array}{cc}
                    1    & 0 \\
                    \varrho & \sqrt{1 - \varrho^2}
                    \end{array} \right).$
 Clearly, vector $N = (N_1, N_2)'$ from Proposition \ref{prop:corrNorm} has the Gaussian distribution $\mathcal{N}(0, \Sigma)$. Moreover, $N = LZ$. There is an interesting correspondence between $\Sigma$ and $L$ -- simple matrix multiplication shows that $\Sigma = LL'$.  
\end{remark}

The derivation of the above fact was intuitive, but we need a more general version. It turns out that Proposition \ref{prop:corrNorm} can be generalized for higher dimensions.
\begin{thm}
 For the positive-definite matrix $\Sigma$ there exists a lower triangular matrix $L$, such that $\Sigma = LL'$. This representation of $\Sigma$ is called the \textbf{Cholesky decomposition}.
\end{thm}
Matrix $L$ from the above theorem may be obtained by the \textbf{Cholesky algorithm}. Utility of the Cholesky decomposition arises from the following fact.
\begin{prop}
 Let $Z$ be a random vector with the multivariate normal distribution $\mathcal{N}(0, \mathbbm{I})$. Let $\Sigma$ be a co variance matrix and $\Sigma = LL'$. Then vector $N = \mu + LZ$ has the distribution $\mathcal{N}(\mu, \Sigma)$.
\end{prop}
\begin{proof}
 It is clear that $N$ has the multivariate normal distribution. Moreover,
 \[ \E N = \E(\mu + LZ) = \mu + L\cdot\E Z = \mu, \]
 \begin{equation*}
  \begin{split}
 \Cov(N) &= \E \bigl( (N - \E N) (N - \E N)' \bigr) = \E \bigl( (LZ - L\cdot \E Z) (LZ - L\cdot \E Z)' \bigr) =\\
         &= \E \bigl( L(Z - \E Z) (Z - \E Z)'L' \bigr) = L\cdot \E \bigl( (Z - \E Z) (Z - \E Z)' \bigr) \cdot L' =\\
         &= L\cdot \Cov(Z) \cdot L'= LL' = \Sigma.   
  \end{split}
 \end{equation*}
\end{proof}


That explains how works Algorithm \ref{alg:gaussMulti} for generating values from the multivariate normal distribution with given mean vector $\mu$ and positive-definite covariation matrix $\Sigma$. 
\begin{enumerate}
 \item Use Cholesky decomposition to obtain matrix $L$ such that $\Sigma = LL^T$,
 \item Using previously shown method generate vector $N$ of independent random variables with standard normal distribution.
 \item Return $\mu + LN$.
\end{enumerate}
\begin{algorithm}
 \begin{algorithmic}[1]
  \Function{GaussMultiVariate}{$\mu$, $\Sigma$}
    \State $L \gets $ \Call{CholeskyAlgorithm}{$\Sigma$}
    \State $Z \gets $ vector of independent standard normal random variables.
    \State \Comment{{\color{comment} $Z$ is obtained from \Call{BoxMuller}{} or \Call{PolarRejection}{}}}
    \State \Return $\mu$ + $\Sigma \cdot Z$
  \EndFunction
 \end{algorithmic}
 \caption{The polar rejection method.}
 \label{alg:gaussMulti}
\end{algorithm}


\chapter{Basics of option pricing}
One of the main fields of interest in the financial mathematics is pricing so-called contingent claims, which are financial instruments whose value depends on some other assets. In case of the European options there exists a straightforward formula, derived by Black and Scholes in 1973, which gives the option's price. Another way of pricing is using Monte Carlo simulations.
Although the simulations are time-consuming and for that reason less effective than Black-Scholes formula, they are widely used because of the possibility to adjust them to very sophisticated contingent claims. 

\section{Elements of arbitrage theory}
Mathematical approach to the analysis of assets' prices requires modelling the prices as stochastic processes. This section describes fundamental concepts. The notation and the description of the market model presented here is based on \cite{follmer}.

\subsection{Underlying framework}
Throughout this thesis we assume we are given a probability space $(\Omega, \mathcal{F}, (\mathcal{F}_t)_{t=0}^T, \P)$. Since it is observed by investors, $\P$ is sometimes called a \textbf{real measure}, in opposite to an artificial \textbf{risk-neutral measure}, which is later on. Elements of $\Omega$ are called \textbf{market scenarios}.
Furthermore $\sigma$-algebra $\mathcal{F}_t$ may be seen as a set of all events observable up to time $t$.

We consider a market with $d+1$ assets, where each asset $S^{(i)} = (S^{(i)}_t)_{t=0}^T$ is modelled as a $\R_+$-valued stochastic process adapted to $(\mathcal{F}_t)_{t=0}^T$. The $0$\textsuperscript{th} asset is the money stored in a locally riskless bank account and is given by 
\[S^{(0)}_t = \exp\left\{ \int\limits_0^t r(t)dt \right\},\]
where $r(t)$ is a short term riskless interest rate at time $t$. In general $r(t)$ may also be a stochastic process, however, in this thesis we focus only on the case $r(t) \equiv r$ and then $S^{(0)}$ may also be regarded as a bond, paying $e^{rT}$ at time $T$. By $S$ we denote a $d$-dimensional (column) vector of prices of the risky assets, i.e.
\begin{equation*}
 S_t = (S^{(1)}_t, S^{(2)}_t, \ldots, S^{(d)}_t)', \ \ \ 0 \leq t \leq T.
\end{equation*}
We also introduce notation $\Sa$ for a $(d+1)$-dimensional vector of prices of all assets, that is
\begin{equation*}
 \Sa_t = (S^{(0)}_t, S_t)' = (S^{(0)}_t, S^{(1)}_t, \ldots, S^{(d)}_t)', \ \ \ 0 \leq t \leq T.
\end{equation*}

For convenience we also consider discounted time processes
\[ X^{(i)}_t = \frac{S^{(i)}_t}{S^{(0)}_t}. \]
It allows us to compare asset prices quoted at different times. In similar manner as previously we use notation $X, \Xa$ for vectors of the discounted prices,
\begin{equation*}
 X_t = (X^{(1)}_t, X^{(2)}_t, \ldots, X^{(d)}_t)', \ \ \ 0 \leq t \leq T.
\end{equation*}
\begin{equation*}
 \Xa_t = (X^{(0)}_t, X^{(1)}_t, \ldots, X^{(d)}_t)', \ \ \ 0 \leq t \leq T.
\end{equation*}

Next definition gives us a notation to describe the content of \textbf{portfolio}.
\begin{mydef}
A \textbf{dynamic trading strategy} is any $\mathcal{F}_t$-adapted,  $\R^{d+1}$-valued process $\xia = (\xia_t)_{t=0}^T = (\xi^{(0)}_t, \xi_t)_{t=0}^T = (\xi^{(0)}_t, \xi^{(1)}_t, \ldots, \xi^{(d)}_t)_{t=0}^T$.
\end{mydef}
Each $\xi^{(i)}_t$ has an interpretation of the quantity of shares of the $i$\textsuperscript{th} asset held in portfolio at time $t$ (it may be negative, then it corresponds to the short sale). Thus, the notion of the portfolio and the dynamic trading strategy may be equated. The value of the portfolio at time $t$ equals
\[\xia_t \cdot \Sa_t = \sum\limits_{i=0}^d \xi^{(i)}_t S^{(i)}_t,\]
where $\cdot$ is the inner product of two vectors.

Assume that investor's portfolio is worth 101,000\$ today and a year ago it was worth 100,000\$. When was the investor's financial condition better, today or a year ago? If the riskless interest rate equals 5\%, then the investor would gain more if he put all his capital into a bank account. This example shows the necessity of discounting to compare portfolios whose values are quoted at different times.
\begin{mydef}
 The \textbf{discounted value process} $V^{\xi} = (V^{\xi}_t)_{t=0}^T$ associated with a trading strategy $\xia$ is given by 
 \begin{equation*}
  V^{\xi}_t = \xia_t \cdot \Xa_t.
 \end{equation*}
\end{mydef}

\subsection{Arbitrage opportunities}
The value $V^{\xi}_0$ is the initial investment into the portfolio. Following definition introduces portfolios which do not receive cash flows from the ``outside world'' after initialization. They are rearranged in such a way that purchases of new assets must be covered by selling some other assets, so value of the portfolio before and after rearranging stays the same.
\begin{mydef}
 A dynamic trading strategy is called \textbf{self-financing} if and only if for every $0 \leq t \leq T$
 \[ \sum\limits_{i=0}^d d\xi^{(i)}_t (S^{(i)}_t + dS^{(i)}_t) = 0. \]
\end{mydef}
This definition, although being very simple, looks completely incomprehensible at first sight. To convey some intuition let us think that $dt$ is a very small, even infinitesimal, time period. Vector $\xia_t$ describes content of the portfolio at the beginning of the period. After time $dt$ change of the stock prices equals $d\Sa_t$. The portfolio needs a rearrangement -- $d\xia_t$ represents changes of quantities of the held assets.
Thus $d\xia_t \cdot (\Sa_t + d\Sa_t)$ is the total cost of the rearrangement. From the definition it equals 0, what explains the name \textit{self-financing}.

From now on we consider only those market models that are efficient in the sense that they are arbitrage-free.
\begin{mydef}
 An \textbf{arbitrage opportunity} is a self-financing portfolio $\xia$, such that
 \begin{align*}
  V^{\xi}_0 &= 0\\
  \P(V^{\xi}_T \geq 0) &= 1\\
  \P(V^{\xi}_T > 0 ) &> 0
 \end{align*}
 The market is \textbf{arbitrage-free} if it does not allow for arbitrage opportunities.
\end{mydef}
What does arbitrage opportunity mean in practice? Assume an investor entering the market without any capital. He builds his portfolio by short sale of some assets and purchase of other assets for received money. Arbitrage opportunity is a situation when the investor at time $T$ can cover his short positions by sale of assets held long, and with positive probability he has some remaining cash.
In the other words, he can make money without exposure to any downside risk. 

Now we are moving to an important concept of a martingale measure.
\begin{mydef}
 A probability measure $\Pm$ is called a \textbf{martingale measure} (or a \textbf{risk neutral measure}) if and only if the discounted price process $X$ is a $\Pm$-martingale, i.e. for all  $0 \leq s \leq t \leq T$
 \begin{equation}
  \label{eq:X-martingale}
  \Em[X_t] < \infty \hbox{\ \ and\ \ } \Em[X_t | \mathcal{F}_s] = X_s
 \end{equation}
\end{mydef}

\begin{remark}
 Condition (\ref{eq:X-martingale}) is written for a vector process $X$, hence for every $1 \leq i \leq d$
  \begin{equation*}
  \Em[X^{(i)}_t] < \infty \hbox{\ \ and\ \ } \Em[X^{(i)}_t | \mathcal{F}_s] = X^{(i)}_s
 \end{equation*}
\end{remark}

Let us recall that two measures $\P$ and $\Pm$ defined on $\sigma$-algebra $\mathcal{F}$ are equivalent if and only if for all $A \in \mathcal{F}$, $\P(A)=0$ if and only if $\Pm(A)=0$. Next theorem, known as the \textbf{first fundamental theorem of asset pricing}, shows the importance of martingale measures.
\begin{thm}[\bfseries First FTAP]
 \label{thm:fftap}
 The following to statements are \emph{essentially} equivalent:
 \begin{enumerate}
  \item The market model is arbitrage free.
  \item There exists martingale measure $\Pm$ equivalent to $\P$.
 \end{enumerate}
\end{thm}
Unfortunately, due to an appearance of a word ``essentially, this is rather a ``meta-theorem''. It can be given a sharp, mathematical sense -- see for example Theorems 10.9 and 10.10 in \cite{bjork}. In the discrete case, however, theorem holds without the word ``essentially, as it is proven in \cite{follmer} (Theorem 5.17). This case is sufficient for us, since in Monte Carlo methods price trajectories are simulated only in a finite number of points.

\section{European contingent claims}
\label{sec:ECC}
We start this section with a definition of a mentioned contingent claim.
\begin{mydef}
 \label{def:cc_eu}
 An \textbf{European contingent claim} is a non-negative random variable $C$ on $(\Omega, \mathcal{F}_T, \P)$. A \textbf{derivative} of the underlying assets $S^{(1)}, \ldots, S^{(d)}$ is a contingent claim which is measurable with respect to $\sigma$-algebra generated by price processes.
\end{mydef}
European contingent claims may be seen as assets yielding a random payoff at the \textbf{expiration date} $T$ (also called \textbf{maturity}). Of course the seller of such contingent claim cannot take a random amount of money from the buyer. What should be then the price at time $0$? The answer to this question is in general definitely non-trivial. However, for some simple derivatives, e.g. European options, there exists a straightforward formula for the price.

\subsection{Examples of derivatives}
\begin{mydef}
 An \textbf{European call option} on the asset $S^{(i)}$ with expiration date $T$ and \textbf{strike price} $E$ gives its owner the right, but not the obligation, to \underline{buy} that asset at time $T$ for price $E$.
 
 An \textbf{European put option} on the asset $S^{(i)}$ with expiration date $T$ and \textbf{strike price} $E$ gives its owner the right, but not the obligation, to \underline{sell} that asset at time $T$ for price $E$.
\end{mydef}
From the definition
\begin{equation}
 \label{eq:ecall}
 C\textsuperscript{call} = (S^{(i)}_T - E)_+
\end{equation}
\begin{equation}
 \label{eq:eput}
 C\textsuperscript{put} = (E - S^{(i)}_T )_+
\end{equation}
where $(x)_+ = \max(0,x)$.

Options defined above are also called \textbf{vanilla options}, while derivatives with additional features are called \textbf{exotic options}. \textbf{Barrier options} may serve as an example. Their payoff depends not only on the stock price at the maturity, but also on the historical prices. Barrier options are divided into \textit{knock-in} options, which may be ``turned on'', and \textit{knock-out} options, which may be ``turned off'' in case of reaching some \textbf{barrier}. Let us write down two example definitions.
\begin{mydef}
The payoff of the \textbf{up-and-in call} option on asset $S^{(i)}$ with expiration date $T$, strike price $E$ and barrier $\overline{B}$, equals
\[ C^{\hbox{\scriptsize call}}_{\hbox{\scriptsize u\&i}} = 
\begin{cases}
 (S^{(i)}_T - E)_+    & \hbox{if } \sup\limits_{0 \leq t \leq T} S^{(i)}_t \geq \overline{B}\\
 0                    & \hbox{otherwise.}
\end{cases}
\]
\end{mydef}
\noindent The payoff of the knock-out option is zeroed when the barrier is hit.
\begin{mydef}
The payoff of the \textbf{down-and-out put} option on asset $S^{(i)}$ with expiration date $T$, strike price $E$ and barrier $\underline{B}$, equals
\[ C^{\hbox{\scriptsize put}}_{\hbox{\scriptsize d\&o}} = 
\begin{cases}
 (E - S^{(i)}_T)_+    & \hbox{if } \inf\limits_{0 \leq t \leq T} S^{(i)}_t \leq \underline{B}\\
 0                    & \hbox{otherwise.}
\end{cases}
\]
\end{mydef}
We have eight types of barrier options, as each of them is call or put, up or down, in or out. They are all defined in a similar manner. The best way to get a grip on the barrier options is possibly through a graphical example. Figure \ref{fig:barrier} discusses payoffs from an up-and-out call option in three different scenarios. 
\begin{figure}
\centering
 \includegraphics[scale=0.6]{images/BasicsOfOptionPricing/barrier.pdf}
\caption{Consider an up-and-out call option with strike 100, barrier 130, expiring at time 1. In the red scenario the option is in-the-money at the maturity, however, in the past the barrier was crossed, thus payoff is 0. In the blue scenario stock price ends about the level 113, the barrier was not reached, hence the payoff equals 13. In case of the green scenario the payoff is 0, as it would be for vanilla option, because the option ended out-of-the-money. }
\label{fig:barrier}
\end{figure}

Another modification of the vanilla options are the \textbf{Asian options}. Their payoff depends on an average price of the asset during options lifetime.
\begin{mydef}
 Payoffs of Asian call and put options on asset $S^{(i)}$ with exercise date $T$ and strike price $E$, are given by
\begin{align*}
 C^{\text{call}}_{\text{asian}} &= (\bar{S}^{(i)}_T - E)_+, \\
 C^{\text{put}}_{\text{asian}} &= (E - \bar{S}^{(i)}_T)_+, \\
\end{align*}
\end{mydef}
\noindent where
\[ \bar{S}^{(i)}_T = \frac{1}{K} \sum\limits_{j=0}^K S^{(i)}_{j\cdot T/K}, \]
for some $K$. 

So far we presented only options on one asset. Multi-asset options are also in usage, for example \textbf{basket options} are options on the value $\eta \cdot S$. Vector $\eta$ describes quantity of shares of the assets contained in a basket. The basket call option gives us right to buy whole set of assets for the specified price, and the basket put allows us to sell it.
\begin{mydef}
 Payoffs of basket call and put options on basket $\eta$ with exercise date $T$ and strike price $E$, are given by
\begin{align*}
 C^{\text{call}}_{\eta} &= (\eta \cdot S_T - E)_+, \\
 C^{\text{put}}_{\eta} &= (E - \eta \cdot S_T)_+, \\
\end{align*}
\end{mydef}

\subsection{Motivation for the option usage}

\paragraph{Why do we need derivatives in the first place?}
The well known anecdote claims that the first man who used derivatives was Tales of Miletus. His skills allowed him to predict that the olive harvest next year will be extraordinarily large. In the winter, when nobody needed olive presses, he reserved them for summer. During the season demand for the olive presses increased and Tales rented them for a good price. After all, it turned out that he earned much more then paid for the reservation. From this story appears the first reason to use options: they give opportunity to make money on accurate predictions.

Second, and probably more important reason of option usage, is possibility to hedge against inconvenient scenarios. Consider a producer making his articles from some raw material. The cost of his production depends on the price of this material. If it goes too high, then the factory may become unprofitable. By buying the options, the producer may ensure that the cost of the production will not exceed above known level. If the price of the raw material stays low, then the options expire worthless, but the producer stays content, because the production is not endangered. If asset price peaks, he can exercise the options. Either way he wins.


\paragraph{What is the purpose of exotic options?}
Altering the rules of ``typical'' payoffs may be caused by many reasons. The seller of a call option puts himself into a risk, induced by the fact, that his maximum possible loss is unbounded. Thus, he may be interested in entering only contracts with up-and-out barrier, which prevents too large payoffs. The buyer of the call option may want to hedge himself against high prices.
He may purchase cheaper version of the option, with down-and-out  barrier -- maybe if the asset price is low, he does not need any extra hedge. Asian options may be a good choice for risk-averse investors, because they are less sensitive to changes in the underlying price, especially in the time close to the maturity.

Both described stories show that derivatives idea arises in a natural way. In both stories there was an exchange of the money for some goods. In the real live, however, situation is not always that clear. Sometimes, even if the investor exercise his option on some commodity, there is no real transaction performed, only the difference between prices is paid off.
It may seem that the derivatives are artificial tools, created by people living only in the theoretical, mathematical models. Some exotic options may be really complicated and at the first glance no one can tell what was the motivation of entering such contract. However, it is worth to remember, that many of the derivative contracts are designed by investors, economist or producers, and they correspond to their real needs. Mathematical models lend a hand in pricing such contracts.


\section{Black-Scholes model}
\label{sec:blackScholes}
So far we discussed a very general market model. In order to obtain some specific results, we have to make several more assumptions.
In this section we recall famous Black-Scholes model, leading to a straightforward formula for the price of the European options.

\subsection{One-asset model}
At first we discuss a case when $d=1$, i.e. the model has only one risky asset and a bond. For convenience we write $S$ instead of $S^{(1)}$, and $B$ instead of $S^{(0)}$. We use this convention every time when considering a market with one risky asset.

The assumptions are following:
\begin{enumerate}
  \item[BS1.] \textbf{The market does not admit arbitrage opportunities.}

 \item[BS2.] \textbf{The stock price of the underlying follows a geometric Brownian motion.} Moreover drift $\mu$ and volatility $\sigma$ are constant in time. Thus, SDE of the option price is described by equation
\begin{equation}
 \label{eq:BS_dynamics}
 dS_t = \mu S_t dt + \sigma S_t dW_t. 
\end{equation}
 
  \item[BS3.] \textbf{There exists constant risk-free interest rate $r$.} In the other words, dynamics of $B$ is given by 
\[ dB_t = rB_t dt. \]
Investors may both, borrow and lend, any amount of money at rate $r$.

  \item[BS4.] \textbf{It is possible to buy and sell any amount of stock.} It means that investors can even trade fractional numbers of stock, and sell short unbounded quantity of shares.

  \item[BS5.] \textbf{All transactions do not incur any additional costs.}

  \item[BS6.] \textbf{The underlying does not pay a dividend.}
  
  \item[BS7.] We have to make one additional assumption on the options value. Many authors forget to mention it, although it is required for It\^{o}'s lemma. Let $F$ be the process of the option's price\footnotemark.
  \textbf{For some smooth function $\varphi$ the price process has the form:}
    \begin{equation}
    \label{eq:PriceProcessForm}
    F_t = \varphi(S_t, t).
    \end{equation}
This assumption looks entirely natural, but it cannot be concluded from what we discussed so far. It must be taken as an axiom.
\end{enumerate}
\footnotetext{$F_t$ is the value of the option quoted at time $t$ (not discounted to time 0). The discounted price process is in this thesis denoted by letter $V$. }

Instead of (\ref{eq:PriceProcessForm}) we write $F_t = F(S_t, t)$. Then $F$ has an ambiguous meaning -- left $F$ denotes the price process and the right one is some function. However, as in many other literature, we identify them, since it does not lead to misunderstanding.

Suppose that we are constructing a portfolio consisting of a short position in one option and a long position in $\Delta$ shares. Let $\Pi$ be the value process of the portfolio. Equation of the portfolio is given by
\begin{equation}
 \label{eq:portfolio}
  \Pi = \Delta S - F 
\end{equation}
We analyse how much the portfolio changes in a short period of time. We have
\[ d\Pi = \Delta dS - dF  \]
Note that we do not have to differentiate $\Delta$, because it is constant in an infinitesimal increment of time. To handle $dF$ we use It\^{o}'s lemma. Thus
\[ d\Pi = \Delta dS - \frac{\partial F}{\partial S}dS - \frac{\partial F}{\partial t}dt - \frac{1}{2}\sigma^2 S^2 \frac{\partial^2 F}{\partial S^2}dt  \]
The risk in an increment of the portfolio's value is caused by changes of the stock price. By choosing
\begin{equation}
 \label{eq:delta}
 \Delta = \frac{\partial F}{\partial S}
\end{equation}
we get rid off the uncertainty. Now we have
\begin{equation}
  \label{eq:portfolio_inc}
 d\Pi = -(\frac{\partial F}{\partial t} + \frac{1}{2}\sigma^2 S^2 \frac{\partial^2 F}{\partial S^2})dt.
\end{equation}
The increment of $\Pi$ does not depend on any risky asset, hence no-arbitrage assumption induces
\[ d\Pi = r\Pi dt. \]
After substituting (\ref{eq:portfolio}), (\ref{eq:delta}) and (\ref{eq:portfolio_inc}) into above equation, we obtain
\[ -(\frac{\partial F}{\partial t} + \frac{1}{2}\sigma^2 S^2 \frac{\partial^2 F}{\partial S^2})dt = r(\frac{\partial F}{\partial S} S - F)dt, \]
which after simple calculation gives
\begin{equation}
 \label{eq:BSeq}
 \frac{\partial F}{\partial t} + \frac{1}{2}\sigma^2 S^2 \frac{\partial^2 F}{\partial S^2} + r\frac{\partial F}{\partial S} S - rF = 0.
\end{equation}
Formula (\ref{eq:BSeq}) is known as the \textbf{Black-Scholes equation}. Note that so far we did not say anything about the final condition of (\ref{eq:BSeq}), i.e. about the payoff off the option, thus this is a general equation. For European options it has straightforward solution, known as \textbf{Black-Scholes formula}.
\begin{prop}
\label{prop:BSFormula}
Prices of the European call and put options with time to the expiration $T$, strike price $E$ and the underlying dynamics given by (\ref{eq:BS_dynamics}) are given by the following equations:
\begin{equation*}
  \begin{split}
    F_t\textsuperscript{call} &= S_t \Phi\bigl( d_1(t) \bigr) - e^{-r(T-t)} E \Phi\bigl(d_2(t) \bigr),\\
    F_t\textsuperscript{put} &= -S_t \Phi\bigl(-d_1(t) \bigr) + e^{-r(T-t)} E \Phi\bigl(-d_2(t) \bigr), 
  \end{split}
\end{equation*}
where
\begin{align*}
d_1(t) &= \frac{\ln\left(\frac{S_t}{E}\right)+\left(r+\frac{\sigma^{2}}{2}\right)(T-t)}{\sigma\sqrt{T-t}}\\
d_2(t) &= \frac{\ln\left(\frac{S_t}{E}\right)+\left(r-\frac{\sigma^{2}}{2}\right)(T-t)}{\sigma\sqrt{T-t}} = d_{1}(t)-\sigma\sqrt{T-t},\\
\Phi & \hbox{ is the distribution function of the standard normal distribution. } 
\end{align*}
\end{prop}
We prove this proposition at the end of the section \ref{sec:risk-neutral}, using the risk-neutral measure.

\subsection{Multi-asset model.}
All assumptions for a one-asset model carry over almost instantly to a multi-asset model with just a little edition:
\begin{enumerate}
 \item[BS2.] \textbf{Stock prices of all risky assets follow a geometric Brownian motion.} Each risky asset $S^{(i)}$ has constant drift $\mu_i$ and volatility $\sigma_i$. In symbolic form:
\begin{equation}
 \label{eq:BS_multi_dynamics}
 dS^{(i)}_t = \mu S^{(i)}_t dt + \sigma S^{(i)}_t dW^{(i)}_t\ \ \ \ (i=1,2,\ldots,d). 
\end{equation}
  \item[BS3.]  The dynamics of $S^{(0)}$ is given by 
\[ dS^{(0)}_t = rS^{(0)}_t dt. \]
\end{enumerate}
  
The movements of the asset prices are usually not independent.
\begin{mydef}
 We say that the correlation between two risky assets $S^{(i)}$ and $S^{(j)}$ equals $\varrho_{ij}$ if and only if $\Corr(W^{(i)}, W^{(j)}) = \varrho_{ij}$.
\end{mydef}
In the other words by the correlation of two assets we understand the correlation between corresponding Wiener processes, appearing in their dynamics. A matrix of the correlation between all risky processes is denoted by $\Sigma$,
\begin{equation*}
 \Sigma = \left( \begin{array}{cccc}
           \varrho_{11} & \varrho_{12} & \cdots & \varrho_{1d} \\
           \varrho_{21} & \varrho_{22} & \cdots & \varrho_{2d} \\
           \vdots & \vdots & \ddots & \vdots \\
           \varrho_{d1} & \varrho_{d2} & \cdots & \varrho_{dd} \\
          \end{array} \right).
\end{equation*}

\textbf{The assumptions described in this section hold true to the rest of the thesis.}

\section{Model calibration}
By looking at the previous section we can specify a set of values by which the model is parameterized:
\begin{itemize}
 \item drifts $\mu_1, \mu_2, \ldots, \mu_d$,
 \item volatilities $\sigma_1, \sigma_2, \ldots, \sigma_d$,
 \item correlation $\Sigma$,
 \item riskless interest rate $r$.
\end{itemize}
While calibrating a model it is convenient to assume that today is time $0$. We are interested in modelling asset prices in the future, up to time $T$. It is natural to use negative $t$ to denote times in the past. For example $S^{(i)}_{-0.5}$ means the price of the $i$\textsuperscript{th} asset half of a year ago. Hence for $t > 0,\ S_t$ is random vector, which we are about to model, and for $t \leq 0,\ S_t$ is a vector with historical prices, which may be obtained from the stock archives.

Since asset prices follow (\ref{eq:BS_multi_dynamics}) and its solution is given by (\ref{eq:gmb_sol}), thus for all $i$
\[  S^{(i)}_{t_{n+1}} = S^{(i)}_{t_n} \exp\left\{ (\mu_i - \frac{1}{2}\sigma_i^2)\Delta t + \sigma_i \sqrt{\Delta t} Z_i \right\}, \]
where $\Delta t = t_{n+1} - t_n$, $Z$ is a random vector with correlation matrix $\Sigma = \bigl( \varrho_{ij} \bigr)_{i,j=1}^d$, and for each $Z_i$, $Z_i \sim \mathcal{N}(0,1)$. Let
\[  L^{(i)}_{n+1} = \ln\left( \frac{S^{(i)}_{t_{n+1}}}{S^{(i)}_{t_n}} \right). \]
It is clear that $ L^{(i)}_{n+1} \sim \mathcal{N}\left(  (\mu_i - \frac{1}{2}\sigma_i^2)\Delta t,\ \sigma_i^2 \Delta t \right)$

The key to the calibration is an assumption that in the past, price processes followed (\ref{eq:BS_multi_dynamics}) as well. Let $\overline{T}$ denote how old is the oldest price observation, and $N$ be such that $N \cdot \Delta t = \overline{T}$. Furthermore let $t_k = -(N-k)\Delta t\ (k=0,1,\ldots,N)$. Values $t_k$ are times in which we take historical prices, from $t_0 = -\overline{T}$, to $t_N = 0$, which is today. Let us focus on the $i$\textsuperscript{th} asset. We have $N+1$ historical prices:
\[  \bigl( S^{(i)}_{t_0}, S^{(i)}_{t_1}, \ldots,S^{(i)}_{t_N} \bigr), \]
from which we obtain a vector
\[ \bigl( L^{(i)}_{1}, L^{(i)}_{2}, \ldots, L^{(i)}_{N} \bigr) \]
with $N$ samples from the distribution $\mathcal{N}\left(  (\mu_i - \frac{1}{2}\sigma_i^2)\Delta t,\ \sigma_i^2 \Delta t \right)$.

In practice, usually $\Delta t = 1/252$, because there are about 252 working days in a year. However, it is debatable how long should be $\overline{T}$. 

Finally, we present how to obtain values of model parameters.
\paragraph{Drifts and volatilities.}
Let $L^{(i)} \sim \mathcal{N}\left(  (\mu_i - \frac{1}{2}\sigma_i^2)\Delta t,\ \sigma_i^2 \Delta t \right)$. It means that
\begin{equation*}
 \begin{split}
  \E\left( L^{(i)} \right) &= (\mu_i - \frac{1}{2}\sigma_i^2)\Delta t, \\
  \Var\left( L^{(i)} \right) &= \sigma_i^2 \Delta t.
 \end{split}
\end{equation*}
Hence
\begin{equation*}
 \begin{split}
  \mu_i  &= \frac{ 2\E\left( L^{(i)} \right) + \Var\left( L^{(i)} \right) }{2\Delta t}, \\
  \sigma_i^2 &= \frac{\Var\left( L^{(i)} \right)}{\Delta t}.
 \end{split}
\end{equation*}
Now we can use the sample vector to estimate expectation and variance. Let
\begin{equation*}
 \begin{split}
  \alpha_i &= \frac{1}{N}\sum\limits_{k=1}^{N} L^{(i)}_k\\
  \beta_i &= \frac{1}{N-1}\sum\limits_{k=1}^{N}(L^{(i)}_k - \alpha_i)^2
 \end{split}
\end{equation*}
Values $\alpha_i$ and $\beta_i$ are unbiased estimators of expectation and variance respectively. Thus we assign
\begin{equation*}
 \begin{split}
  \mu_i  &= \frac{ 2\alpha_i + \beta_i }{2\Delta t}, \\
  \sigma_i^2 &= \frac{\beta_i}{\Delta t}.
 \end{split}
\end{equation*}
\begin{remark}
 Calculating drifts is in many applications redundant. Note that there is no drift term in Black-Scholes formula nor equation. As it is shown in section \ref{sec:risk-neutral} also the dynamics in the martingale measure does not depend on the drift.
\end{remark}


\paragraph{Correlation.} At first recall that $\Corr(aX + b, cY + d) = \Corr(X,Y)$. Thus finding a correlation between $Z_i$ and $Z_j$ is equivalent to finding correlation between $L^{(i)}$ and $L^{(j)}$. We can do it using Pearson's estimator:

\[ \varrho_{i,j} = \frac{1}{N-1} \frac{\sum\limits_{k=1}^N(L^{(i)}_k - \alpha_i)(L^{(j)}_k - \alpha_j)}{\sqrt{\beta_i \beta_j}} \]

\paragraph{Riskless interest rate.}
In order to calculate the interest rate it is necessary to choose a bond whose maturity is close to the expiry of the valued option. The interest rate implied by that bond reflects well the real interest rate in the concerned time.

It is clear that
\[ C = Ne^{-rT}, \]
where $C$ is price of the bond, $N$ is its nominal value, $r$ is the interest rate implied by the bond, and $T$ is the maturity time. By simple transformation
\begin{equation*}
r = \dfrac{\ln(\frac{N}{C})}{T}.
\end{equation*}

\section{Pricing general contingent claims}
\label{sec:risk-neutral}
Black-Scholes formula allows us to price only European vanilla options. In this section we present a general method for pricing European contingent claims. 

\subsection{Risk neutral pricing}
\begin{mydef}
 The discounted value of the contingent claim $C$ is given by
 \begin{equation*}
  H = \frac{C}{S^{(0)}_T}.
 \end{equation*}
 Random variable $H$ is called a \textbf{discounted claim}.
\end{mydef}

Values $C$ and $H$ correspond to the payoff of an instrument. We need notation to talk about its price also before the expiration.
\begin{mydef}
 The (discounted) \textbf{price process} of the contingent claim is denoted by $V_t$. 
\end{mydef}

The next theorem is the key to defining prices of contingent claims. But first we define a class of claims for which valuation is pretty straightforward. 
\begin{mydef}
 A contingent claim is called \textbf{attainable} if there exists a self-financing trading strategy $\xia$ whose portfolio coincides with $C$ at the expiration, i.e.
 \[ C = \xia \cdot \Sa_T. \]
 The trading strategy $\xia$ is then called the \textbf{replicating strategy} for $C$.
\end{mydef}

\begin{thm}
\label{thm:pricesProcess_mtg}
 For every attainable discounted claim $H$ and for every equivalent martingale measure $\Pm$
 \[ \Em [H] < \infty. \]
 Moreover, for every replicating strategy $\xia$ its value process satisfies
 \begin{equation}
  V^\xi_t = \Em[H | \mathcal{F}_t] \hbox{ P-a.s., } 0 \leq t \leq T .
 \end{equation}
\end{thm}
Proof of this theorem reader may find in \cite{follmer} (Theorem 5.26).

Since there is no $\xia$ in term $\Em[H | \mathcal{F}_t]$, so value $V^\xi_t$ does not depend on choice of $\xia$. Note also that
\begin{equation*}
 \Em[H | \mathcal{F}_t]  = \Em[\xia \cdot \Xa_T | \mathcal{F}_t]  = \xia \cdot \Xa_t
\end{equation*}
Thus, value $\Em[H | \mathcal{F}_t]$ does not depend on the choice of $\Pm$. 
Since attainable discounted claim $H$ and its replicating strategy $\xia$ have the same payoff at time $T$, thus no-arbitrage assumption induces
\[ V_t = V^\xi_t,\ \ 0 \leq t \leq T. \]
Thus in particular it implies
\begin{coro}[\textbf{Risk neutral valuation formula}]
\label{coro:price_mth}
Let $H$ be a discounted attainable contingent claim and $V$ be its price process. Then
\begin{equation}
 \label{eq:price_mtg}
 V_0 = \Em[H].
\end{equation}
\end{coro}
\noindent This equation tells us that the \textbf{value of the option is an expectation of its discounted payoff} under the risk-neutral measure.

The above theorem suggests how price processes should look in general.
\begin{mydef}
 The (discounted) price process of the discounted claim $H$ is given by
 \begin{equation*}
  V_t = \Em[H | \mathcal{F}_t].
 \end{equation*}
 Such $V$ is a $\Pm$-martingale.
\end{mydef}
For general claims process $V$ depends on the choice of an equivalent martingale measure. However, it may be proven that the market model consisting of the discounted assets
\[(X^{(0)}, X^{(1)},\ldots,X^{(d)}, V)\]
is arbitrage-free, regardless of the choice of $\Pm$. In that sense every possible price process $V$ is equally good.

\subsection{Change of the measure}
The martingale measure allows us to write the option's value if a form of a concise formula, but so far we did not tell how to find it. It turns out that we do not really need the martingale measure itself. The only matter is how the asset's price process can be expressed under the risk neutral measure. 

In the literature there are many formulations of the theory which is presented here. However, instead of referring to any other authors, we will prove some facts which exactly match our needs.

\begin{lemma}
\label{lemma:measure_change}
 Suppose that $S$ is a $d$-dimensional stochastic process, where each $S^{(i)}$ follows the geometric Brownian motion, that is
 \begin{equation}
  \label{eq:dynamics_real}
  dS^{(i)} = \mu_i S^{(i)} dt + \sigma_i S^{(i)} d\bar{W}^{(i)},\ \ \ (i=1,2,\ldots,d)
 \end{equation}
 where each $\bar{W}^{(i)}$ is a Brownian motion under measure $\P$ and $\Corr(\bar{W}^{(i)}, \bar{W}^{(j)}) = \rho_{ij}$. For every vector $(\nu_1, \nu_2, \ldots, \nu_d)$ there exists an equivalent probability measure $\mathbb{Q}$, such that the equation (\ref{eq:dynamics_real}) may be rewritten in the form
  \begin{equation}
  dS^{(i)} = \nu_i S^{(i)} dt + \sigma_i S^{(i)} dW^{(i)},\ \ \ (i=1,2,\ldots,d)
 \end{equation}
 where each $W^{(i)}$ is a Brownian motion under the equivalent measure $\mathbb{Q}$ and\\ $\Corr^Q(W^{(i)}, W^{(j)}) = \rho_{ij}$.
\end{lemma}

\begin{proof}
 Let $\Sigma = (\rho_{ij})_{i,j=1}^d$ be a correlation matrix. Cholesky's algorithm allows us to decompose $\Sigma$ to the form
 \[ \Sigma = LL^T, \]
 where $L$ is lower triangular matrix. Hence, $\bar{W}$ may be written in the form
 \[ \bar{W} = L \bar{V}, \]
 where $\bar{V}$ is a standard $d$-dimensional Wiener process under measure $\P$. Let us apply Theorem \ref{thm:girsanov} (Girsanov theorem) with $\varphi := \theta = (\theta_1, \theta_2, \ldots, \theta_d)'$, where all $\theta_i$ are some constants. It implies that
 \[ V_t = \bar{V}_t - t\theta \]
 is a $d$-dimensional standard Wiener process under an equivalent measure $\mathbb{Q}$, which is defined as
 \[\frac{d\mathbb{Q}}{d\P} = \exp\left\{ \sum\limits_{i=1}^d \theta_i \bar{W}^{(i)}_T - \frac{T}{2} ||\theta||^2 \right\}.\]
 Let $W := LV$. Thus
 \begin{equation*}
  \begin{split}
   dS^{(i)} &= \mu_i S^{(i)} dt + \sigma_i S^{(i)} d\bar{W}^{(i)} \\
            &= \mu_i S^{(i)} dt + \sigma_i S^{(i)} d\left(\sum\limits_{k=1}^i l_{ik} \bar{V}^{(k)}\right) \\
            &= \mu_i S^{(i)} dt + \sigma_i S^{(i)} d\left(\sum\limits_{k=1}^i \bigl[ l_{ik}\theta_k t + l_{ik} V^{(k)} \bigr] \right) \\    
            &= \left[ \mu_i + \sigma_i \sum\limits_{k=1}^i l_{ik}\theta_k \right] S^{(i)}dt + \sigma_i S^{(i)} d\left( \sum\limits_{k=1}^i l_{ik} V^{(k)} \right) \\    
            &= \left[ \mu_i + \sigma_i \sum\limits_{k=1}^i l_{ik}\theta_k \right] S^{(i)}dt + \sigma_i S^{(i)} dW^{(i)}. \\      
  \end{split}
 \end{equation*}
 By substituting
 \begin{align*}
  \theta_1 &:= \frac{\nu_1 - \mu_1}{\sigma_1 l_{11}}\\
  \theta_i &:= \frac{\nu_i - \mu_i - \sigma_i \sum\limits_{k=1}^{i-1} l_{ik}\theta_k}{\sigma_il_{ii}}\ \ \ (i=2,3,\ldots,d)\\
 \end{align*}
 we get the thesis.
\end{proof}

The above lemma allows us to describe the vector of price processes in terms of some equivalent measures, however we need a very particular measure -- the martingale measure.

\begin{prop}
\label{prop:rn-dynamics}
 Under the real measure $\P$ the risky assets follow a geometric Brownian motion, as in equation (\ref{eq:dynamics_real}). There exists an equivalent martingale measure $\Pm$, such that the dynamics has the form
 \begin{equation}
  \label{eq:dynamics_neutral}
  dS^{(i)} = r S^{(i)} dt + \sigma_i S^{(i)} dW^{(i)}.
 \end{equation}
\end{prop}
\begin{proof}
 Lemma \ref{lemma:measure_change} states that there exists an equivalent measure $\Pm$ under which price processes are described by equation (\ref{eq:dynamics_neutral}). We show that it is martingale measure.
 From (\ref{eq:gmb_sol})
 \[ X_t = e^{-rt} S_t = S_0 e^{ -\frac{1}{2}\sigma^2 t + \sigma W_t }. \]
 Let $0 \leq s \leq t \leq T$. We have
 \begin{equation*}
  \begin{split}
   \Em[X_t | \mathcal{F}_s] &= \Em[ S_0 e^{ -\frac{1}{2}\sigma^2 t + \sigma W_t } | \mathcal{F}_s] \\
       &= S_0 e^{-\frac{1}{2}\sigma^2 t} \cdot \Em[ e^{  \sigma W_s} e^{ \sigma (W_t - W_s) } | \mathcal{F}_s] \\
       &= S_0 e^{-\frac{1}{2}\sigma^2 t + \sigma W_s} \cdot \Em[ e^{ \sigma (W_t - W_s) }] \\
       &= S_0 e^{-\frac{1}{2}\sigma^2 t + \sigma W_s}  e^{ \frac{1}{2}\sigma^2(t-s) } \\
       &= S_0 e^{-\frac{1}{2}\sigma^2 s + \sigma W_s} = X_s.
  \end{split}
 \end{equation*}
Hence $\Pm$ is an equivalent martingale measure.
\end{proof}

Propositions \ref{prop:rn-dynamics} and \ref{prop:solution_dynamics} have crucial meaning in our applications. They allow us to generate trajectories of the asset prices under the risk-neutral measure, which is essential in the Monte Carlo pricing. We show one more application of the risk-neutral pricing -- it can be used to derive Black-Scholes formula.
\begin{proof}[(Proof of Proposition \ref{prop:BSFormula})]
 From Proposition \ref{prop:solution_dynamics}
 \[ S_T = S_t \exp\bigl\{ (r - \frac{1}{2} \sigma^2)(T-t) + \sigma W_{T-t} \bigr\} = S_t e^Z, \]
 where $Z \sim \mathcal{N}\left((r - \frac{1}{2} \sigma^2)(T-t), (T-t)\sigma^2 \right)$ in the measure $\Pm$. Let $f_Z$ be the density of $Z$ and $S_t = s$, thus
 \begin{equation*}
  \begin{split}
    F_t\textsuperscript{call} &= e^{-r(T-t)} \Em\left[(S_T-E)_+ | \mathcal{F}_t \right] \\
    &= = e^{-r(T-t)} \Em\left[(S_T-E)_+ |S_t\right] \\
    &= e^{-r(T-t)} \Em\left[(s e^Z-E)_+ \right] \\
    &= = e^{-r(T-t)} \Em\left[(s e^Z-E) \cdot \mathbbm{1}\left({\textstyle Z \geq \ln\left(\frac{E}{s}\right)} \right) \right] \\
    &= e^{-r(T-t)} \int\limits_{\ln(E/s)}^\infty (se^z - E)f_Z(z) dz \\
    &= s\int\limits_{\ln(E/s)}^\infty e^{-r(T-t)}e^z f_Z(z) dz - e^{-r(T-t)} E\int\limits_{\ln(E/s)}^\infty f_Z(z) dz = (\bigstar)
  \end{split}
 \end{equation*}
 Simple calculation gives
 \[ \int\limits_{\ln(E/s)}^\infty e^{-r(T-t)}e^z f_Z(z) dz = \Phi\bigl( d_1(t) \bigr),  \ \ \ 
    \int\limits_{\ln(E/s)}^\infty f_Z(z) dz = \Phi\bigl( d_2(t) \bigr),  \ \ \ \]
 hence 
 \[ (\bigstar) = S_t \Phi\bigl( d_1(t) \bigr)  - e^{-r(T-t)} E \Phi\bigl( d_2(t) \bigr) . \]
 Derivation of the formula for put's price is analogous.
\end{proof}




\chapter[{Pricing European options using Monte Carlo method}]{Pricing European options using \\Monte Carlo method}
\label{chapter:European}
The Black-Scholes theory gives us compact formula for pricing European vanilla options. Such options gained popularity and are traded in many world markets. However, over the counter (ab. OTC) investors may trade much more complicated instruments, whose value cannot be derived analytically. Thus, other methods must be used.
The most popular are finite difference, binomial trees and Monte Carlo. In this thesis we present the last one.

\section{Vanilla options}
\label{sec:pricing_vanilla}

In order to use the Monte Carlo method in the option pricing, we need to involve the theory presented in the section \ref{sec:risk-neutral}. First we focus on the case, when the only instruments traded in the market are $B = S^{(0)}$ -- a riskless bank account, and a risky asset $S = S^{(1)}$.

As in previous chapter, $V_0$ is the option's price and $H$ is the discounted payoff. From Corollary \ref{coro:price_mth} we have
\begin{equation}
 \label{eq:price_for_MC}
 V_0 = \Em[H].
\end{equation}
By comparison with (\ref{eq:EY}), we see that equation (\ref{eq:price_for_MC}) is exactly what we need for simulations. To calculate options price we have to replicate its payoff many times and take the mean. Note, however, that expectation is taken under the risk-neutral measure. Hence, also the asset price must be generated under the risk-neutral measure. Corollary \ref{prop:rn-dynamics} describes its dynamics:
\[ dS = rSdt + \sigma S dW. \]
Proposition \ref{prop:solution_dynamics} gives the solution to above SDE:
\begin{equation}
 \label{eq:vanilla_St}
 S_t = S_0 \exp\left\{ (r - \frac{1}{2}\sigma^2)t + \sigma W_t \right\}.
\end{equation}
In case of vanilla options only the value at the end of the trajectory is important, i.e. at maturity time $T$. Thus, we need
\begin{equation}
\label{eq:vanilla_ST}
 S_T = S_0 \exp\left\{ (r - \frac{1}{2}\sigma^2)T + \sigma W_T \right\},
\end{equation}
where, from properties of the Wiener process, $W_T \sim \mathcal{N}(0,T)$. The value of $S_T$ depends on $W_T$, hence it is justified to treat $S_T$ as a function of $W_T$ and write $S_T = S_T(W_T)$.

Let $H = g(S_T)$ and $E$ be the strike price. For instance, if $H$ is a call option \linebreak ${g(x) = e^{-rT}(x-E)_+}$, and if $H$ is a put $g(x) = e^{-rT}(E - x)_+$, but in fact $H$ might be any claim whose payoff depends only on $S_T$. Equation (\ref{eq:vanilla_ST}) tells us how to generate the asset price; by applying function $g$ we generate the payoff.
Since $S_T$ is also a function of some $Z \sim \mathcal{N}(0,T)$, thus actually $H = g(S_T(Z)) =: f(Z)$, for $f = g \circ S_T$. By $Z_i, i = 1,2,...$, we denote replications of $Z$.
The crude Monte Carlo estimator has the form:
\begin{equation}
 \label{eq:vanilla_CMC}
 \CMCa[H, 2n] = \frac{1}{2n}\sum\limits_{i=1}^{2n} f(Z_i).
\end{equation}
We also use antithetic variates, where an antithetic variable to $f(Z_i)$ is  $f(-Z_i)$. 
\begin{equation}
 \label{eq:vanilla_AV}
 \AVa[H, n] = \frac{1}{n}\sum\limits_{i=1}^{n} \frac{f(Z_i) + f(-Z_i)}{2}.
\end{equation}
To use the control variates method, recall that $e^{rT} S_0 = \Em[S_T]$. It implies that we can take $S_T$ as a control variate, hence
\begin{equation}
 \label{eq:vanilla_CV}
 \CVa[H, n] = \frac{1}{n}\sum\limits_{i=1}^{n} \left( f(Z_i) + c (S_T(Z_i) - e^{rT}S_0) \right),
\end{equation}
where $c$ is a value calculated as in equation (\ref{eq:CVc}).

To get a grip on using above estimators in practice, we present how exactly looks pricing call options using the control variates method. It is shown in Algorithm \ref{alg:priceCallCV}. Argument of the algorithm is \texttt{n} -- number of simulations. 
\begin{algorithm}
 \begin{algorithmic}[1]
  \Function{PriceCallCV}{$n$, $S_0$, $\sigma$, $r$, $T$, $E$ }
    \State  $S,H,Y \gets $ arrays with indices from 1 to $n$.
    \For{$i = 1$ {\bf to} $n$} 
      \State $Z \gets$ generate standard normal
      \State $S[i] \gets S_0 \cdot \exp\{ (r - \frac{1}{2}\sigma^2 ) \cdot T + \sigma \cdot Z \}$
      \State $H[i] \gets \max(S - E, 0) \cdot \exp\{-rT\}$
    \EndFor
    \State $c \gets -\Cov(H,S)/\Var(S)$
    \For{$i = 1$ {\bf to} $n$} 
      \State $Y[i] \gets  H[i] + c\cdot \bigl(S[i] - S_0\cdot \exp\{rT\} \bigr)$
    \EndFor    
    \State $price \gets mean(Y)$
    \State $var \gets var(Z)$
    \State $se \gets \sqrt{var / n}$
    \State \Return $(price, var, se)$
  \EndFunction
 \end{algorithmic}
 \caption{Valuation of a call option using CV method.}
 \label{alg:priceCallCV}
\end{algorithm}
Values calculated in lines 12-14 are price of the option, variance and standard error of the estimation.

An implementation of an option pricer based on estimators (\ref{eq:vanilla_CMC})-(\ref{eq:vanilla_CV}) allows us to compare these methods. In sections \ref{sec:pricing_vanilla} and \ref{sec:pricing_complicated} we always assume following parameters:
\begin{equation}
\label{eq:marketParams}
\begin{split}
 S &= 100 \\
 \sigma &= 0.20 \\
 r &= 0.05 \\
 T &= 1 \hbox{\ \ \ (options expire after one year)}.
\end{split}
\end{equation}

\begin{example}
First consider a call option with strike 90. The Black-Scholes value of the option is 16.70. Results of the Monte Carlo pricing are shown in Table \ref{tab:vanilla1} and Figure \ref{fig:vanilla1}.

It is clear, that in this case CV method proved itself the best. It is caused by the fact, that in most simulations option expires in the money. In consequence the payoff is highly correlated with the asset price at the end of the path. 
\end{example}

\begin{table}
\centering
 \caption{Results of pricing call@90. Black-Scholes price is 16.70.}
 \label{tab:vanilla1}
\begin{tabular} {||c | c | c | c | c |c | c ||}  
 \hline 
  & \multicolumn{2}{|c|}{ CMC } & \multicolumn{2}{|c|}{ AV } & \multicolumn{2}{|c|}{ CV } \\
  $\log_{10}(n)$ & \multicolumn{1}{c}{ $\CMCa[H, 2n]$ } & \multicolumn{1}{c|}{ s.e. } & \multicolumn{1}{c}{ $\AVa[H, n]$ } & \multicolumn{1}{c|}{ s.e. } & \multicolumn{1}{c}{ $\CVa[H, n]$ } & \multicolumn{1}{c|}{ s.e. } \\ \hline \hline 
  3 & 17.09 & 0.393 & 16.67 & 0.185 & 16.74 & 0.127 \\ \hline 
  4 & 16.98 & 0.123 & 16.66 & 0.061 & 16.67 & 0.041 \\ \hline 
  5 & 16.71 & 0.039 & 16.67 & 0.019 & 16.70 & 0.013 \\ \hline 
  6 & 16.69 & 0.012 & 16.70 & 0.006 & 16.70 & 0.004 \\ \hline 
\end{tabular}  
\end{table}
\begin{figure}
\centering
 \includegraphics[scale=0.5]{images/PricingEuropean/boxCall90.pdf}
 \includegraphics[scale=0.5]{images/PricingEuropean/convergenceCall90.pdf}
\caption{The accuracy of pricing call@90. Box plot on the left was created by running estimation 100 times for each method, each estimation used 10000 replicated pairs. The chart on the right shows speed of the convergence, i.e. how the estimation changes as the number of performed replications increases. The horizontal line is the options value calculated form Black-Scholes formula. }
\label{fig:vanilla1}
\end{figure}

\newpage
\begin{example}
Let us consider now a call with higher strike, 130, whose Black-Scholes price equals 1.64. Look at the Table \ref{tab:vanilla2} and Figure \ref{fig:vanilla2}. 
 
This time the asset price usually ended above the options strike, what means that in most simulations payoff was 0. Thus, correlation between payoff and assets final price is low, and in consequence CV did not bring a significant improvement.
\end{example}

\begin{table}[!hl]
\centering
 \caption{Results of pricing call@130. Black-Scholes price is 1.64.}
 \label{tab:vanilla2}
\begin{tabular} {||c | c | c | c | c |c | c ||}  
 \hline 
  & \multicolumn{2}{|c|}{ CMC } & \multicolumn{2}{|c|}{ AV } & \multicolumn{2}{|c|}{ CV } \\
  $\log_{10}(n)$ & \multicolumn{1}{c}{ $\CMCa[H, 2n]$ } & \multicolumn{1}{c|}{ s.e. } & \multicolumn{1}{c}{ $\AVa[H, n]$ } & \multicolumn{1}{c|}{ s.e. } & \multicolumn{1}{c}{ $\CVa[H, n]$ } & \multicolumn{1}{c|}{ s.e. } \\ \hline \hline 
 3 & 1.67 & 0.135 & 1.56 & 0.126 & 1.67 & 0.147 \\ \hline 
 4 & 1.61 & 0.043 & 1.60 & 0.041 & 1.60 & 0.046 \\ \hline 
 5 & 1.63 & 0.014 & 1.64 & 0.013 & 1.64 & 0.015 \\ \hline 
 6 & 1.64 & 0.004 & 1.64 & 0.004 & 1.64 & 0.005 \\ \hline 
\end{tabular}  
\end{table}
\begin{figure}[!hl]
\centering
 \includegraphics[scale=0.5]{images/PricingEuropean/boxCall130.pdf}
 \includegraphics[scale=0.5]{images/PricingEuropean/convergenceCall130.pdf}
\caption{The accuracy of pricing call@130. Plots were created in the similar manner as in Figure \ref{fig:vanilla1}.}
\label{fig:vanilla2}
\end{figure}

\begin{remark}
 Box plot from Figure \ref{fig:vanilla2} shows that CMC estimator has smaller dispersion than AV and CV. That is because we compare $\CMCa[H, 2n]$ (index is $2n$) with $\AVa[H, n]$ and $\CVa[H, n]$. We do so, because then the number of used random variables is the same in each estimator.
 However, generating an antithetic variate or a control variate is often instantaneous. Hence, in many cases calculating $\CMCa[H, 2n]$ may take almost twice as much time. If in time $t$ we can compute  $\CMCa[H, a]$, $\AVa[H, b]$ and $\CVa[H, c]$, then $\CMCa[H, a]$ always\footnote{Of course we assume that antithetic variate is really antithetic, i.e. it is negatively correlated with the base variate, and control variate is not independent from the base variate.} has greater variance than $\AVa[H, b]$ and $\CVa[H, c]$.
\end{remark}

\newpage
\section{Path-dependent instruments}
\label{sec:pricing_complicated}

To price vanilla options it was sufficient to generate the asset price only at the maturity. However, there are contingent claims whose payoff depends on the entire history, for example barrier options or Asian options. Of course, we cannot generate the entire trajectory, since it has continuum points. Thus, values of the asset must be generated in a finite number $K$ of points. To approximate a continuous model we have to take sufficiently large $K$.

It follows from equation (\ref{eq:vanilla_St}) that
\begin{equation}
 \label{eq:priceChange}
 S_{t + {\Delta} t} = S_t \exp\left\{ (r - \frac{1}{2}\sigma^2)\Delta t + \sigma \sqrt{\Delta t} Z \right\},
\end{equation}
where $Z$ is standard normal. The above formula allows us to generate asset's prices in specified points step by step.

Figure \ref{fig:trajectories} shows a thousand of trajectories simulated accordingly to equation (\ref{eq:priceChange}). Note that the mean of the asset price at the final time is slightly above $S_0 = 100$. It corresponds to the fact that $\Em [S_T] = e^{rT}S_0$, which for parameters of the simulations equals $100\cdot e^{0.05} \approx 105.13$.

\begin{figure}[h]
\centering
 \includegraphics[scale=0.75]{images/PricingEuropean/trajectories50.pdf}
\caption{A thousand of simulated trajectories of the asset price, under parameters ${T=1},\ {\sigma=0.2}$, ${r=0.05},\ {S_0=100}$. Each path was generated in $K=50$ points. The darker is the area the greater is the concentration of trajectories.}
\label{fig:trajectories}
\end{figure}

In theory, the value of the discounted claim now has the form $H = g(S)$, for some $g$, i.e. $H$ is a function of the whole process $S$. As mentioned before, in order to make $H$ computable, we are forced to treat $H$ as a function of $S$ in limited number of points, that is $H = g\bigl(S_0, S_{\Delta t}, ..., S_T\bigr),\ T = K\cdot\Delta t$.

Further in this chapter we use the following notation. Path replications are denoted by $S_i\ (i=1,2,...$), and value $S_{i,t}$ means what was the asset price at time $t$, on $i$-th simulated path.

The CMC estimator is analogous to (\ref{eq:vanilla_CMC})
\begin{equation*}
 \CMCa[H, n] = \frac{1}{n}\sum\limits_{i=1}^n g\bigl(S_{i,0}, S_{i,\Delta t}, ..., S_{i,T}\bigr)
\end{equation*}
For example for Asian call option
\begin{equation*}
 \CMCa[H, n] = \frac{1}{n}\sum\limits_{i=1}^n \left( \frac{1}{K} \sum\limits_{j=0}^K S_{i,j \Delta t} - E \right)_+
\end{equation*}
and for a put option with down-and-out barrier $B$
\begin{equation*}
 \CMCa[H, n] = \frac{1}{n}\sum\limits_{i=1}^n \left( (E - S_{i,T})_+ \cdot \prod\limits_{j=0}^K \mathbbm{1}\bigl(S_{i,j \Delta t} \geq B\bigr) \right).
\end{equation*}
Adjusting above estimators to use control variates does not bring any difficulties. The asset price at the end of the path may still be the control variate, however, for many contingent claims a better control variate is the payoff from a similar vanilla option. For instance for Asian put option
\begin{equation*}
 \CVa[H, n] = \CMCa[H, n] + \frac{c}{n} \sum\limits_{i=1}^n \Bigl( (E - S_{i,T})_+ - \Em (E - S_T)_+ \Bigr),
\end{equation*}
where $c$ is calculated as in (\ref{eq:CVc}) and value $\Em (E - S_T)_+$ is known from the Black-Scholes formula.

In previous section while using AV method, we generated negatively correlated variables representing the asset price at the option's expiry. Since now, payoff depends on the whole process, thus we have to generate the antithetic trajectories. Algorithm \ref{alg:single-tr} presents a proper procedure. Figure \ref{fig:trajectoriesAnti} gives an idea how antithetic paths look like.
We omit exact formulas for $\AVa[H, n]$, because they are becoming lengthy and overly complicated. Analysing an example application of AV method should be sufficient to understand how it works in general. To this end, we present procedure of pricing Asian call options (Algorithm \ref{alg:priceAsianAV}).

\begin{figure}[!ht]
\centering
 \includegraphics[scale=0.55]{images/PricingEuropean/trajectoriesAnti.pdf}
\caption{One pair of antithetic trajectories generated by an implementation of Algorithm \ref{alg:single-tr}.}
\label{fig:trajectoriesAnti}
\end{figure}
\begin{algorithm}[!ht]
 \begin{algorithmic}[1]
  \Function{Trajectory}{$S_0$, $\sigma$, $r$, $T$, $K$}
  \State pos, neg $\gets$ arrays with indices from $0$ to $K$
  \State pos[0] $\gets$ neg[0] $\gets S_0$
  \State $dt \gets T/K$
  \For{$i=1$ {\bf to} $K$}
    \State $Z \gets$ generate standard normal
    \State pos[$i$] $\gets$ pos[$i-1$] $\cdot \exp\left\{ (r - \frac{1}{2}\sigma^2) dt + \sigma \sqrt{dt} Z \right\}$
    \State neg[$i$] $\gets$ neg[$i-1$] $\cdot \exp\left\{ (r - \frac{1}{2}\sigma^2) dt - \sigma \sqrt{dt} Z \right\}$
  \EndFor
  \State \Return (pos, neg)
  \EndFunction
 \end{algorithmic}
 \caption{Generating antithetic trajectories.}
 \label{alg:single-tr}
\end{algorithm}
 
\begin{algorithm}[!ht]
 \begin{algorithmic}[1]
  \Function{PriceAsianCallAV}{$n$, $S_0$, $\sigma$, $r$, $T$, $E$, $K$}
    \State  $sum \gets sum\_sq \gets 0$
    \For{$i=1$ {\bf to} $n$}
      \State (pos,neg) $\gets$ \Call{Trajectory}{$S_0$, $\sigma$, $r$, $T$, $K$}
      \State $H_{pos} \gets \max($mean(pos)$- E, 0) \cdot \exp\{-rT\}$
      \State $H_{neg} \gets \max($mean(neg)$- E, 0) \cdot \exp\{-rT\}$
      \State $H \gets \frac{1}{2} \cdot (H_{pos} + H_{neg})$
      \State $sum \gets sum + H$
      \State $ sum\_sq \gets sum\_sq + H^2$
    \EndFor
    \State $var \gets (sum\_sq - sum \cdot sum/n) / (n-1)$
    \State $se \gets \sqrt{var / n}$
    \State $price \gets sum / n$
    \State \Return $(price, var, se)$
  \EndFunction
 \end{algorithmic}
 \caption{Pricing Asian call options}
 \label{alg:priceAsianAV}
\end{algorithm}

In a similar manner we can value any path-dependent options. For instance we present results of pricing some barrier options. In this section we still set market parameters as in (\ref{eq:marketParams}), moreover we take
\[ K = 50, \]
it may be regarded as checking once a week if the barrier was hit. As a control variate we used the payoff from a vanilla put with strike equal to the spot price (i.e. 100).

\FloatBarrier
\begin{example}
Suppose that $S$ is an exchange rate between some currencies. Consider an exporter whose production becomes unprofitable when exchange rate becomes too low. He would like to be hedged for pessimistic scenarios, thus he is interested in purchasing put options with strike 100, whose price from the Black-Scholes formula is 5.57.
In order to save some capital he prefers cheaper barrier options rather than vanilla options. He may think the following: ``If the exchange rate at some point will be very high, then the profit made at that time will cover any eventual losses when the rate sinks down''. Thus he may decide to buy put options with strike 100 and with up-and-out barrier 115.

Monte Carlo methods allow us to price such an instrument, and the results are gathered in Table \ref{tab:barrier1} and Figure \ref{fig:barrier1}. 
\end{example}

\begin{table}[!ht]
\centering
 \caption{Results of pricing put@100 with an up-and-out barrier 115. Black-Scholes price of the option without barrier equals 5.57.}
 \label{tab:barrier1}
\begin{tabular} {||c | c | c | c | c |c | c ||}  
 \hline 
  & \multicolumn{2}{|c|}{ CMC } & \multicolumn{2}{|c|}{ AV } & \multicolumn{2}{|c|}{ CV } \\
  $\log_{10}(n)$ & \multicolumn{1}{c}{ $\CMCa[H, 2n]$ } & \multicolumn{1}{c|}{ s.e. } & \multicolumn{1}{c}{ $\AVa[H, n]$ } & \multicolumn{1}{c|}{ s.e. } & \multicolumn{1}{c}{ $\CVa[H, n]$ } & \multicolumn{1}{c|}{ s.e. } \\ \hline \hline 
 3  & 5.10 & 0.189 & 5.13 & 0.150 & 5.20 & 0.072 \\ \hline 
 4  & 5.14 & 0.060 & 5.15 & 0.049 & 5.15 & 0.023 \\ \hline 
 5  & 5.20 & 0.019 & 5.17 & 0.015 & 5.18 & 0.007 \\ \hline 
 6  & 5.18 & 0.006 & 5.17 & 0.005 & 5.18 & 0.002 \\ \hline 
\end{tabular}  
\end{table}
\begin{figure}[!ht]
\centering
 \includegraphics[scale=0.5]{images/PricingEuropean/boxPut100UaO115.pdf}
 \includegraphics[scale=0.5]{images/PricingEuropean/convergencePut100UaO115.pdf}
\caption{The accuracy of pricing put@100 with an up-and-in barrier 115. Plots were created in the similar manner as in Figure \ref{fig:vanilla1}.}
\label{fig:barrier1}
\end{figure}

\FloatBarrier
\begin{example}
 On the other hand, the exporter from the previous example may think: ``My losses are not too severe, until the exchange rate sinks down extremely low. In this case I would like to retrieve my losses at the option's expiration date.''. Hence, he may be interested in put options with strike 100 and with down-and-in barrier 70.
Table \ref{tab:barrier2} and Figure \ref{fig:barrier2} present results of pricing such instrument with Monte Carlo methods.
\end{example}


\begin{table}[!ht]
\centering
 \caption{Results of pricing put@100 with down-and-in barrier 70. Black-Scholes price of the option without barrier equals 5.57.}
 \label{tab:barrier2}
\begin{tabular} {|c |c |c |c |c |c |c |}  
 \hline 
  & \multicolumn{2}{|c|}{ CMC } & \multicolumn{2}{|c|}{ AV } & \multicolumn{2}{|c|}{ CV } \\
  $\log_{10}(n)$ & \multicolumn{1}{c}{ $\CMCa[H, 2n]$ } & \multicolumn{1}{c|}{ s.e. } & \multicolumn{1}{c}{ $\AVa[H, n]$ } & \multicolumn{1}{c|}{ s.e. } & \multicolumn{1}{c}{ $\CVa[H, n]$ } & \multicolumn{1}{c|}{ s.e. } \\ \hline \hline 
 3 & 1.33 & 0.138 & 1.38 & 0.135 & 1.37 & 0.152 \\ \hline 
 4 & 1.29 & 0.043 & 1.31 & 0.043 & 1.31 & 0.049 \\ \hline 
 5 & 1.38 & 0.014 & 1.36 & 0.014 & 1.36 & 0.016 \\ \hline 
 6 & 1.35 & 0.004 & 1.35 & 0.004 & 1.36 & 0.004 \\ \hline 
\end{tabular}
\end{table}  
\begin{figure}[!ht]
\centering
 \includegraphics[scale=0.5]{images/PricingEuropean/boxPut100DaI70.pdf}
 \includegraphics[scale=0.5]{images/PricingEuropean/convergencePut100DaI70.pdf}
\caption{The accuracy of pricing put@100 with a down-and-in barrier 70. Plots were created in the similar manner as in Figure \ref{fig:vanilla1}.}
\label{fig:barrier2}
\end{figure}

As previously, CV seems to be the best method when payoffs from vanilla option and barrier options are highly correlated. It is the case during the valuation of put@100 with up-and-out barrier 115. The option expires in the money when asset prices are low, while reaching the barrier happens when prices are high.
Hence, in most cases if the option ended in the money, then the barrier was not hit, and if the barrier was hit, then the option would expire worthless anyway. Thus, the payoff of barrier and vanilla options are highly correlated.

In the case of put@100 with down-and-in barrier 70, the barrier is set very low; hitting it happens seldom, thus often the option expires worthless even if an analogous vanilla option ends in the money. In consequence the correlation between payoffs of the barrier and vanilla options becomes low, and in this case AV method gives slightly better accuracy. 

At the end of this section, it is worth to mention, that there exist analytical formulas for prices of European barrier options, see \cite{wilmott}, chapter 23.

\section{Multi-asset instruments}
\label{sec:multi-asset}
The valuation of derivatives whose payoff depends on many assets does not differ much from one dimensional case. We only have to remember to include correlations between the assets. Let
\[ \Sigma = \left( \varrho_{ij} \right)_{i,j=1}^d \]
be the matrix describing correlation between risky assets (recall that by that we understand the correlation between Wiener processes appearing in assets dynamics). Equation (\ref{eq:priceChange}) holds for every asset, i.e.
\begin{equation}
 \label{eq:priceChangeMulti}
  S^{(i)}_{t + {\Delta} t} = S^{(i)}_t \exp\left\{ (r - \frac{1}{2}\sigma_i^2)\Delta t + \sigma_i \sqrt{\Delta t} Z_i \right\}\ \ (i = 1,2,\ldots,d),
\end{equation}
where each $Z_i$ is standard normal. However, we have to take into account correlation between variables $Z_i$. To be precise $Z \sim \mathcal{N}(0, \Sigma)$.

Equation (\ref{eq:priceChangeMulti}) is the key to generating multi-asset scenarios. Its usage is shown in Algorithm \ref{alg:multi-tr}, which creates two antithetic scenarios of market evolution. Figure \ref{fig:corrPaths} illustrates trajectories of prices of correlated assets.

\begin{algorithm}
 \begin{algorithmic}[1]
  \Function{MultiTrajectory}{$S_0$, $\sigma$, $r$, $\Sigma$, $T$, $K$}
  
  \Comment{{\color{comment} $S_0$ and $\sigma$ are now arrays, for example $\sigma[3]$ is the volatility of the third asset}}
  \State $S \gets$ two dimensional array \Comment{{\color{comment} S[i,k] is the price of i-th asset at k-th time point}}
  \State $S^\star \gets$ two dimensional array \Comment{{\color{comment} antithetic scenario}}
  \State $L \gets$ Cholesky decomposition of $\Sigma$ \Comment{{\color{comment} $\Sigma = LL'$}}
  \For{$i=1$ {\bf to} $d$}
    \State $S[i,0] \gets S_0[i]$
    \State $S^\star[i,0] \gets S_0[i]$
  \EndFor
  \State $dt \gets T/K$
  \For{$k=1$ {\bf to} $K$}
    \State $Z \gets$ array of $d$ independent standard normal variates
    \State $Z \gets LZ$ \Comment{{\color{comment} now $Z$ is a sample from $\mathcal{N}(0, \Sigma)$ distribution}}
    \For{$i=1$ {\bf to} $d$}
      \State $S[i, k] \gets S[i, k-1] \cdot \exp\left\{ (r - \frac{1}{2}\sigma[i]^2) dt + \sigma[i] \sqrt{dt} Z[i] \right\}$
      \State $S^\star[i, k] \gets S^\star[i, k-1] \cdot \exp\left\{ (r - \frac{1}{2}\sigma[i]^2) dt - \sigma[i] \sqrt{dt} Z[i] \right\}$
    \EndFor
  \EndFor
  \State \Return $(S, S^\star)$
  \EndFunction
 \end{algorithmic}
 \caption{Generating multi-asset trajectories.}
 \label{alg:multi-tr}
\end{algorithm}

\begin{figure}
\centering
 \includegraphics[scale=0.5]{images/PricingEuropean/correlatedPaths.pdf}
\caption{Price trajectories of correlated assets. The correlation between red and green equals 0.8, between red and blue -0.8, and between green and blue also -0.8.}
\label{fig:corrPaths}
\end{figure}

\begin{remark}
 If the payoff depends only on the assets values at the end of the path, then there is no need to generate whole trajectories. It is sufficient to generate asset prices only at the expiry, using equations
 \[ S^{(i)}_T = S^{(i)}_0 \exp\left\{ (r - \frac{1}{2}\sigma_i^2)T + \sigma_i \sqrt{T} Z_i \right\} \ \ (i = 1,2,\ldots,d),\]
  where  $Z \sim \mathcal{N}(0, \Sigma)$. We do not have to write a new algorithm to do that, we can use function 
   \Call{MultiTrajectory}{}
 with $K=1$.

\end{remark}

Since we have a method for scenario generation, valuation of the multi-asset options is pretty straightforward. For example we write a procedure for pricing basket put options using AV method. The basket is described by an array $\eta$, where $\eta[i]$ means how many units of $i$-th asset are contained in a basket. In Algorithm \ref{alg:priceBasketPutAV} we called function \Call{MultiTrajectory}{} with $K=1$, because payoff of the basket option does not depend on the history. 

\begin{algorithm}[!ht]
 \begin{algorithmic}[1]
  \Function{PriceBasketPutAV}{$n$, $S_0$, $\sigma$, $r$, $\Sigma$, $T$, $\eta$, $E$}
  \State  $sum \gets sum\_sq \gets 0$
  \For{$j=1$ {\bf to} $n$}
    \State $(S, S^\star) \gets$ \Call{MultiTrajectory}{$S_0$, $\sigma$, $r$, $\Sigma$, $T$, $1$}
    \State $H \gets \frac{1}{2}\exp\{-rT\} ($ \Call{BasketPutPayoff}{$S$, $\eta$, $E$, $1$} +\\ 
    \hspace{132pt} \Call{BasketPutPayoff}{$S^\star$, $\eta$, $E$, $1$} $)$
  \EndFor
  \State $var \gets (sum\_sq - sum \cdot sum/n) / (n-1)$
  \State $se \gets \sqrt{var / n}$
  \State $price \gets sum / n$
  \State \Return $(price, var, se)$
  \EndFunction
  \Function{BasketPutPayoff}{$S$, $\eta$, $E$, $K$}
    \State sum $\gets 0$
    \For{$i=1$ {\bf to} $d$}
      \State sum $\gets$ sum + $\eta[i]\cdot S[i, K]$
    \EndFor
    \State \Return $\max(E - sum, 0)$
  \EndFunction
 \end{algorithmic}
 \caption{Pricing basket put option.}
 \label{alg:priceBasketPutAV}
\end{algorithm}


Modifications of options and their payoffs are only bounded by investors imagination, however, it should be clear how to modify Algorithm \ref{alg:priceBasketPutAV} to price options with any arbitrary payoff, even path-dependent. This is the main advantage of Monte Carlo methods in option pricing -- flexibility, which cannot be provided by the finite difference or binomial trees.

\begin{example}
We used the described technique to price a basket vanilla call option. We took following parameters:
\begin{equation}
\label{eq:EuBasketParams}
 \begin{split}
  d &= 3,\ S_0 = (10, 50, 100)\\
  r &= 0.05,\ \sigma = (0.4, 0.2, 0.3),\\ 
  \Sigma &= \left( \begin{array}{rrr}
            1 & 0.8 & -0.8\\
            0.8 & 1 & -0.8\\
            -0.8 & -0.8 & 1
           \end{array} \right)\\
  E &= 90,\ \eta = (10, -2, 1),\ T=1
 \end{split}
\end{equation}
Such contract at the expiration date gives its owner right to change 2 shares of the second asset and amount of money $E$ for 10 shares of the first asset and one of the third. The results are gathered in Table \ref{tab:EuBasket} and Figure \ref{fig:EuBasket}. In CV method value of the basket at the expiry was used as the control variate. From section \ref{sec:risk-neutral} we know its expectation: $\Em[\eta \cdot S_T] = \eta \cdot S_0$. 
CV method gave the most accurate prices.
\end{example}

\begin{table}[!ht]
\centering
 \caption{Results of pricing the basket option described in equations (\ref{eq:EuBasketParams}). }
 \label{tab:EuBasket}
\begin{tabular} {|c |c |c |c |c |c |c |}  
 \hline 
  & \multicolumn{2}{|c|}{ CMC } & \multicolumn{2}{|c|}{ AV } & \multicolumn{2}{|c|}{ CV } \\
  $\log_{10}(n)$ & \multicolumn{1}{c}{ $\CMCa[H, 2n]$ } & \multicolumn{1}{c|}{ s.e. } & \multicolumn{1}{c}{ $\AVa[H, n]$ } & \multicolumn{1}{c|}{ s.e. } & \multicolumn{1}{c}{ $\CVa[H, n]$ } & \multicolumn{1}{c|}{ s.e. } \\ \hline \hline 
 3  & 18.91 & 0.507 & 19.90 & 0.444 & 19.85 & 0.241 \\ \hline 
 4  & 19.66 & 0.165 & 19.84 & 0.143 & 19.73 & 0.077 \\ \hline 
 5  & 19.65 & 0.052 & 19.68 & 0.045 & 19.70 & 0.024 \\ \hline 
 6  & 19.72 & 0.017 & 19.70 & 0.014 & 19.70 & 0.008 \\ \hline 
\end{tabular}
\end{table}  
\begin{figure}[!ht]
\centering
 \includegraphics[scale=0.5]{images/PricingEuropean/boxBasket.pdf}
 \includegraphics[scale=0.5]{images/PricingEuropean/convergenceBasket.pdf}
\caption{The accuracy of pricing basket option described in equations (\ref{eq:EuBasketParams}). Prices calculated using AV and CMC methods look like they were converging to different values. It is caused by the fact, that one price is underestimated and the second is overestimated. However, in both cases the distance to the correct value 19.70 is not greater than two standard errors. It means that the above plot gives us no reason to worry -- for greater numbers of simulations green and red lines would ``stick'' to each other almost surely. }
\label{fig:EuBasket}
\end{figure}

\chapter[{Pricing American options using Least Squares Monte Carlo}]{Pricing American options using \\Least Squares Monte Carlo}
\label{chapter:pricingAmerican}
In chapter \Roman{chapter} we describe American-style derivatives. The difference between them and previously discussed European-style contracts is that the American feature allows the owner of the derivative to exercise it in \emph{any time} up to the expiration date. This additional attribute makes the instrument much harder to analyse and more advanced theory is necessary to value and hedge it.
Due to the practical nature of this thesis we do not get deep into details. Nevertheless, we provide mathematical tools necessary for pricing American contingent claims, advising the reader to find proofs in the more specialized literature. 

It turns out that the Monte Carlo method from previous chapter cannot be carried over directly to price American contracts. However, we describe a method invented by Francis Longstaff and Eduardo Schwartz, called the Least Squares Monte Carlo. It introduces a clever trick, so the simulations can still be involved. 

\section{American contingent claims}
In opposite to European contingent claims, American contingent claims are not random variables, but \emph{processes}.
\begin{mydef}
 \label{def:cc_am}
 An \textbf{American contingent claim} is a non-negative adapted process $C = (C_t)_{t=0}^T$ on the filtered probability space $(\Omega, \mathcal{F}, (\mathcal{F}_t)_{t=0}^T, \P)$.
From now on by a \textbf{derivative} of the underlying assets $\Sa$ we understand such American contingent claim $C$ which is adapted also to the filtration generated by $\Sa$ (which in general may be smaller than $\mathcal{F}$).
\end{mydef}
Value $C_t$ may be interpreted as the payoff obtained from the claim if it is exercised at time $t$. The definitions from section \ref{sec:ECC} carry over directly to their American counterparts, for example American call and put options on the $i$\textsuperscript{th} asset are defined as derivatives with the payoffs
\begin{equation*}
 \begin{split}
  C^{\text{call}}_t &= (S^{(i)}_t - E)_+, \\
  C^{\text{put}}_t &= (E - S^{(i)}_t)_+,
 \end{split}
\end{equation*}
where $E$ is the strike price.
\begin{remark}
 European contracts are in fact  a particular case of American contracts. An European claim $\tilde{C}$ may be seen as an American claim $C = (C_t)_{t=0}^T$ with payoff
 \[ C_t = \begin{cases}
         \tilde{C},\ \ \ \ t = T \\
         0,\ \ \ \ \ t < T.
        \end{cases}
        \]
\end{remark}

As usually, it is convenient to quote payoff values in terms of time 0.
\begin{mydef}
 The discounted value of the American contingent claim $C$ is a process $H = (H_t)_{t=0}^T$ given by
 \begin{equation*}
  H_t = \frac{C_t}{S^{(0)}_t}.
 \end{equation*}
 The process $H$ is called an \textbf{American discounted claim}.
\end{mydef}

\subsection{Exercise strategies}
The exercise time is entirely up to the buyer. He or she dynamically decides when to claim the payoff, watching the market evolution. However, at the very beginning he or she may plan under which conditions the option should be exercised.
\begin{mydef}
 An \textbf{exercise strategy} for an American contingent claim $C$ is a stopping time $\tau$ taking values in $[0,T]$. The payoff resulting from following the strategy $\tau$ is defined for any $\omega \in \Omega$ as
 \[ C_{\tau}(\omega) = C_{\tau(\omega)}(\omega).\]
 The set of exercise strategies is denoted by $\mathcal{T}$. 
\end{mydef}
An exercise strategy may be seen as an oracle telling at any time $t$ whether or not the option should be exercised now, basing only on the informations available up to time $t$. Note that the definition of an exercise strategy is limited to these stopping times which do not take value $\infty$. That is so, because the option is exercised always, however, if the owner postpones the exercise to the expiration date, then the payoff may equal 0.

Of course the option buyer looks for the best exercise strategy. Thus, we need to write precisely what we mean by that.
\begin{mydef}
 An exercise strategy $\hat{\tau}$ is called optimal (with respect to $\Pm$) if and only if
\begin{equation}
\label{eq:AM_optStrategy}
\Em[H_{\hat{\tau}}] = \sup\limits_{\tau \in T} \Em[H_{\tau}]. 
\end{equation}
\end{mydef}

\begin{remark}
 An optimal exercise strategy does not always exist. However, for each $\epsilon > 0$ there exist such a stopping time $\tau_\epsilon$ that
 \[ \Em[H_{\tau_\epsilon}] \geq \sup\limits_{\tau \in T} \Em[H_{\tau}] - \epsilon.\]
 Moreover, if the optimal exercise strategy exists it is not necessarily unique.
\end{remark}
As we can see the owner of the claim may choose such a strategy that its expected value is arbitrarily close to the $\sup\limits_{\tau \in T} \Em[H_{\tau}]$. It suggests how the option price should be defined.
\begin{mydef}
 Price of the discounted American contingent claim $H$ is given by
\begin{equation}
\label{eq:AM_optPrice}
V_0 = \sup\limits_{\tau \in T} \Em[H_{\tau}]. 
\end{equation}
\end{mydef}
We also need to analyse how the option price changes over time.
\begin{mydef}
 \label{def:valueProcess}
 Let  $\mathcal{T}_t = \{ \tau \in \mathcal{T}:\ \tau \geq t \}$. The discounted \textbf{price process} (or the \textbf{value process}) of the  discounted American contingent claim $H$ is given by
\begin{equation}
\label{eq:AM_valueProcess}
V_t = \esssup\limits_{\tau \in  \mathcal{T}_t} \Em[H_{\tau} | \mathcal{F}_t]. 
\end{equation}
 If there exists the stopping time realizing this essential supremum, then we call it optimal in $\mathcal{T}_t$ and denote it by $\hat{\tau}_t$.
\end{mydef}

\begin{remark}
 The Reader may wonder why in the definitions above the martingale measure is used. It may be even confusing that we \emph{define} what is the value of the option instead of \emph{proving} that some formula gives the value. Remember that the goal of the theory is to answer what should be the price of the option. The proposed definition is good in the sense that the extended market model $(S^{(0)}, S^{(1)}, \ldots, S^{(d)}, V)$ is still arbitrage free -- proof for discrete time is in \cite{follmer} (Theorem 6.33).
\end{remark}


\begin{prop}
 At any time $t$ value of the American contingent claim is not less than the value of its European counterpart.
\end{prop}
\begin{proof}
 Following the exercise strategy $\tau_t \equiv T$ always results in the same payoff from the American contract as from the European. Therefore, the price of the American claim must be greater or equal, because the supremum is taken over all stopping times in $\mathcal{T}_t$.
\end{proof}

\begin{prop}
\label{prop:amCall}
 Assume that the interest rate $r$ is non-negative. The exercise strategy $\tau \equiv T$ is optimal for an American call option on a non-dividend-paying stock.
\end{prop}
\begin{proof}
 For brevity let $S$ be the price process of the underlying asset (not the vector process of all the risky assets). From the previous Proposition and Theorem \ref{thm:pricesProcess_mtg} we have
 \[V_t \geq \Em[e^{-rT}(S_T - E)_+ | \mathcal{F}_t].\] Moreover,
 \[\Em[(S_T - E)_+ | \mathcal{F}_t] \geq \Em[S_T - E | \mathcal{F}_t],\]
 because negative values of $S_T - E$ are zeroed in the left hand side expectation. Hence,
 \[ e^{rt} V_t \geq e^{-r(T-t)}\Em[S_T - E | \mathcal{F}_t] = S_t - e^{-r(T-t)}E > S_t - E.\]
 The obtained inequality tells us that value of the option is always higher than the immediate exercise, thus it is more profitable to sell the option rather than exercise it.
\end{proof}
The above proposition simply says that \textbf{it is never worth to exercise the American call option before the expiration date} (if the stock does not pay a dividend and the interest rate is non-negative). American puts do not have that property -- for low prices of the underlying it is optimal to exercise the option immediately. These facts are illustrated in Figure \ref{fig:lsm:amPrice}.

\begin{figure}
\centering
 \includegraphics[scale=0.5]{images/LSM/put.pdf}
 \includegraphics[scale=0.5]{images/LSM/call.pdf}
\caption{Prices of American and European vanilla options. In case of a put (left picture), we see that for low stock ratings the European option's price is lower than its intrinsic value. On the other hand, the price of the put with an American exercise feature coincides with the immediate payoff for low asset prices -- this is the situation when the option should be exercised immediately. In case of a call (right picture), the option's price is always greater than its intrinsic value. We cannot see the curve of the European option's price, because it is entirely covered by the line of the American option's price -- their values are the same for every spot price. Calculations were performed under parameters: strike~$=100,\ T=1,\ \sigma=0.3,\ r=0.08$.}
\label{fig:lsm:amPrice}
\end{figure}

For negative interest rates the situation is inverted. A similar proof shows that in this case the exercise of an American put should be postponed to the expiry and American calls should be exercised for high stock prices. However, a negative interest rate is a very rare situation and somewhat pathological.

Unfortunately the above definition of the value process is not constructive. It does not tell us how to find the optimal exercise strategy nor how to compute the option's value. We need to introduce a little more theory. The following two definitions are formulated for a general measure $\Q$. 

\begin{mydef}
 Let $X$ and $Y$ be two processes on the same probability space $(\Omega, \mathcal{F}, (\mathcal{F}_t)_{t=0}^T, \Q)$. We say that $X$ \textbf{dominates} $Y$ if for all $t \geq 0$, $X_t \geq Y_t$ $\Q$-a.s.
\end{mydef}
 
\begin{mydef}
 Let $Y$ be the process such that for all $0 \leq t \leq T$, $\E^{\Q}[Y_t] < \infty$. Its \textbf{Snell envelope} $U^{\Q}$ is defined as the smallest supermartingale dominating $Y$. In the other words $U^{\Q}$ is a supermartingale dominating $Y$ and if $\tilde{U}$ is another supermartingale dominating $Y$, then for all $t \geq 0$, $U_t \leq \tilde{U}_t$, $\Q$-a.s.
\end{mydef}
It is proven that Snell envelope exists for a vast class of processes. Following statement shows its importance (it is quoted after \cite{bjork}, Theorem 21.23):
\begin{thm}
 \label{thm:snell}
 If $\sup\limits_{\tau \in T} \Em|H_{\tau}| < \infty$, then the value process $V$ is the Snell envelope of the claim $H$ with respect to the measure $\Pm$. Moreover, if there exists an optimal stopping time (not necessarily unique), then the smallest one is given by
 \begin{equation}
  \label{eq:optStop}
  \hat{\tau}_t = \inf\{ s \geq t:\ V_s = H_s \}.
 \end{equation}
\end{thm}
Theorem \ref{thm:snell} not only states that the Snell envelope $U^{\Pm}$ of the claim $H$ coincides with its value $V$. It also gives us a recipe for the optimal stopping: \textbf{exercise the option when its value and immediate payoff are equal}.

\subsection{Explanations in discrete time}
The presented theory may look incomprehensible at a first glance. In order to get some intuition let us explain the above concepts in a discrete time. Let us divide the time left to the expiration to $K$ intervals separated by points $0 = t_0 < t_1 < \ldots < t_K = T$. In order to reduce the number of indices we use the following notation in the discrete time:
\begin{align*}
 V_k &:= V_{t_k} \\
 H_k &:= H_{t_k} \\
 \mathcal{F}_k &:= \mathcal{F}_{t_k}
\end{align*}
Also the stopping times $\tau$ now takes values in $\{0, 1, \ldots, K\}$ (and $\tau_k$ denotes such stopping time, that $\tau_k \geq k$ a.s.).
Since $k$ is usually associated with natural numbers and $t$ with real numbers, we hope that whenever Reader encounters $V_k$ he or she bears in mind that it is the value at the $k$\textsuperscript{th} time point, i.e. at time $t_k$, not at time $k$.

It turns out that in discrete time equation (\ref{eq:AM_valueProcess}) can be written in more friendly form.
\begin{prop}
\label{prop:AM_valueProcessDisc}
 In the discrete time the value process of the American contingent claim is given by the recursion
 \begin{equation}
  \label{eq:AM_valueProcessDisc}
  V_{k} = \begin{cases}
             H_T &\text{if } k = K,\\
             \max\Bigl(H_{k}, \Em\bigl[ V_{k+1} | \mathcal{F}_{k} \bigr] \Bigl)\ \ \ \ &\text{otherwise.}
            \end{cases}
 \end{equation}
 Moreover, it is optimal to stop at $k$\textsuperscript{th} time point if and only if $V_{k} = H_{k}$. 
\end{prop}
\begin{proof}
Let us consider the value of the claim at the $k$\textsuperscript{th} time point.  We have to decide if it is optimal to stop at this point. If it is not optimal, then $\hat{\tau}_{k} = \hat{\tau}_{k+1}$ must hold. By the definition, $V_{k}$ is the expected value of the payoff resulting from following strategy $\hat{\tau}_{k}$. Similarly, at the $(k+1)$\textsuperscript{th} time point the claim's value resulting from following $\hat{\tau}_{k+1}$ equals $V_{k+1}$, hence
\[ V_{k} = \Em[H_{\hat{\tau}_{k}} | \mathcal{F}_{k}] = \Em[H_{\hat{\tau}_{k+1}} | \mathcal{F}_{k}] = \Em\bigl[\Em[H_{\hat{\tau}_{k+1}} | \mathcal{F}_{k+1}] \bigl|\bigr. \mathcal{F}_{k} \bigr] = \Em[V_{k+1} | \mathcal{F}_{k}].\]
On the other hand, if it is optimal to stop at time $t_k$, then the price equals the immediate payoff, that is $V_{k} = H_{k}$. Hence, at the $k$\textsuperscript{th} time point we have to compare $H_{k}$ and $\Em[V_{k+1} | \mathcal{F}_{k}]$, and we stop if and only if the former is greater.
\end{proof}
Proposition 6.11 in \cite{follmer} states that process defined via (\ref{eq:AM_valueProcessDisc}) is the Snell envelope of $H$ (in the discrete sense).

An obvious conclusion from Proposition \ref{prop:AM_valueProcessDisc} gives us a useful formula for the optimal stopping.
\begin{coro}
 \label{coro:AM_optStopDisc}
 The optimal exercise strategies at time 0, and at $k$\textsuperscript{th} time point are given by:
 \begin{equation}
  \label{eq:AM_optStopDisc}
  \begin{split}
   \hat{\tau} &= t_n,\ \ \ \text{where } n = \min\{m \geq 0:\ V_m = H_m \}.\\
   \hat{\tau}_{k} &= t_n,\ \ \ \text{where } n = \min\{m \geq k:\ V_m = H_m \}.\\
  \end{split}
 \end{equation}
\end{coro}
\begin{remark}
 Since finding the optimal exercise strategy in discrete time involves taking minimum, not infimum, hence it always exist. However, it is not necessarily unique.
\end{remark}

\begin{example}
 Figure \ref{fig:binTree} illustrates terms discussed above. It contains a binomial tree (Reader not familiar with binomial trees is advised to read about the CRR model) used to price an American put option. The red colour indicates nodes where $V_{k} = H_{k}$, i.e. the immediate payoff exceeds the expected payoffs in the future. Hence, Corollary \ref{coro:AM_optStopDisc} states that following procedure is an optimal exercise strategy: exercise the option when you step in the red node.
 
 It is also remarkable that decision whether the option should be exercised depends not only on the price of the underlying, but also on time. Note that we should exercise the option for asset price 91 in the time step one before last, but not in the previous time steps.
\end{example}

\begin{figure}
\centering
 \includegraphics[scale=0.4]{images/LSM/binomialTree.png}
\caption{The binomial tree used to price American put@100. In nodes marked red it is optimal to exercise the option. Blue indicates those nodes where the exercise should be postponed. }
\label{fig:binTree}
\end{figure}

\section{Least Squares Monte Carlo}
After wading through the theory we want to use it in practice. We showed in the previous chapter by examples of European options that Monte Carlo methods are very flexible and can be used to price exotic options with very sophisticated payoffs (at least when they are European-styled). Therefore, it is natural to try to adjust them for pricing American claims. 

\subsection{Difficulties with American options}

Let us think how pricing American options using Monte Carlo method may look like. Of course, as for European options we need to simulate some paths and for each such a path we have to determine what payoff we would obtain. Since the option may be exercised at any time, it is not sufficient to check the payoff at the expiration date. We have, therefore, to follow the path and at each time step compare the immediate payoff with the expectation of the future payoff.

Consider an American put option. Suppose that we are already at time $k$, and we need to compute $V_{k}$. From (\ref{eq:AM_valueProcessDisc}) we know that it is the greater of values: $H_{k}$ or $\Em\bigl[ V_{k+1} | \mathcal{F}_{k} \bigr]$. Value $H_{k}$ is obtained trivially, but how can we get $\Em\bigl[ V_{k+1} | \mathcal{F}_{k} \bigr]$? Note that it is the price of a similar option, with a little shorter time to the expiration. In order to calculate it, we simulate the value $S_{k+1}$ and we call the procedure recursively, with the initial asset price $S_{k+1}$. Algorithm \ref{alg:naiveProcedure} is based on this approach.

\begin{algorithm}
 \begin{algorithmic}[1]
  \Function{PriceAmericanPut}{$n$, $k$, $K$, $S_0$, $\sigma$, $r$, $T$, $E$ }
    \If{$k=K$}
      \State \Return $e^{-rT} \max(0, E - S_0)$
    \EndIf
    \State $dt \gets T/K$
    \State $t \gets k\cdot dt$
    \State $S \gets$ simulate $n$ asset prices in the next step 
    \State $H \gets$ $n$-element array
    \For{$i = 1$ {\bf to} $n$} 
       \State $H[i] \gets$ \Call{PriceAmericanPut}{$n$, $k+1$, $K$, $S[i]$, $\sigma$, $r$, $T$, $E$ }
    \EndFor 
    \State $av \gets$ average$(H[i])$
    \State \Return max($e^{-rt} \max(0, E - S_0),\ av$)
  \EndFunction 
 \end{algorithmic}
 \caption{Pricing American options by ``Monte Carlo on Monte Carlo''. This is how pricing \emph{cannot} be done.}
 \label{alg:naiveProcedure}
\end{algorithm}

Unfortunately, this is a very ineffective procedure. We have trivial
\begin{prop}
 The computational complexity of Algorithm \ref{alg:naiveProcedure} equals $\mathcal{O}(N^K)$, where $N$ is the number of simulated paths and $K$ is the number of time steps.
\end{prop}

Computing $\Em\bigl[ V_{k+1} | \mathcal{F}_{k} \bigr]$ involves generating new paths starting at time point $k$. We perform so-called ``Monte Carlo on Monte Carlo''. Since we have to do it at every time step, complexity of such algorithm becomes exponential. These difficulties are illustrated on Figure \ref{fig:MC_difficulties}.
\begin{figure}
\centering
 \includegraphics[scale=0.6]{images/LSM/LSMidea1.pdf}
\caption{The illustration of difficulties occurring with attempts to involve Monte Carlo method in pricing American claims. At each time point we have to generate subtrajectories in order to calculate expected value of continuation.}
\label{fig:MC_difficulties}
\end{figure}

As we can see Monte Carlo methods used to price European options cannot directly translated to price American options.
\newpage
\subsection{LSM idea}
\label{subsec:LSMidea}

We present one of methods allowing to avoid the difficulties described above. It is called \textbf{Least Squares Monte Carlo}, or \textbf{Longstaff-Schwartz method}, after its discoverers Francis Longstaff and Eduardo Schwartz, abbreviated \textbf{LSM}.

Consider again the problem of pricing American options with the Monte Carlo method. At the beginning we simulate thousands of trajectories. At every time point on every path we need to obtain an expectation of the future payoff. Assume that we follow one path and we are already at time $t_k$. In order to obtain the option's value at this point, we have to compute $\Em\bigl[ V_{k+1} | \mathcal{F}_{k} \bigr]$. Previously we came up with an idea to generate new trajectories starting from that point.
Note that we have exactly the same problem on every path. Look at Figure \ref{fig:LSMidea}. There are several points marked at line $x = t_k (= 0.6)$, they are asset prices on simulated trajectories, at time $t_k$. Using previous method we had to release thousands of ``subpaths'' from each such a point. Clearly, we cannot afford this. However, these points are not lying far from each other. The main idea of the LSM is: instead of generating new ``subpaths'' starting at time $t_k$, we can use the remainders of the ``original'' paths.
\begin{figure}
\centering
 \includegraphics[scale=0.6]{images/LSM/LSMidea2.pdf}
\caption{The illustration of the idea allowing to overcome the difficulties with Monte Carlo described in the previous section. Suppose that we are following the purple path and we decide whether the option should be exercised at time $0.6$. We cannot release new subpaths exactly from the purple point in order to get the estimated value of continuation, but we can use remainders of the rest of trajectories. They start almost in the purple point and the differences will be compensated by regression.}
\label{fig:LSMidea}
\end{figure}

In LSM at each path we move backwards. At every time point $t_k$ we estimate the value of the option conditional on not exercising the option before that time. Having the estimated value of continuation, we easily decide whether the exercise should performed immediately or postponed. Hence, we dynamically build a stopping time approximating the optimal strategy.

As in chapter on pricing European options, $S_{i,k}$ denotes the price of the underlying at time $t_k$, on the $i$\textsuperscript{th} simulated path (similarly for $V_{i,k}$ and $H_{i,k}$). Moreover, let $D_{i,k}$ be the payoff (discounted to time 0) obtained on the $i$\textsuperscript{th} path conditional on not exercising the option up to time $t_k$ (inclusively) and following the constructed strategy after then.

\begin{remark}
  Some values appearing in the discussion have an ambiguous meaning. The reader must aware of both interpretations.
 \begin{itemize}
  \item It is clear that $S_{k}$ is a random variable, but $S_{i,k}$ may be seen in two ways. After simulations $S_{i,k}$ is a real value, a concrete realization of $S_{k}$. However, prior to the simulations $S_{i,k}$ is also a random variable with the same distribution as $S_k$. The same applies to  $V_k$, $H_k$, $D_k$, etc.
  \item Let $Y$ be any random variable measurable with respect to $\mathcal{F}_k$. From the perspective of time 0, $Y$ is unknown and depends on the market evolution up to time $t_k$. However, from the perspective of time $t_k$ we already know the history up to time $t_k$, hence $Y$ is already a concrete real value.
 \end{itemize}
\end{remark}

If we a priori knew what is the optimal exercise strategy, then $D_{k} = (H_{\hat{\tau}_{k+1}} | \mathcal{F}_{k})$ would hold. Furthermore,
\begin{equation}
 \label{eq:LSM_futurePayoff}
 \Em\bigl[ V_{k+1} | \mathcal{F}_{k} \bigr] = \Em\bigl[ \Em[H_{\hat{\tau}_{k+1}} | \mathcal{F}_{k+1}] \bpipe \mathcal{F}_{k} \bigr] = \Em\bigl[ H_{\hat{\tau}_{k+1}} | \mathcal{F}_k \bigr] = \Em[ D_{k} ]. 
\end{equation}
Of course, the optimal strategy is not known before-hand and $D_k$ is the payoff obtained from following the strategy which is built dynamically. However, we \emph{hope}\footnote{What we present here is, of course, just an intuition, not a proof.} that optimal stopping is approximated sufficiently and hence
\[ \Em[ D_{k} ] \approx  \Em\bigl[ V_{k+1} | \mathcal{F}_{k} \bigr] \]
still holds. At time point $K$ values $V_{i,K}$ and $D_{i,K-1}$ are obtained instantaneously, we have
\[ D_{i,K-1} := H_{i,K},\ \ \ \ V_{i,K} := H_{i,K}.\]
At time $t_{K-1}$ situation is not that clear. For each path we have to estimate value $\Em\bigl[ V_{i,K} | \mathcal{F}_{K-1} \bigr]$, which we believe is close to $\Em[D_{i,K-1}]$ \footnote{Few lines above we do assignment $D_{i,K-1} := H_{i,K}$, but this is done from the perspective of time $t_K$. At time $t_{K-1}$, $D_{i,K-1}$ is not known, thus we must use the expectation rather than a concrete realization.}. Calculation of $D_{i,K-1}$ is to be described below. If we knew that value, we could assign to $V_{i,K-1}$ maximum of $H_{i,K-1}$ and $D_{i,K-1}$ ($\approx \Em\bigl[ V_{i,K} | \mathcal{F}_{K-1} \bigr]$). If the former value is greater, then we decide to exercise the option at time $t_{k-1}$, and $D_{i,K-2} := H_{i,K-1}$. Otherwise we postpone the exercise, hence $D_{i,K-2} := D_{i,K-1}$. In general, for $0 < k < K$ we do
\begin{align*}
 V_{i,k} &:= \max(H_{i,k}, \Em[D_{i,k}]) \\
 D_{i,k-1} &:= \begin{cases}
                 D_{i,k},\ \ \ \ \text{if } H_{i,k} < \Em[D_{i,k}], \\
                 H_{i,k},\ \ \ \ \text{otherwise.}
               \end{cases}
\end{align*}
\begin{remark}
 Someone might argue that in above assignments instead of $\Em[D_{i,k}]$ we should take $D_{i,k}$, because after simulations this value is already known and an option should be exercised if and only if $D_{i,k} < H_{i,k}$. However, we have to take the point of view of the investor who at time $t_k$ does not know the future and must rely on the expectations. 
\end{remark}

So far we did not explain how to calculate $\Em[D_{i,k}]$. Here we take advantage of the main idea of the LSM. We cannot create thousands of ``subpaths'' starting from $S_{i,k}$ to obtain thousands of realizations of $D_{i,k}$ and assign to $\Em[D_{i,k}]$ their average, but we \emph{already have} single realizations of $D_{1,k}, D_{2,k},\ldots, D_{n,k}$. In the other words, we have mapping
\begin{equation}
 \label{eq:AM_mapping}
  S_{i,k} \mapsto D_{i,k}.
\end{equation}
Similarly as in assumption BS7. from section \ref{sec:blackScholes} we assume that
\begin{equation*}
 \Em\bigl[ V_{k+1} | \mathcal{F}_{k} \bigr] = F(S_{k}),
\end{equation*}
for some continuous function $F$. Stone-Weierstra\ss{} theorem states that $F$ can be approximated as closely as desired by a polynomial of a sufficiently large degree $m$. Hence
\begin{equation*}
 \Em[D_{k}] \approx \Em\bigl[ V_{k+1} | \mathcal{F}_{k} \bigr] \approx P(S_{k}) = a_m S_{k}^m + \ldots + a_1 S_{k} + a_0.
\end{equation*}
The method of least squares can be used to find coefficients $a_m, a_{m-1},\ldots,a_0$ fitting the mapping (\ref{eq:AM_mapping}) the best.

\subsection{Step by step example}
Description of the LSM may look a little bit overwhelming, however, the idea is very simple. We explain it once again, using a clear though quite a general example.

Suppose we are pricing an American put@100. We take parameters
\begin{equation*}
\begin{split}
 S_0 = 98,&\ r = 6.06\%,\\
 E = 100,&\ T = 1,\ K = 3. 
\end{split}
\end{equation*}
Under those parameters $t_0 = 0,\ t_1 = \frac{1}{3},\ t_2 = \frac{2}{3},\ t_3 = 1,\ e^{-\frac{1}{3}r} = 0.98$.
\bigskip

\noindent \textbf{1. The algorithm starts by simulating $n$ trajectories of the asset price.} The asset dynamics under the risk-neutral measure must be used. In our example we take $n=10$. The simulated paths are shown in Table \ref{tab:LSM_paths}. Table \ref{tab:LSM_cashflows_t3} presents what would be the payoff from the option if the owner did not exercise it before the expiration date. It corresponds to the values $D_{i,K-1} = D_{i,2}$ described previously ($D_{i,k}$ are discounted to time 0, but in tables we omit discounting for readability, instead we provide time of the payoff, hence everybody may calculate discounted value on his own).
\begin{table}[h]
 \parbox{.45\linewidth} {
   \centering
   \caption{Paths simulated under the risk-neutral measure.}
   \label{tab:LSM_paths}
   \begin{tabular} {||c |c |c |c |c ||}  
    \hline 
    i\textbackslash k & 0   &  1  &  2  &  3  \\ \hline \hline
    1 & 98 & 108 & 102 & 112 \\ \hline 
    2 & 98 & 98 & 95 & 105 \\ \hline 
    3 & 98 & 102 & 97 & 93 \\ \hline 
    4 & 98 & 105 & 98 & 106 \\ \hline 
    5 & 98 & 92 & 102 & 98 \\ \hline 
    6 & 98 & 107 & 103 & 111 \\ \hline 
    7 & 98 & 108 & 113 & 118 \\ \hline 
    8 & 98 & 96 & 99 & 109 \\ \hline 
    9 & 98 & 107 & 98 & 96 \\ \hline 
    10 & 98 & 97 & 103 & 97 \\ \hline 
   \end{tabular} 
 }
 \qquad 
 \parbox{.45\linewidth} {
  \centering
  \caption{Cash flows obtained conditional on not exercising the option before the maturity.}
  \label{tab:LSM_cashflows_t3}
  \begin{tabular}{|| c | c | c ||}
    \hline 
    i  & time  &  payoff \\ \hline \hline
    1  & 1     &      0 \\ \hline
    2  & 1     &      0 \\ \hline
    3  & 1     &      7 \\ \hline
    4  & 1     &      0 \\ \hline
    5  & 1     &      2 \\ \hline
    6  & 1     &      0 \\ \hline
    7  & 1     &      0 \\ \hline
    8  & 1     &      0 \\ \hline
    9  & 1     &      4 \\ \hline
    10  & 1     &      3 \\ \hline
  \end{tabular}
}
\end{table}

\FloatBarrier
\noindent \textbf{2. At time $t_2 = \frac{2}{3}$ the owner must decide whether or not to exercise the option.} Of course he has choice only in that paths where option is in the money. From Table \ref{tab:LSM_paths} we see that in our case at time $t_2$ option is in the money at paths number 2,3,4,8,9. To estimate future payoffs we use payoffs obtained on these paths conditional on not exercising the option up to time $t_2$ (these values are gathered in Table \ref{tab:LSM_cashflows_t3}).

\begin{table}[ht]
\parbox{.45\linewidth} {
  \centering
  \caption{Estimating future payoffs at time point 2.}
  \label{tab:LSM_regression_t2}
  \begin{tabular} {||c |c |c || c | c ||}  
  \hline 
    i & X   &  Y  & now   &  future\\ \hline \hline
    2 & 95 & $0\cdot 0.98$ & 5 & 0.33 \\ \hline
    3 & 97 & $7\cdot 0.98$ & 3 & 4.90  \\ \hline
    4 & 98 & $0\cdot 0.98$ & 2 & 3.27 \\ \hline
    8 & 99 & $0\cdot 0.98$ & 1 & -0.98 \\ \hline
    9 & 98 & $4\cdot 0.98$ & 2 & 3.27 \\ \hline
  \end{tabular}
}
\qquad
\parbox{.45\linewidth} {
  \centering
  \caption{Cash flows obtained conditional on not exercising the option before time point 2.}
  \label{tab:LSM_cashflows_t2}
  \begin{tabular}{||c|c|c||}
  \hline 
  i  & time  &  payoff \\ \hline \hline
  1  & 1     &      0 \\ \hline
  2  & 2/3   &      5 \\ \hline
  3  & 1     &      7 \\ \hline
  4  & 1     &      0 \\ \hline
  5  & 1     &      2 \\ \hline
  6  & 1     &      0 \\ \hline
  7  & 1     &      0 \\ \hline
  8  & 2/3   &      1 \\ \hline
  9  & 1     &      4 \\ \hline
  10  & 1     &      3 \\ \hline
  \end{tabular}
}
\end{table}
Look at Table \ref{tab:LSM_regression_t2}. Column X presents the asset price at time $t_2$ (only paths for which option is in the money at that time are taken into consideration). Column Y contains payoffs on that paths conditional on not exercising the option at or prior to time $t_2$, discounted to time $t_2$. The regression with a quadratic function leads to the following formula for the future estimated payoff:
\begin{equation}
 \label{eq:LSM_example1}
 Y = -1.31\cdot X^2 + 253.17\cdot X - 12257.87 
\end{equation}
Values in column ``now'' are the payoffs obtained from the immediate exercise and those in column ``future'' present the estimation of the future payoff calculated by using (\ref{eq:LSM_example1}). We decide to exercise the option on those paths where expectation of the future payoff is lower than immediate exercise. In our case these paths are 2 and 8. The update to the optimal stopping is shown on Table \ref{tab:LSM_cashflows_t2}.

\begin{remark}
 Here an inefficiency of the LSM may be seen: expected future payoff may be negative, what is of course impossible in the real life. Moreover, for asset price 95 expected future payoff is lower than for asset price 97 and 98, what also does not reflect the reality. The reason of that problem is a small number of paths for regression. In practice we use thousands of paths, then the approximating polynomial gives better estimations.
\end{remark}

\noindent \textbf{3. At time $t_1 = \frac{1}{3}$ we proceed as in time $t_2$.}
Table \ref{tab:LSM_regression_t1} shows the data for regression. It is worth to mention that at time $t_1$ we still use realized cash flows, not for example option's prices calculated at previous step -- such approach would lead to an upward bias.
\begin{table}[ht]
\parbox{.45\linewidth} {
  \centering
  \caption{Estimating future payoffs at time point 1.}
  \label{tab:LSM_regression_t1}
  \begin{tabular} {||c |c |c || c | c ||}  
  \hline 
    i & X   &  Y  & now   &  future\\ \hline \hline
    2 & 98 & $5\cdot 0.98$ & 2 & 5.0 \\ \hline
    5 & 92 & $2\cdot 0.98^2$ & 8 & 1.9  \\ \hline
    8 & 96 & $1\cdot 0.98$ & 4 & 1.1 \\ \hline
  10 & 97 & $3\cdot 0.98^2$ & 3 & 2.7 \\ \hline
  \end{tabular}
}
\qquad
\parbox{.45\linewidth} {
  \centering
  \caption{Cash flows obtained conditional on not exercising the option before time point 1.}
  \label{tab:LSM_cashflows_t1}
  \begin{tabular}{|| c | c | c ||}
  \hline 
  i  & time  &  payoff \\ \hline \hline
  1  & 1     &      0 \\ \hline
  2  & 2/3   &      5 \\ \hline
  3  & 1     &      7 \\ \hline
  4  & 1     &      0 \\ \hline
  5  & 1/3   &      8 \\ \hline
  6  & 1     &      0 \\ \hline
  7  & 1     &      0 \\ \hline
  8  & 1/3   &      4 \\ \hline
  9  & 1     &      4 \\ \hline
  10  & 1/3   &      3 \\ \hline
  \end{tabular}
}
\end{table}
This time quadratic function representing estimated future payoff has the form
\[ Y = 0.36\cdot X^2 - 67.32\cdot X + 3173.71. \]
We see that option should be exercised at paths 5,8,10, and we update properly the table of realized cash flows (Table \ref{tab:LSM_cashflows_t1}).

\noindent \textbf{4. At time 0 we have to perform final calculations to find options value.} Here we proceed different than at times $t_1$, $t_2$, because there is no data for regression. However, Table \ref{tab:LSM_cashflows_t1} already tells us what would be the future payoffs if the option was not exercised immediately. In order to obtain estimated future payoff at time 0, we take the average of the realized payoffs discounted to time 0. If the average is greater than immediate payoff, then this is the value of the option. Otherwise option should be exercised at time 0 and its value equals the immediate payoff.

In our example the average of the realized cash flows equals 2.99. It is greater than value of immediate exercise, hence \textbf{the price of the option is 2.99}.

\subsection{Mathematical background of LSM}
In the paragraph \ref{subsec:LSMidea} we described the LSM method. It gave some intuitions, however, the sign ``$\approx$`` appeared very often, thus it is certainly unclear if this method really works. Of course, we would not present it if it did not. After Longstaff and Schwartz we quote two convergence results. Reader may find the proofs in the appendix to their article (\cite{l-sch}).

\begin{mydef}
 For $\omega \in \Omega,\ 0 \leq t < s \leq T$ we define 
 \begin{itemize}
  \item $C(\omega, s; t)$ -- a cash flow generated by the option at time $s$ in the scenario $\omega$ conditional on the option not being exercised at or prior to time $t$ and the investor following the optimal strategy afterwards,
  \item $F(\omega, t)$ -- the estimated future value of the option at time $t$ in the scenario $\omega$, conditional on the option not being exercised at or prior to time $t$ and the investor following the optimal strategy afterwards. We assume that $F$ depends on $\omega$ only through $S$ -- the asset price in the scenario $\omega$.
 \end{itemize}
\end{mydef}

\begin{remark}
 The above term $C(\omega, s; t)$ is similar to $D_{i,k}$ from paragraph \ref{subsec:LSMidea}. The main difference is that $C(\omega, s; t)$ emerges from following the optimal stopping, while $D_{i,k}$ is based on the strategy which is constructed during the operation of the algorithm.
\end{remark}
As in paragraph \ref{subsec:LSMidea} we assume that an option may be exercised in $K+1$ time points $0 = t_0 < t_1 < \ldots < t_K  = T$. At time $t_k < T$ the investor must decide whether to exercise the option. He or she must compare the value from immediate payoff with the value of continuation $F(\omega, t)$. Risk neutral valuation formula implies, that
\[ F(\omega, t_k) = \Em \Bigl[ \sum\limits_{j=k+1}^K e^{-r(t_j - t_k)}\cdot C(\omega, t_j; t_k) \Bpipe \mathcal{F}_k \Bigl]. \] 
\begin{remark}
 Of course the option may be exercised only once, hence only one element in the above sum is non-zero.
\end{remark}
We assume that $F(\omega, t) \in L^2$. Since $L^2$ is a Hilbert space, thus its elements can be represented as countable series, i.e. for some coefficients $a_0^{(t)},a_1^{(t)},\ldots$
\[ F(\omega, t) = \sum\limits_{j=0}^{\infty} a_j^{(t)} S_t(\omega)^j. \]
Here we use the simplest possible basis functions -- simple powers of the state variable, however, Longstaff and Schwartz in their article consider also different sets of basis functions, e.g. Laguerre or Hermit polynomials.

Since we cannot store an infinite series in a computer's memory, hence in the LSM algorithm we take only the first $M < \infty$ basis functions and denote this projection by $F_M(\omega, t)$, i.e.
\[ F_M(\omega, t) = \sum\limits_{j=0}^{M} a_j^{(t)} S_t(\omega)^j. \]
At each time $t_k$, the value $F_M(\omega, t_k)$ is unknown, but we approximate it with an estimator $\hat{F}_M(\omega, t_k)$ which is obtained by regressing the discounted values of $C(\omega, s; t_k)$ onto the basis functions.
In order to decrease the range of the region over which the estimation is performed, we take only in-the-money paths. It is irrelevant to have a good estimation of the future payoff on the out-of-the-money paths, because there we do not have to decide whether the option should be exercised. $\hat{F}_M(\omega, t_k)$ converges to $F_M(\omega, t)$ in $L^2$ and in probability, as the number of in-the-money paths goes to infinity.
The algorithm rolls back from $k=K$ to $k=0$ and when the exercise decisions are settled at each time point on each path, then the algorithm checks the realized cash flow at each path, calculates the mean, and returns it as the option's price.

Let $V(S)$ be the true value of the American option. The following two propositions come from~\cite{l-sch}.
\begin{prop}
 For any finite choice of $M, K$, and vector $\theta \in \mathbb{R}^{M \times (K-1)}$ representing the coefficients for the $M$ basis functions at each of the $K-1$ early exercise dates, let $LSM(\omega; M, K)$ denote the discounted cash flow resulting from following the LSM rule of exercising when the immediate exercise value is positive and greater than or equal to $\hat{F}_M(\omega_i, t_k)$ as defined by $\theta$. Then the following inequality holds almost surely,
 \[ V(S) \geq \lim_{N \rightarrow \infty} \frac{1}{N} \sum\limits_{i=1}^N LSM(\omega_i; M, K). \]
\end{prop}
\begin{prop}
 Assume that the value of an American option depends on a single state variable $S$ with support on $(0, \infty)$ which follows a Markov process. Assume further that the option can only be exercised at times $t_1$ and $t_2$, and that the conditional expectation function $F(\omega, t_1)$ is absolutely continuous and
 \[ \int_0^\infty e^{-S} F^2(\omega, t_1) dS < \infty,\ \ \ \ \  \int_0^\infty e^{-S} \left(\frac{\partial F}{\partial S}\right)^2(\omega, t_1) dS < \infty. \]
 Then for any $\varepsilon > 0$, there exists an $M < \infty$ such that
 \[ \lim_{N \rightarrow \infty} \P \Bigl[\bpipe V(S) - \frac{1}{N} \sum\limits_{i=1}^N LSM(\omega_i; M, K) \bpipe > \varepsilon \Bigr] = 0. \]
\end{prop}

The first proposition has a simple interpretation. The LSM algorithm constructs an exercise strategy and as the price of the option returns the mean of payoffs resulting from following that strategy. But the true value of the American option is maximized over all stopping times, hence for large numbers of simulations it should be greater than the price resulting from the LSM procedure.

The second proposition tells that in case $K = 2$ and under some technical assumptions the LSM method really works.

\section{LSM algorithm}
In this section we move to practical aspects of pricing American contingent claims.

\subsection{Regression}
It turned out that regression is an important step in LSM. We explained that the estimated future payoff may be represented by a polynomial function of $S_k$ whose coefficients may be obtained from the least squares method. We briefly describe the general method of finding the polynomial of degree $m$, best fitting given set of points $(x_i, y_i)$, $i=1,2,...n$.
\[ \begin{cases}
    a_0 + a_1 x_1 + \ldots + a_m x_1^m &= y_1 \\
    a_0 + a_1 x_2 + \ldots + a_m x_1^m &= y_2 \\
    \vdots & \\
    a_0 + a_1 x_n + \ldots + a_m x_n^m &= y_n
   \end{cases}
\]
We are minimizing a functional
\[ \psi(a_0,a_1,\ldots,a_m) = \sum\limits_{i=1}^n \left( y_i - \sum\limits_{j=0}^m a_jx_i^j \right)^2. \]
From basic analysis we know that $\psi$ may have minimum only in those points where partial derivative with respect to each variable equals 0, hence we are looking for such $a_0,a_1,\ldots,a_m$, that for each $k \in \{0,1\ldots,m\}$
\[ 0 = \frac{\partial \psi}{\partial a_k} = 2  \sum\limits_{i=1}^n \Bigl[ \bigl( y_i - \sum\limits_{j=0}^m a_jx_i^j \bigr) \bigl( -x_i^k \bigr) \Bigl]. \]
Simple calculation gives
\begin{equation}
 \label{eq:regression1}
 \sum\limits_{j=0}^m \left( a_j \sum\limits_{i=1}^n  x_i^{j+k} \right) = \sum\limits_{i=1}^n y_i x_i^k. 
\end{equation}
For each $k \in \{0,1\ldots,m\}$ and $l \in \{0,1\ldots,2m\}$ let
\[ s_{l} := \sum\limits_{i=1}^n  x_i^l \text{\ \ \ and \ \ } t_k := \sum\limits_{i=1}^n y_i x_i^k .\]
From (\ref{eq:regression1}) we obtain a system of linear equations
\begin{equation}
 \label{eq:regression2}
 \begin{cases}
  s_0 a_0 + s_1 a_1 + \ldots + s_m a_m &= t_0 \\
  s_1 a_0 + s_2 a_1 + \ldots + s_{m+1} a_m &= t_1 \\
    \vdots & \\
  s_m a_0 + s_{m+1} a_1 + \ldots + s_{2m} a_m &= t_m
  \end{cases}
\end{equation}
This system has $m+1$ variables and $m+1$ equations. It can be solved using the Gauss elimination method.
\begin{prop}
 The algorithm of approximating set of points $\{ (x_i,y_i) \}_{i=1}^n$ by a polynomial of degree $m$, which was described above, has the computational complexity \mbox{$\mathcal{O}(nm + m^3)$}.
\end{prop}
\begin{proof}
 For fixed $k$ computing $s_k$ or $t_k$ involves summation of $n$ values, and there is $\mathcal{O}(m)$ of $k$'s, hence the first part of the algorithm  has complexity $\mathcal{O}(nm)$. After that we have to solve the system of linear equations what can be done using the Gauss elimination method in time $\mathcal{O}(m^3)$.
\end{proof}

\subsection{Pseudocode}
We already have a base necessary to present the LSM method in the form of a pseudocode (Algorithm \ref{alg:LSM}).

\begin{algorithm}
 \begin{algorithmic}[1]
  \State CashFlow = \textbf{record}
    \State\ \ \ \ time: \textbf{real number}
    \State\ \ \ \ val: \textbf{real number}
  \State \textbf{end record} \\
  \\
  \textbf{input parameters:} $N$, $K$, $M$, $S_0$, $\sigma$, $r$, $T$, $E$
  \\
  \Function{LSM}{\null}
    \State $S \gets $ \Call{generateTrajectories}{\null}\Comment{{\color{comment} $S[i,k]$ is the asset price at $i$\textsuperscript{th} path at time $t_k$.}}
    \State $CF \gets$ \Call{getRealizedCashFlows}{$S$}
    \Comment{{\color{comment} $CF[i]$ is the cash flow obtained at the $i$\textsuperscript{th} path.}}
    \State $mean \gets$ \Call{getMean}{$CF$}
    \State $H_0 \gets$ \Call{getPayoff}{$S_0$}
    \State \Return $\max(mean, H_0)$
  \EndFunction
  \algstore{algLSM}
 \end{algorithmic}
 \caption{The valuation of an American option using LSM.}
 \label{alg:LSM}
\end{algorithm}
  
\begin{algorithm}[!ht]
 \begin{algorithmic}[1]
 \algrestore{algLSM}
  \Function{getRealizedCashFlows}{$S$}
    \For{$i = 1$ {\bf to} $N$}
      \State $CF[i].time \gets T$
      \Comment{{\color{comment} Realized cash flows conditional on }}
      \State $CF[i].val \gets $ \Call{getPayoff}{$S[i,K]$}
      \Comment{{\color{comment} not exercising before expiration. }}
    \EndFor
      \For{$k = K-1$ {\bf to} $1$} 
      \State $nrs \gets $ indices $i$ such that immediate payoff is positive
      \State $X \gets S[nrs, k]$
      \State $Y \gets \exp\bigl(-r\cdot(CF[nrs].time-t_k) \bigr)\cdot CF[nrs].val$
      \State $P \gets $ polynomial of degree $M$ resulting from regression $Y = P(X)$
      \State \Call{updateCashFlows}{\null}
    \EndFor
  \EndFunction \\
  
  \Function{updateCashFlows}{$k$, $S$, $CF$}
    \For{$i = 1$ {\bf to} $N$}
      \State $now \gets$ \Call{getPayoff}{$S[i,K]$}
      \If{$now > 0$ and $now > P(S[i,k])$}
	\State $CF[i].time \gets t_k$ \Comment{{\color{comment} Immediate exercise gives higher payoff}}
	\State $CF[i].val \gets now$ \Comment{{\color{comment} than expected payoff in the future.}}
      \EndIf
    \EndFor
  \EndFunction \\
  
  \Function{getMean}{$CF$}
    \State $sum \gets 0$
    \For{$i = 1$ {\bf to} $N$} 
      \State $sum \gets sum + \exp\bigl(-r\cdot CF[i].time \bigr) \cdot CF[i].val$
    \EndFor
    \State \Return $sum/N$
  \EndFunction
 \end{algorithmic}
\end{algorithm}
\newpage
\begin{prop}
 The computational complexity of Algorithm \ref{alg:LSM} equals $\mathcal{O}(KNM + KM^3)$, where $N$ is the number of simulated paths, $M$ is the degree of approximating polynomial, $K$ is the number of time steps.
\end{prop}
\begin{proof}
 Trajectories are generated in time $\mathcal{O}(KN)$. In order to construct the exercise strategy we have to $\mathcal{O}(K)$ times perform regression, which takes $\mathcal{O}(NM + M^3)$ operations. In total we do $\mathcal{O}(KNM + KM^3)$ operations. The last step, calculating the mean, is done in $\mathcal{O}(N)$.
\end{proof}

In practice $N$ is greater than $M^2$ and then the complexity is simply $\mathcal{O}(KNM)$. As we can see we made a huge progress between Algorithms \ref{alg:naiveProcedure} and \ref{alg:LSM}. We reduced the exponential complexity of ''Monte Carlo on Monte Carlo`` $\mathcal{O}(N^K)$ to $\mathcal{O}(KNM)$.

It turns out that now the bigger problem is the memory complexity. The algorithm requires generation of all trajectories at once, hence the complexity equals $\mathcal{O}(KN)$. Although an ordinary computer can execute LSM procedure with large numbers $K$ and $N$ in a reasonable time, it may run out of memory.

\begin{example}
 An implementation of Algorithm \ref{alg:LSM} was used to price American vanilla put under parameters: $K=100,\ M=3,\ S=100,\ \sigma=0.2,\ r=0.05,\ E=100,\ T=1$. Figure \ref{fig:LSM_res} presents a result cloud created by running the LSM procedure 100 times for $N \in \{10^3, 10^4, 10^5\}$. Table \ref{tab:LSM_res} compares results obtained by LSM with the true value calculated using the Finite Difference method. 
\end{example}

\begin{SCfigure}[][h]
\centering
 \includegraphics[scale=0.5]{images/LSM/put100.pdf}
\caption{The dispersion of the results obtained by executing the LSM procedure for different numbers of simulations. For each each $N$, the procedure was run 100 times. Hence for each $N$ we obtained 100 estimations of the price. Small points indicate obtained values. }
\label{fig:LSM_res}
\end{SCfigure}

\begin{table}[!ht]
\centering
 \caption{Results of pricing American put option under parameters listed in the text.}
 \label{tab:LSM_res}
\begin{tabular} {||c | c | c | c ||}  
 \hline 
  $\log_{10}(n)$ & FD price & LSM price & s.e. \\ \hline
  3 & 6.09 & 5.94 & 0.221 \\ \hline 
  4 & 6.09 & 6.20 & 0.071 \\ \hline 
  5 & 6.09 & 6.08 & 0.023 \\ \hline 
\end{tabular}  
\end{table}

Performed tests indicate that Finite Difference is better than LSM for vanilla option pricing. It gives more accurate results in shorter time and does not load the memory to the same extent as LSM. Therefore, we recommend using Finite Difference rather than LSM for American vanilla options.

\subsection{Variance reduction}
The variance reduction techniques described in the previous chapters may also be incorporated in the LSM.
\paragraph{Antithetic variates.} Involving antithetic variates is simple. The antithetic trajectories are generated identically as in the previous chapter (Algorithm \ref{alg:single-tr}). We only have to ensure that \Call{generateTrajectories}{} from line 9 in Algorithm \ref{alg:LSM} returns alternately the positive and negative trajectories from subsequent generations. 

\paragraph{Control variates.} The LSM algorithm may be divided into three parts: generating trajectories, building exercise strategy, collecting payoffs obtained by following constructed strategy on each path. Control variate method may be taken into account in the last part. 

After the second step we have a concrete stopping time $\tau$. What we do in the last part is a calculation of $\Em[H_\tau]$. For fixed $\tau$, $H_\tau$ is an ordinary random variable, so its expectation can be estimated using the methods described in section \ref{sec:introMC}. As a control variate we can take the value of a corresponding European option.

\section{LSM for exotic options}
As we mentioned before, Finite Difference seems to be better method for pricing American vanilla options. However, it is hard to adjust Finite Difference for exotic options, while LSM is quite flexible and may be used to price a vast class of instruments.
\subsection{Multi-asset options.}
From chapter III we know how to generate multi-asset trajectories. The main difficulty lies in the regression. The observations vector is constructed in a similar manner as in Algorithm \ref{alg:LSM}, line 23, let us denote it by $\mathbb{Z}$. However, preparation of the matrix of explanatory variables is more complicated. We have two state variables (asset prices) $X$, $Y$. We attempt to estimate the value of continuation using the polynomial of the form 
\[ P(X,Y) = \sum\limits_{i=0}^M \sum\limits_{j=0}^M a_{i,j} X^i Y^j. \]
Let
\begin{equation*}
 \mathbb{X} = \left[ \begin{array}{cccccccccccc}
           1 & Y_1 & \cdots & Y_1^M & X_1 & X_1 Y_1 & \cdots & X_1 Y_1^M & X_1^2 & X_1^2 Y_1 & \cdots & X_1^M Y_1^M \\
           1 & Y_2 & \cdots & Y_2^M & X_2 & X_2 Y_2 & \cdots & X_2 Y_2^M & X_2^2 & X_2^2 Y_2 & \cdots & X_2^M Y_2^M \\
           \vdots & \vdots & \ddots & \vdots \\
           1 & Y_N & \cdots & Y_N^M & X_N & X_N Y_N & \cdots & X_N Y_N^M & X_N^2 & X_N^2 Y_N & \cdots & X_N^M Y_N^M \\
          \end{array} \right],
\end{equation*}
\[ a = (a_{0,0},\ a_{0,1},\ \ldots,\ a_{0,M},\ a_{1,0},\ a_{1,1},\ \ldots,\ a_{M,M})'. \]
The regression problem can now be written in the form $\mathbb{Z} = \mathbb{X} a $, and the least squares method gives its solution:
\[ a = (\mathbb{X}'\mathbb{X})^{-1}\mathbb{X}' \mathbb{Z} \]

In theory, this procedure could be used for even higher dimensions, but unfortunately the number of coefficients grows exponentially. However, Longstaff and Schwartz mention in their article another procedure for approximating polynomial construction which allows to limit the coefficients to the number which grows only polynomially. 

\subsection{Path-dependent instruments.}
Algorithm \ref{alg:LSM} cannot be used directly to price most path-dependent options. Let us consider an Asian option. Suppose that at some time $t_k$ there exist such $i$ and $j$ that $S_{j,k} = S_{i,k}$, i.e. two paths have the same value at time $t_k$. For both of them Algorithm \ref{alg:LSM} would return the same value of the estimated future flow. However, it is possible that one of the trajectories for most of the time was above the current value, and the second one was below. Hence, the difference between the averages on both trajectories is high and in consequence also the value of continuation should be different. Clearly, Algorithm \ref{alg:LSM} does not work for that option.

Fortunately, there is an easy trick allowing to price such instruments using LSM. We just have to add second state variable which in case of the Asian option is the cumulative average price at the trajectory. The Asian option can imitate a derivative of two assets, where the first is the ''proper`` asset and the second is the average. Then it can be priced using the method from \textbf{Multi-asset options} paragraph.

This method can be used to price many other path-dependent instruments, for example lookback options.

\subsection{Barrier options.}
Barrier options are of course also path-dependent instruments, but it turns out that we do not need any modification to Algorithm \ref{alg:LSM} in order to price them. Consider for example knock-out options. Let us focus on a single trajectory and a time point $t_k$. The only aspect of the trajectory's history that matters in estimating the future flow, is the fact if the barrier was hit. If it was, then this trajectory will not be taken into account in the regression, because value of the immediate payoff equals zero. On the other hand, if the barrier was not hit, then the only factor influencing the future payoff is the current price.

\subsection{Forward and Bermuda options.}
\textbf{Bermudan options} are options which may be exercised only at the specified set of times. These instruments ``lie'' between European options, which may be exercised only at the expiry, and American options, which may be exercised at any occasion, just as Bermuda lies between Europe and America.

\textbf{Forward options} are a particular case of Bermudan options. They may be exercised only at the final period of the option's life. In the other words a forward option is an American option, whose exercise is not available until a specified time before the maturity.

Pricing such instruments does not require any special adjustments. They differ from the vanilla options in that they disable possibility of the exercise at some time points. Therefore, in that points the exercise will be postponed. In the rest of time points we have to calculate the estimated future payoff, but it depends only on the current price.



\chapter{Application architecture}
An inherent component of the thesis is an application for pricing miscellaneous financial instruments. The application was written in Java and is taking advantage of the best object oriented paradigm practices and design patterns. This chapter describes architecture of the created financial library which forms application's back end. The library is to some extent an implementation of algorithms presented in previous parts of the thesis, however, they are generalized and adjusted to the object oriented language.

\section{Scenarios and trajectories}
The basis of the Monte Carlo pricing is the trajectory generation. Hence, we begin with a presentation of classes performing this task. The corresponding class diagram is presented in Figure \ref{fig:arch:traj}.

\begin{sidewaysfigure}
\centering
 \includegraphics*[scale=1, viewport = 25 270 610 590]{images/Architecture/trajectories.pdf}
\caption{Class diagram presenting classes designated for scenario generation.}
\label{fig:arch:traj}
\end{sidewaysfigure}

Class \texttt{TimeSupport} is a simple auxiliary class allowing to switch between discrete and continuous time. Let $T$ be the final time on a trajectory and $K$ be the natural number such that the trajectory is generated in points $0 = t_0 < t_1 < \ldots < t_K$, where $t_k = k \cdot \Delta T$ and $\Delta T = T / K$. Method \texttt{nrToTime} from class \texttt{TimeSupport} for given $k$ returns $t_k$. Method \texttt{timeToNr} is somewhat an opposite and for given $t$ returns natural number $k$ such that $t_k \leq t < t_{k+1}$. 

\texttt{Trajectory} is an interface for classes representing realizations of price movements of one asset. Method \texttt{price} returns price of that asset at k\textsuperscript{th} time point. Interface provides also several auxiliary functions like \texttt{average}, which gives an average price of the option during the time between given time points, or \texttt{cumMax}, which returns asset's maximum price until given time point. These methods are useful for determining payoffs from instruments like Asian, barrier or lookback options. Class \texttt{ConcreteTrajectory} is an implementation of the \texttt{Trajectory} interface.

\texttt{Scenario} is an interface for classes representing realizations of whole market scenarios. A scenario may be considered as a set of trajectories of all assets. Method \texttt{getNumberOfAssets} answers how many assets are in the model. We can obtain trajectory of chosen asset by a call to \texttt{getTrajectory}, which takes as an argument the asset's name. A \texttt{TimeSupport} object obtained by a call to \texttt{getTimeSupport} represents set of points in which the trajectory was generated. Antithetic variates method makes a use from \texttt{getAnti} which returns antithetic scenario. \texttt{Scenario} is implemented by class \texttt{MultiTrScenario}.

When we have classes for holding scenarios and trajectories we can finally develop their generators. They must implement \texttt{Generator} interface, which provides a method \texttt{generate} returning an array of Scenarios. Class \texttt{MultiTrGenerator}, which is an implementation of \texttt{Generator}, follows Algorithm \ref{alg:multi-tr} from section \ref{sec:multi-asset}.

\section{Instrument hierarchy}
\lstset{language=Java, basicstyle=\small}

Classes representing financial instruments are arranged in a hierarchy based on the so-called decorator pattern. Idea of such construction arises from the fact that instruments like barrier options are just modifications of vanilla options. Moreover, we can add several modifications to one instrument. The decorator pattern allows us to take into account different sets of modifications without a need to create a separate class for each set. Figure \ref{fig:arch:instr} shows the instrument hierarchy.
\begin{sidewaysfigure}
\centering
 \includegraphics*[scale=1, viewport = 25 285 580 595]{images/Architecture/instr.pdf}
\caption{Class diagram presenting the hierarchy of financial instruments.}
\label{fig:arch:instr}
\end{sidewaysfigure}


\texttt{Instr} is an abstract class representing a general financial instrument. It stores a field \texttt{T} which holds time to the expiration. Classes deriving from \texttt{Instr} must implement methods \texttt{exAvail} and \texttt{payoff\_}. The first method, for specified scenario and time point, answers if the exercise is available. For example an European option may be exercised if and only if given time point is the last one on the trajectory. On the other hand, exercise of knock-out option is possible if and only if barrier was not hit in given Scenario. Method \texttt{payoff\_} returns a real number which means what would be the payoff obtained in given scenario if the exercise was possible and the instrument was exercised at given time point. Method \texttt{price} from class \texttt{Instr} uses both aforementioned methods:
\begin{lstlisting}
  public double payoff(Scenario s, int k) {
      if (exAvail(s,k)) return payoff_(s,k);
      else return 0;
  }
\end{lstlisting}
Clearly, \texttt{payoff} returns the value which would be obtained from the immediate exercise, it the exercise is possible, otherwise it returns zero. Since \texttt{payoff} is a ``sketch'', which delegates part of the logic to subclasses, hence we may say that we used here the factory method pattern. 

\texttt{Bond} and \texttt{Option} are concrete instruments. As the name indicates \texttt{Bond} represents instruments paying \texttt{nominal} after time \texttt{T}. Implementation of the methods from class \texttt{Instr} looks as follows: 
\begin{lstlisting}
  protected boolean exAvail(Scenario s, int k) {        
      return s.getTS().getK() == k;
  }
  protected double payoff_(Scenario s, int k) {
      return nominal;
  }
\end{lstlisting}

	
Class \texttt{Option} is designed for American options. It stores information about the strike price, its type (call or put) and name of the underlying asset. American vanilla option may be exercised at any time, hence \texttt{exAvail} has the form:
\begin{lstlisting}
  protected boolean exAvail(Scenario s, int k) {
      return true;
  }
\end{lstlisting}
Implementation of \texttt{payoff} is also not complicated:
\begin{lstlisting}
  protected double payoff_(Scenario s, int k) {
      Trajectory tr = s.getTr(underlying);
      double spot = tr.price(k);
      if (type == CALL) return max(0, spot - strike);
      else return max(0, strike - spot);
  }
\end{lstlisting}

\texttt{Modificator} class is the place where the decorator pattern does its job. \texttt{Modificator} is a wrapper for another instrument, which allows for a modification of its payoff and exercise availability. \texttt{EuExercise} and \texttt{Barrier}, as well as many other modifications, do not change the payoff, hence \texttt{payoff\_} is coded in \texttt{Modificator} as follows (of course it may be changed in subclasses):
\begin{lstlisting}
  protected double payoff_(Scenario s, int k) {
      return wrapped.payoff_(s, k);
  }
\end{lstlisting}
On the other hand, while determining if the exercise is available we have to take into account that an instrument may have many modifications. We are allowed to exercise the option if all \texttt{Modificator}s agree. Hence, we introduce new abstract method \texttt{modExAvail}, which checks if a single modification permits for the option exercise, and we define \texttt{exAvail} as
\begin{lstlisting}
  protected final boolean exAvail(Scenario s, int k) {
      return wrapped.exAvail(s, k) && modExAvail(s,k);
  }
\end{lstlisting}
That way, if the instrument has many modifications, all of them are checked.

Let us discuss it on several examples. In class \texttt{EuExercise} we modify the instrument in such a way, that it may be exercised only in the final time point, hence
\begin{lstlisting}
  protected boolean modExAvail(Scenario s, int k) {
      return s.getTS().getK() == k;
  }
\end{lstlisting}
Class \texttt{Barrier} checks if the barrier was hit and then, depending on whether the option is knock-out or knock-in, it allows for the option's exercise or not. Object \texttt{bp}, which is used below, stores the information about the type of the barrier and its level.
%\begin{absolutelynopagebreak}
\begin{lstlisting}
  public boolean modExAvail(Scenario s, int k)
  {
      Trajectory tr = s.getTr(underlying);
      boolean hit;
      if (bp.isFromUp()) hit = tr.cumMax(k) >= bp.level;
      else hit = tr.cumMin(k) <= bp.level;
      return (bp.isKnockIn() ? hit : !hit);
  }
\end{lstlisting}
%\end{absolutelynopagebreak}

Class \texttt{Binary} does not change the exercise availability, thus its \texttt{modExAvail} simply returns \texttt{true}, but the payoff function is modified:
\begin{lstlisting}
  protected double payoff(Scenario s, int k) {
      return wrapped.payoff(s, k) > 0 ? 1 : 0;
    }
\end{lstlisting}
	
This design may look overly complicated. Let us explain what is its advantage. Suppose that at some point we will have a need to price not only typical barrier options, but also options with double barrier, for example knock-out corridor options, which has zero payoff if the stock price goes beyond some specified range during the option's lifetime. Our design lets us to instantiate an object representing such instrument without necessity to write any additional classes! We just type\footnote{Constructors' arguments which are irrelevant for the example were omitted.}:
\begin{lstlisting}
  Instr corridor = new Barrier(new Barrier(new Option()));
\end{lstlisting}
That's it! What if we would like to price European binary option with a double barrier? Do we need some additional code? Of course not, we simply write
\begin{lstlisting}
  Instr veryExotic = new EuExercise(new Barrier(new Barrier(
				  new Binary(new Option()))));
\end{lstlisting}
If we had to create special classes for such instruments as the one mentioned above, then soon we would sink in the sea of instrument classes and the code would become unmaintainable.
	
\section{Method hierarchy}
In the thesis several versions of the Monte Carlo algorithm were presented. Methods have their own class hierarchy, on the top of which lies an interface named not other than \texttt{Method}. It has only one method\footnote{Coincidental name clash.} called \texttt{price}. It takes an arbitrary \texttt{Instr} and returns its price as a number of type \texttt{double}.

\subsection{"Classic" Monte Carlo}
Every variant of the Monte Carlo has its own representation in the methods' class hierarchy shown in Figure \ref{fig:arch:MonteCarlo}.
\begin{figure}
\centering
 \includegraphics*[scale=0.75, viewport = 20 405 575 750]{images/Architecture/MonteCarlo2.pdf}
\caption{Class diagram presenting hierarchy of classes used for option pricing.}
\label{fig:arch:MonteCarlo}
\end{figure}

\texttt{MonteCarlo} implements \texttt{Method}, however, it is still an abstract class. It forms a basis for classes \texttt{CMC}, \texttt{AV} and \texttt{CV}. \texttt{MonteCarlo} takes advantage of the factory method pattern. It implements \texttt{price}, however, the implementation calls abstract methods \texttt{initPricing},\\ \texttt{oneSimulation}, \texttt{currentResult}:
\begin{absolutelynopagebreak}
\begin{lstlisting}
  public double price(Instr instr) throws InterruptedException {
      initPricing(instr);
      for (int i = 1 ; i <= N; ++i) {
	  oneSimulation(instr);
	  // other instructions responsible for notifying progress
	  // observers, handling interruption, preparing data
	  // for convargance plot.
      }
      return currentResult(N);
  }
\end{lstlisting}
\end{absolutelynopagebreak}
   
This design extracts all the actions which are common for each method, and allows subclasses to perform some particular operations in their specific way. For example method \texttt{currentResult}, which returns intermediate results after given number of simulations, simply calculates mean in case of \texttt{CMC} and \texttt{AV}, but in \texttt{CV} it also determines value $c$ (recall equation (\ref{eq:CV})). Method \texttt{oneSimulation} also differs between classes -- in \texttt{CMC} it has the form

\begin{lstlisting}
  protected void oneSimulation(Instr instr) {
      Scenario scenario = gen.generate();
      double p = getDiscountedPayoff(instr, scenario);
      sum += p;
      sumSq += p*p;
  }
\end{lstlisting}
but in \texttt{AV} it uses antithetic variates, and in \texttt{CV} it calculates also the value of the control variate.

Here we can see the incredible power of the Monte Carlo method, especially when it is joined with polymorphism in OO programming languages. The \textbf{pricing algorithm is entirely independent from the instrument under valuation.} The only place where the properties of the instrument are taken into account is a method determining the realized payoff, e.g. in \texttt{CMC} it has the form

\begin{lstlisting}
  private double getDiscountedPayoff(Instr instr, Scenario scenario)
  { 
      return instr.payoff(scenario, K) *
	      Math.exp(-params.getR()*instr.getT());
  }
\end{lstlisting}

The payoff is obtained via a polymorphic call to \texttt{payoff} from class \texttt{Instr}. It has an important and very pleasant consequence. When we come up with an idea of a completely new contract, we do not have to implement any new class for performing valuation. The only thing we have to do is create a class for that contract as a member of \texttt{Instr} hierarchy. Note that most pricing algorithms do not have that property, they require specific adjustments to price some instruments, e.g. in the finite difference method one has to ensure that a barrier is lying on the grid if he is pricing a barrier option.
   

\subsection{LSM}
The diagram in Figure \ref{fig:arch:LSM} is a LSM branch of the \texttt{Method} hierarchy.

An abstract class \texttt{LSMRoot} is an implementation of \texttt{Method} which employs Least Squares Monte Carlo. Unfortunately,  it is not as universal as the ordinary Monte Carlo, because, for example, path-dependent options require  special treatment. Thus, several variants of LSM were implemented: \texttt{LSM} is a basic version of LSM, which follows the original algorithm, \texttt{LSM2} is for options depending on two assets, for example basket options, \texttt{LSM\_Asian} can price Asian options.
However, while implementing these methods we also strived to avoid code repetition, so the factory method pattern was involved once again. The following code snippet presents how the \texttt{price} method was implemented in \texttt{LSMRoot}.

\begin{lstlisting}
  public double price(Instr instr) throws InterruptedException
  {       
      initPricing(instr);
      Generator gen = makeGenerator();
      Scenario[] paths = gen.generate(N);
      CashFlow[] realizedCF = realizedCashFlows(paths, instr);
      double payoffAtTime0 = instr.payoff(paths[0], 0);
      Result res = makeResult(realizedCF);
      return Math.max(res.result, payoffAtTime0);
  }
\end{lstlisting} 
\begin{absolutelynopagebreak}
\begin{lstlisting}
  protected CashFlow[] realizedCashFlows(Scenario[] paths, Instr instr)
	  throws InterruptedException
  {
      CashFlow[] realizedCF = new CashFlow[N];
      for (int i = 0; i < N; ++i)
	  realizedCF[i] = new CashFlow(instr.payoff(paths[i], K), getT());
      for (int j = K-1; j > 0; --j)
      {
	  FutureEstimatedFlow fef = calcFutureEstimatedFlow(
	      paths, realizedCF, j, instr);
	  updateRealizedCFs(paths, instr, realizedCF, fef, j);
	  // other instructions responsible for notifying progress
	  // observers, handling interruption
      }       
      return realizedCF;
  }
\end{lstlisting} 
\end{absolutelynopagebreak}

\begin{figure}
\centering
 \includegraphics*[scale=0.6, viewport = 5 130 670 575]{images/Architecture/LSM2.pdf}
\caption{Class diagram presenting hierarchy of classes used for option pricing.}
\label{fig:arch:LSM}
\end{figure}
\newpage

The general scheme is the same for all variants of LSM. Classes inheriting \texttt{LSMRoot} must implement three methods. In \texttt{initPricing} subclasses can set auxiliary variables specific for their variant of LSM. Method \texttt{makeGenerator} returns generator of scenarios. For example, the version of LSM for Asian options returns \texttt{Generator} which generates scenarios that have two trajectories: one is the proper asset price and the second is the cumulative average. The main difference between LSM for options depending on one and two assets is the estimation of the future flow. This task is done in the third abstract method of \texttt{LSM} -- \texttt{calcFutureEstimatedFlow}. In case of the basic LSM only one explanatory variable is used in regression, while in LSM for two assets or for path-dependent options future estimated flow is a function of two variables. Introduction of \texttt{calcFutureEstimatedFlow} allows to isolate these differences. It returns an object of a class implementing \texttt{FutureEstimatedFlow}. It has a single method \texttt{getEstimation} which for given scenario and time step returns estimated value of continuation. With its help updating the table of realized cash flows is very convenient:
\begin{lstlisting}
  private void updateRealizedCFs(Scenario[] paths, Instr instr,
      CashFlow[] realizedCF, FutureEstimatedFlow fef, int timeStep)
  {
      for (int i = 0; i < N; ++i)
      {
	  double expected = fef.getEstimation(paths[i], timeStep);
	  double payoff = instr.payoff(paths[i], timeStep);
	  if (payoff > 0 && payoff > expected)
	      realizedCF[i] = new CashFlow(payoff,ts.nrToTime(timeStep));
      }
  }
\end{lstlisting}

Since regression is very similar for those versions of LSM which use two explanatory variables, so class \texttt{LSM2\_Support} was designed. It performs the calculation of the estimated future flow basing on two assets. Classes like \texttt{LSM2} or \texttt{LSM\_Asian} simply delegate \texttt{calcFutureEstimatedFlow} to \texttt{LSM2\_Support}.
   
As we can see, in case of LSM we do not have the property that every new instrument can be priced instantaneously. Of course we can pass every \texttt{Instr} to \texttt{LSM}'s \texttt{price} method, but it is possible that the result may be entirely inadequate. For some instruments it may be compulsory to implement a separate version of LSM. However, we hope that the interface, which is already provided, makes this task easy. All the programmer has to do is inherit \texttt{LSMRoot} and implement three abstract methods, what should not be a big effort.

\chapter{Results illustrations}
The last chapter presents pricing results of a bunch of instruments, received from the created program. Collected values were moved to \texttt{R} code, which drew presented charts using \texttt{ggplot2} library.

\begin{remark}
 Many charts have bothering irregularities in their shape. Keep in mind that they were obtained by simulation, which is vitiated by an error, what causes these plots to be not as smooth as they should be.
\end{remark}


\section{Options on a dividend-paying stock}
From Proposition \ref{prop:amCall} we know that the value of an American call on a share of a non-dividend paying stock is equal to its European counterpart. 
It turns out that it may be optimal to exercise an American call if the underlying pays a dividend.

In Figure \ref{fig:results:vanillaAmCallPriceBeforeDividend} we can see the immediate payoff and the option's value as functions of the stock price just before the dividend payment. Both lines coincide for big values of the asset price. From chapter IV we know that in that case the option should be exercised immediately. 

% \begin{SCfigure}
% \centering
%  \includegraphics[scale=0.5]{images/Results/vanillaEuAndAmPut.pdf}
% \caption{Comparison of prices of vanilla European and American puts depending on the spot price. Parameters: $vol = 30\%$, $r=8\%$, strike $=100$, $T=1$.}
% \label{fig:results:vanillaEuAndAmPut}
% \end{SCfigure}

\begin{figure}[!htb]
\centering
 \includegraphics[scale=0.5]{images/Results/vanillaAmCallPriceBeforeDividend.pdf}
\caption{The comparison of the option's value and the immediate payoff just before a dividend. The stock pays a dividend equal to 5\% of its price at time $0.5$. Parameters: $vol = 30\%$, $r=5\%$, $E=100$, $T=1$. }
\label{fig:results:vanillaAmCallPriceBeforeDividend}
\end{figure}

% \begin{figure}
%  \parbox{.45\linewidth} {
%    \centering
%    \caption{Paths simulated under the risk-neutral measure.}
%    \label{tab:LSM_paths}
%    \includegraphics[scale=0.4]{images/Results/vanillaEuAndAmPut.pdf}
%  }
%  \qquad 
%  \parbox{.45\linewidth} {
%   \centering
%   \caption{Cash flows obtained conditional on not exercising the option before the maturity.}
%   \label{tab:LSM_cashflows_t3}
%   \includegraphics[scale=0.4]{images/Results/vanillaAmCallDividend.pdf}
% }
% \end{figure}

In Figure \ref{fig:results:vanillaAmCallDividend} we consider an American call option on a stock paying two dividends, the first at time $\frac{1}{3}$ and the second at time $\frac{2}{3}$. We look at the cases for which the dividends equal 1, 5, and 10 percents of the stock's price. The chart presents the option's price at time~0 as a function of the spot price. Naturally, the option's value is the greatest when the stock does not pay a dividend -- the dividend lowers the asset price, what does not favour the owner of a call option. The greater the dividend is, the lower should be the option's value. It is seen especially for low spot prices. Surprisingly, for large values of the initial asset price, the option's price becomes equal for each dividend yield. It has a simple explanation. If the asset price is big enough at the day of the first dividend payment, then for each dividend yield the optimal strategy is to exercise the option. For the 1\% dividend yield the decision to exercise the option immediately becomes optimal for very high asset prices. However, for 5\% and 10\% dividend yields the option's owner should exercise it for lower asset prices, therefore corresponding plot lines start to coincide much earlier.


\begin{figure}[!htb]
\centering
 \includegraphics[scale=0.6]{images/Results/vanillaAmCallDividend.pdf}
\caption{The comparison of the option's prices depending on the value of the dividend yield. Dividends are paid at time $1/3$ and $2/3$. Parameters: $\sigma = 20\%$, $r=5\%$, $E=100$, $T=1$.}
\label{fig:results:vanillaAmCallDividend}
\end{figure}

The chart in Figure \ref{fig:results:vanillaAmCallDividend3D} illustrates how the price of the option  changes over time, if the stock pays a 20\% dividend at time $0.5$. We can see a sharp jump at the time of the dividend. This observation is not surprising. Suppose that at time $0.499$ the asset price equals $S$. Due to the dividend payment, soon the asset price will drop to $0.8S$. On the other hand, at time $0.501$ we are already after the dividend payment, so if the asset price equals $S$, then the expectation of the future asset price is greater, therefore, the option's value is significantly greater.


\begin{figure}[!htb]
\centering
 \includegraphics[scale=0.5]{images/Results/vanillaAmCallDividend3D.pdf}
\caption{The value of the option as a function of time and asset price. The stock pays a 20\% dividend at time $0.5$. The chart was generated under parameters $\sigma = 20\%$, $r=5\%$, $E=100$, $T=1$.}
\label{fig:results:vanillaAmCallDividend3D}
\end{figure}

\begin{remark}
 In Figure \ref{fig:results:vanillaAmCallDividend3D} we can observe strange values for low asset prices. To explain their origin we have to describe how this chart is created. The LSM algorithm at each time step calculates the polynomial approximating the estimated future cash flow. To create the chart, for each pair of time $t$ and asset price $S$, we take the polynomial calculated at time $t$, and as the option's price we take the maximum of the immediate payoff and the polynomial's value for argument $S$. This method gives satisfactory results for high asset prices. Of course, the approximating polynomial is vitiated by an error, what causes the lines in the chart to be rugged, but it does not harm a general overview. However, remember that the regression is performed only for those asset prices where the immediate payoff is positive. Therefore, the polynomials approximating the value of continuation have garbage values for asset prices below the strike price (in case of a call option) and we see them in the chart. 
 
 The same remark concerns Figure \ref{fig:results:vanillaAmCallPriceBeforeDividend}. The value of the option at time $0.5$ was calculated using the approximating polynomial, hence we prepared the chart only for high asset prices, due to the occurrence of garbage values for lower prices. However, this issue does not affect Figure \ref{fig:results:vanillaAmCallDividend}. It presents prices at time~0, thus the approximating polynomials were not used, instead the algorithm was run separately for each spot price.
\end{remark}

\section{Forward options}
As we know from Proposition \ref{prop:amCall}, a call option on a non-dividend paying stock should be exercised only at the expiration date, therefore, the forward feature does not influence the optimal stopping strategy. The left chart in Figure \ref{fig:results:forward} confirms this observation -- prices of forward and vanilla calls entirely coincide.

In case of forward puts, sometimes the owner is forced to postpone the exercise, even though he or she would exercise an ordinary American put. This inconvenience decreases the option's value, what we can see in the right chart of Figure \ref{fig:results:forward}. The sooner the exercise is available, the greater is the price of the option.

\begin{figure}[!htb]
\centering
 \includegraphics[scale=0.45]{images/Results/forwardCall.pdf}
 \includegraphics[scale=0.45]{images/Results/forwardPut.pdf}
\caption{Prices forward options of in comparison to vanilla options. Parameters: $\sigma = 30\%$, $r=10\%$, $E=100$, $T=1$.}
\label{fig:results:forward}
\end{figure}


\section{Barrier options}
Barriers have a great influence on the option's value. Surprisingly, price functions of European and American barrier options have totally different properties.

Let us begin with European claims. We know that in case of vanilla options, the higher is the spot price, the greater is the value of the option. The left chart in Figure \ref{fig:results:barrierEu} shows that the same does not hold for barrier options. We see there the price function of an up-and-out call option. For small spots the price function is increasing, at some point it reaches the maximum, then the function decreases, and finally, for spot equal the barrier, the price sinks down to zero. This behavior is understandable -- if the spot price is high, close to the barrier, then the risk of hitting the barrier is high, and in consequence the option's value is low. The right chart gives an idea of the difference between the prices of barrier and vanilla options.
\begin{figure}[!ht]
\centering
 \includegraphics[scale=0.45]{images/Results/barrierEuComp.pdf}
 \includegraphics[scale=0.45]{images/Results/barrierAndVanillaEu.pdf}
\caption{Prices of an European up-and-out call option. The barrier on the right chart equals 140. Other parameters: $\sigma = 30\%$, $r=0$, $E=100$, $T=1$.}
\label{fig:results:barrierEu}
\end{figure}

American claims are not as affected by knock-out barriers as their European counterparts. Of course, even in the American case, prices of knock-out options are lower then vanilla options, due to the fact that barriers prevent large payoffs. However, an American option may be exercised any time, therefore, there is no risk of hitting the barrier (unless the spot price is already above the barrier), and in consequence the price of barrier options is not much lower than the price of vanilla options. Foregoing remarks are illustrated in Figure \ref{fig:results:barrierAm}.
\begin{figure}[!ht]
\centering
 \includegraphics[scale=0.45]{images/Results/barrierAmComp.pdf}
 \includegraphics[scale=0.45]{images/Results/barrierAndVanillaAm.pdf}
\caption{Prices of an American up-and-out call option. As long as the spot price is below the barrier, the value of the barrier option is not much lower than the value of the vanilla option. It is only for asset prices greater than the barrier were the value is cut down to zero. The barrier on the right chart equals 140. Other parameters: $\sigma = 30\%$, $r=0$, $E=100$, $T=1$.}
\label{fig:results:barrierAm}
\end{figure}

Figures \ref{fig:results:barrierEuCall3D} and \ref{fig:results:barrierAmCall3D} picturize prices of up-and-out call option, the first figure in the European case, and the second in the American case. Notice that price of the European derivative smoothly decreases for high spot prices, while the American version has a sharp jump.

\begin{figure}[!ht]
\centering
 \includegraphics[scale=0.5]{images/Results/barrierCall3D.pdf}
\caption{The value of an European up-and-out call as a function of the spot price and the barrier level. Parameters: $\sigma = 30\%$, $r=0$, $E=100$, $T=1$.}
\label{fig:results:barrierEuCall3D}
\end{figure}
\begin{figure}[!ht]
\centering
 \includegraphics[scale=0.4]{images/Results/barrierAmCall3D_1.pdf}
 \includegraphics[scale=0.4]{images/Results/barrierAmCall3D_2.pdf}
\caption{The value of an American up-and-out call as a function of the spot price and the barrier level, viewed from two angles. ``Steps'' seen on both charts are caused by the fact that they were created in a finite number of points. If we could calculate the price of the option in continuum points, then the shape would be smooth. Parameters: $\sigma = 30\%$, $r=0$, $E=100$, $T=1$.}
\label{fig:results:barrierAmCall3D}
\end{figure}

\FloatBarrier
It turns out that high volatility also does not work in favor of European knock-out options, what can be seen in Figure \ref{fig:results:barrierEuPut3D}. Sudden jumps increase the risk of hitting the barrier. In case of American options we can only benefit from high volatility. It allows the asset price to get close to the barrier, and then it may be exercised. In Figure \ref{fig:results:barrierAmPut3D} we can see the value of an option analogical to that on Figure \ref{fig:results:barrierEuPut3D}, but with an American exercise feature. Its price is an increasing function of the volatility.

\begin{figure}[!htb]
\centering
 \includegraphics[scale=0.4]{images/Results/barrierEuPut3D_1.pdf}
 \includegraphics[scale=0.4]{images/Results/barrierEuPut3D_2.pdf}
\caption{The value of an European down-and-out put as a function of the spot price and the volatility, viewed from two angles. Parameters: $r=0$, $E=100$, $T=1$, barrier level $=150$.}
\label{fig:results:barrierEuPut3D}
\end{figure}
\begin{figure}[!htb]
\centering
 \includegraphics[scale=0.4]{images/Results/barrierAmPut3D_1.pdf}
 \includegraphics[scale=0.4]{images/Results/barrierAmPut3D_2.pdf}
\caption{The value of an American down-and-out put as a function of the spot price and the volatility, viewed from two angles. Parameters: $r=0$, $E=100$, $T=1$, barrier level $=150$.}
\label{fig:results:barrierAmPut3D}
\end{figure}

Let us mention one more aspect of barrier option pricing. In order to value such path-dependent instruments, asset prices must be generated in multiple points before the expiry. The chosen number of time steps has a significant influence on the option's price. Look at Figure~\ref{fig:results:barrierTimeSteps}. For a low number of time steps, the value of an knock-in option is higly underestimated. That is because the asset price may hit the barrier between time points and we do not reflect it in simulations. Note that as the number of time steps increases, the estimated value approaches the price resulting from continous model. For knock-out option the situation is opposite -- a small number of time steps decreases the chance of ``turning off'' the option, hence its price is overestimated.

\begin{figure}[!htb]
\centering
 \includegraphics[scale=0.5]{images/Results/barrierTimeStepsCallUAI.pdf}
 \includegraphics[scale=0.5]{images/Results/barrierTimeStepsPutDAO.pdf}
\caption{Prices of barrier options estimated for different barrier levels and numbers of intermediate times steps. In the picture on the left we see results for up-and-in call option, and on the right for down-and-out put. Orange lines indicate the value calculated from analytical formula. Parameters: $E=100,\ T=1,\ S=100,\ r=5\%,\ \sigma=20\%$ (on the left), $\sigma=40\%$ (on the right).}
\label{fig:results:barrierTimeSteps}
\end{figure}

\section{Binary options}
The payoff from the binary option (also called digital option) always equals 1 or 0. The owner obtains the payoff when the asset price is higher than the strike, in case of a call, or when the asset price is lower than the strike, in case of a put. 

Figure \ref{fig:results:binaryPut} shows how the price of such an instrument looks. We see that the value of an European binary put forms a slope from 1 to 0. The lower the volatility, the sharper the slope. It has the following explanation. If the volatility is low and the asset price is far from the strike then the chances of changing the moneyness of the option are very slim. If the volatility is high, then the price line is flattened, because even if the option is out-of-the money, it still has a good chance to end in-the-money, thus the price is much above zero. On the other hand, even if the option is in-the-money, the price is still much lower than 1, because a rapid jump, which may occur as a consequence of high volatility, could move the final asset price at the ``wrong'' side of the strike.

In case of an American binary option the situation is similar. However, note that the option's value becomes 1 at the strike. Obviously, under assumption that the interest rate is non-negative, the optimal strategy of exercising an American binary option is to exercise at the first time when it is in-the-money.
\begin{figure}[!htb]
\centering
 \includegraphics[scale=0.45]{images/Results/binaryEuPut.pdf}
 \includegraphics[scale=0.45]{images/Results/binaryAmPut.pdf}
\caption{Prices of binary puts. On the left -- European option, on the right -- American. Parameters: $r=0$, $E=100$, $T=1$.}
\label{fig:results:binaryPut}
\end{figure}

In Figure \ref{fig:results:binaryCall} we can see how values of binary options change over time.
\begin{figure}[!htb]
\centering
 \includegraphics[scale=0.4]{images/Results/binaryEuCall3D.pdf}
 \includegraphics[scale=0.4]{images/Results/binaryAmCall3D.pdf}
\caption{Prices of binary calls. On the left -- European option, on the right -- American. Parameters: $\sigma = 0.2$, $r=0$, $E=100$, $T=1$.}
\label{fig:results:binaryCall}
\end{figure}

\section{Asian and lookback options}
Lookback options allow its owner to buy or sell (depending on the type) the asset for the most beneficial price. Such an instrument favors the buyer, so obviously, it must be expensive. Asian options are to some extent an opposite, someone might regard them as the most fair contract between buyer and seller. Asian options prevent large payoffs, thus they are relatively cheap. These remarks are confirmed by a chart in the Figure~\ref{fig:results:asianLookbackComp}.

\begin{figure}[!htb]
\centering
 \includegraphics[scale=0.5]{images/Results/asianLookbackComp.pdf}
\caption{The comparison of the prices of European call options of different kinds. The strike of vanilla option equals the spot price. Parameters: $T=1,\ r=5\%,\ \sigma=20\%$.}
\label{fig:results:asianLookbackComp}
\end{figure}

In case of Asian and lookback options we may ask the same question which bothered us at the barrier options section: how does the number of time steps influence the estimated price? Several results are gathered in Figure \ref{fig:results:asianTimeSteps}.

\begin{figure}[!htb]
\centering
 \includegraphics[scale=0.5]{images/Results/asianTimeSteps.pdf}
 \includegraphics[scale=0.5]{images/Results/lookbackTimeSteps.pdf}
\caption{Parameters: $E=100,\ T=1,\ S=100,\ r=5\%,\ \sigma=20\%$.}
\label{fig:results:asianTimeSteps}
\end{figure}

As we could expect, Asian options, especially with European exercise, are not strongly affected by the number of time points. High asset price jumps, which could possibly happen between time steps, are overcome by taking the average. Price of the American-Asian option is slightly rising if the option may be exercised more often, however, even in this case the price is quickly stabilizing.

Prices of lookback options are visibly rising, as the number of time steps increases. It reflects the fact the asset price takes the maximum and minimum value between the time points. Thus, as we thicken the time points, the extreme value is estimated more accurately. Therefore, the calculated price is closer to the value of continuously exerciseable instrument. Note by the way, that the price of an American-lookback option is significantly greater than its European counterpart. Someone might think that the option should not be exercised before expiry, in order to take maximum or minimum over the longer interval. However, obtained results show that it is not true. Let us focus on a lookback put which gives a right to sell an asset for the maximum price. At the beginning the asset price could have a peak and then rapidly sink down. In this situation it may be optimal to exercise the option immediately, while the asset prices are low.

\section{Pricing real market options}
So far we presented only pricing results for some preconceived parameters. The created application would be worthless if it could not price the instruments existing in real markets.

We present valuation of options on two worldwide known companies, Facebook and Google, traded in Nasdaq. Basing on historical prices from 30 August 2012 to 30 August 2013, the application determined following parameters:
\begin{align*}
S_F &= 41.29,\ \sigma_F = 0.519,\ \mu_F = 0.913,\\
 S_G &= 846.9,\ \sigma_G = 0.198,\ \mu_G =0.238,\\
 \rho &= 0.076 
\end{align*}
As the interest rate we take $r = 0.002$. The valuation is performed from a perspective of day 30 August 2013 and considered options expire at 21 December 2013, thus we have $T = 0.31$. 

In Figures \ref{fig:results:realFacebook} and \ref{fig:results:realGoogle} we can see comparisons of market prices to the values calculated by an application. We observe some differences, but they are not big. They can be a consequence of miscellaneous reasons, for example different models, or at least different parameters. Moreover, prices on real markets adjust to the demand, therefore, they are never exactly the same as in the mathematical model.
\begin{figure}[!htb]
\centering
 \includegraphics[scale=0.5]{images/Results/realFacebookPut.pdf}
\caption{The comparison of calculated and market prices of put options on Facebook.}
\label{fig:results:realFacebook}
\end{figure}
\begin{figure}[!htb]
\centering
 \includegraphics[scale=0.5]{images/Results/realGoogleCall.pdf}
\caption{The comparison of calculated and market prices of call options on Google.}
\label{fig:results:realGoogle}
\end{figure}

\chapter*{Concluding remarks}
\addcontentsline{toc}{chapter}{\bfseries Concluding remarks}
In the thesis we explained how Monte Carlo methods may be used for option pricing. Let us sum up their pros and cons.

In practice every European contingent claim may be valued by the ordinary Monte Carlo method. It is easy to implement and very universal, since the algorithm implemented once may be used to price every new option. This is the greatest advantage of the Monte Carlo method -- preparing new instruments for pricing requires almost no effort. The main disadvantage is time consumption. Convergence of the Monte Carlo estimator is slow, hence obtaining desired accuracy may take a huge number of simulations. Moreover, valuation of path-dependent options requires simulating the trajectory in many points prior to the expiration date. We showed in the thesis that the number of time steps in which trajectories are simulated highly influences  the resulting price. Thus, we may be interested in lowering the number of necessary replications. For this purpose the described variance reduction techniques may be used. When pricing simple instruments like vanilla options it is of course better use analytical formulas. However, for exotic options Monte Carlo may be the best choice.


The Least Squares Monte Carlo method allows to price American options. Its implementation takes much more effort, since it involves regression and handling multi-argument polynomials (in case of multi-asset options), whose numbers of coefficients rapidly grows as the dimensionality of the problem increases. Unfortunately, this algorithm is not that universal as an ordinary Monte Carlo. For some instruments we have to write separate versions of LSM. However, the scheme is always the same and adjusting it to a particular instrument is usually not too difficult. LSM inherits all disadvantages of Monte Carlo. It requires simulations of large numbers of trajectories which must be generated in many times steps. Moreover, in LSM all market scenarios must be generated at once, what makes the algorithm not only time consuming but also memory consuming. As a consequence, when pricing vanilla options LSM loses the comparison with finite difference or binomial tree methods. However, there are not many methods allowing to price exotic American options. LSM is one of the few and gives satisfactory results.


Due to the possibility of pricing a vast class of options, methods based on Monte Carlo simulations have a great significance in finance. Time consumption may be regarded as their weakness, however, the computational power of modern processors is constantly increasing. Moreover, algorithms based on Monte Carlo are suitable for division into multiple threads and being run on huge computer clusters. Every year obtaining some fixed accuracy takes less time. Therefore, the methods presented in the thesis, which already are very popular, over time will even increase their importance.

\begin{thebibliography}{99}
\addcontentsline{toc}{chapter}{\bfseries References}

\bibitem{bjork}
T. Bj\"{o}rk, \emph{Arbitrage Theory in Continuous Time}, Oxford University Press,  New York, 3rd edition, 2009

\bibitem{follmer}
H. F\"{o}llmer, A. Schied, \emph{Stochastic finance. An introduction in discrete time}, Walter de Gruyter, Berlin, 2nd edition, 2004

\bibitem{latala}
R. Latała, \emph{Wstęp do Analizy Stochastycznej}, Warszawa, 2011

\bibitem{london}
J. London, \emph{Modeling derivatives in C++}, John Wiley \& Sons, Hoboken, 2005

\bibitem{l-sch}
F. Longstaff, E. Schwartz, \emph{Valuing American Options by Simulation: A Simple Least Squares Approach}, The Review of Financial Studies, Vol.14, No.1, pp.113-147, 2001

\bibitem{wilmott}
P. Wilmott, \emph{On quantitative finance}, John Wiley \& Sons, Chichester, 2nd edition, 2006

\end{thebibliography}


\end{document}
