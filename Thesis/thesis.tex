\documentclass[a4paper,12pt]{article}
\usepackage[utf8]{inputenc}
\usepackage{amsmath, amsthm, amssymb}
\usepackage[polish, english]{babel}
\usepackage[OT4]{fontenc}
\usepackage[left=2.5cm,right=2.5cm]{geometry}
 
\newtheorem{thm}{Theorem}[section]
\newtheorem{prop}{Proposition}

\theoremstyle{definition}
\newtheorem{mydef}{Definition}[section]
\newtheorem{example}{Example}[section]

\theoremstyle{remark}
\newtheorem{remark}{Remark}

% \newcounter{example}
% \newenvironment{example}
%   {\refstepcounter{example} \par\medskip\noindent \textbf{Example~\arabic{example}.}  }
%   {\begin{flushright}$\square$\end{flushright}}
  

     
%opening
\title{Pricing american and exotic option using Least Squares Monte Carlo approach}
\author{Grzegorz Łoś}

\begin{document}
 
\newpage
\thispagestyle{empty}
\begin{center}
\textbf{\large Uniwersytet Wrocławski\\
Wydział Matematyki i Informatyki\\
Instytut Matematyczny}\\
\textit{\large specjalność: brak}\\
\vspace{4cm}
\textbf{\textit{\large Grzegorz Łoś}\\
\vspace{0.5cm}
{\Large Pricing american and exotic option using Least Squares Monte Carlo approach}}\\
\end{center}
\vspace{3cm}
{\hspace*{6.5cm}\large Praca magisterska\\
\hspace*{6.5cm}\large  napisana pod kierunkiem\\
\hspace*{6.5cm}\large  doktora Pawła Kawy }
\vfill
\begin{center}
{\large Wrocław 2013}\\
\end{center}

\newpage
\thispagestyle{empty}
\vspace*{10cm}
\noindent {\large Oświadczam, że pracę magisterską wykonałem samodzielnie\\ i~zgłaszam ją do oceny.\\[1.5cm]
Data:....................\hfill Podpis autora pracy:.........................\\[1.5cm]
Oświadczam, że praca jest gotowa do oceny przez recenzenta.\\[1.5cm]
Data:.................... \hfill Podpis opiekuna pracy:.........................}

\newpage

\tableofcontents

\newpage

\begin{abstract}
  Thesis presents a technique of option pricing known as Least Squares Monte Carlo or Longstaff-Schwartz model, due to it's discoverers Francis Longstaff and Eduardo Schwartz. In the first chapter some general facts from Stochastic Analysis, Monte Carlo theory and Black-Scholes model are reminded. Next chapters describe LSM algorithm, the implementation, examples of usage for valuation of exotic options.
\end{abstract}

\section{Preliminaries}
We expect from the reader some basic knowledge of stochastic processes and option pricing. Purpose of this section is to recall definitions and facts essential in this work and establish notation.

\subsection{Elements of stochastic analysis}
\begin{mydef}
 Let $(\Omega, \mathcal{F}, {P})$ be a probability space, $(E, \mathcal{E})$ be a measureable space and $T$ be an arbitrary set. A \textbf{stochastic process} with values in a measurable space $E$, indexed by an arbitrary set $T$, is a family of random variables $X = (X_t)_{t \in T}$, where each $X_t$ is $E$-valued.
 
 For given $\omega \in \Omega$ a \textbf{trajectory} of process $X$ is a function $t \mapsto X_t(\omega)$, with domain $T$ and codomain $E$.
\end{mydef}

\begin{remark}
 Stochastic process may be viewed as a function $X: \Omega \rightarrow \mathbb{E}^T$. Then a trajectory is value of such function, i.e. for given $\omega,\ X(\omega)$ is a trajectory. In the other words, a stochastic process is a random function and a trajectory is it's concrete realization.
\end{remark}

\begin{remark}
 In our applications space $E$ will be equal $\mathbb{R}$ or $\mathbb{R}^n$. 
\end{remark}


\begin{mydef}
 A \textbf{Brownian motion} (or a \textbf{Wiener process}) is a stochastic process $(W_t)_{t \geq 0}$ defined by following conditions:
 \begin{itemize}
  \item $W_0 = 0$ a.s.,
  \item for any $t,\ W_t \sim \mathcal{N}(0,t)$,
  \item increments of $W$ are independent (i.e. for any $t_0 \leq t_1 \leq \ldots \leq t_n$ random variable $W_{t_0}, W_{t_1} - W_{t_0},\ \ldots, W_{t_n} - W_{t_{n-1}}$ are independent,
  \item increments of $W$ are stationary (i.e. for every $0 \leq s < t,\ W_t-W_s$ is equal in distribution to $W_{t-s})$,
  \item trajectories of $W$ are continous a.s.
 \end{itemize}
\end{mydef}

\begin{mydef}
 Let $\mu, \sigma \in C^1,\ \xi$ be $\mathcal{F}_s$-measurable random variable. We say that process $X=(X_t)_{t=0}^T$ solves a \textbf{stochastic differential equation (SDE)}
 \[dX_t = \mu(X_t)dt + \sigma(X_t) dW_t,\ X_0 = \xi\]
 if and only if
 \[X_t = \xi + \int\limits_0^t \mu(X_s)ds + \int\limits_0^t\sigma(X_s) dW_s\]
for all $t \in [0,T)$.
\end{mydef}

\begin{remark}
 Note that $\int\limits_0^t \mu(X_s)ds$ is an ordinary integral, while $\int\limits_0^t\sigma(X_s) dW_s$ is an It\^{o} integral.
\end{remark}

\begin{mydef}
 A stochastic process $S$ given by SDE
 \begin{equation}
  dS_t = \mu S_t dt + \sigma S_t dW_t, 
  \label{gbm}
 \end{equation}
where $\mu,\sigma \in \mathbb{R}$ is called a \textbf{geometic Brownian motion}.
\end{mydef}

\begin{mydef}
 \textbf{Filtration} $(\mathcal{F}_t)_{t=0}^T$ on probability space $(\Omega, \mathcal{F}, {P})$ is an increasing family of $\sigma$-algebras contained in $\mathcal{F}$, i.e. for all $s<t,\ \mathcal{F}_s \subseteq \mathcal{F}_t \subseteq \mathcal{F}$.
\end{mydef}
Sometimes $\mathcal{F}_t$ is interpreted as a set of all events observable up to time $t$.

\begin{mydef}
 Process $X=(X_t)_{t=0}^T$ is called $\mathcal{F}_t$-\textbf{adoptable} if and only if for all $t~\in~[0,T],\ X_t$ is $\mathcal{F}_t$-measurable.
\end{mydef}
Minimal filtration to which $X$ is adoptable if of course filtration generated by $X$, defined as $\mathcal{F}_t^X = \sigma(X_s:\ s \leq t)$.

\begin{mydef}
 A \textbf{stopping time} with respect to filtration $(\mathcal{F}_t)_{t=0}^T$ is a random variable $\tau:\ \Omega \rightarrow [0,T]\cup\{\infty\}$, such that $\{\tau \leq t\} \in \mathcal{F}_t$ for all $t \in [0,T]$.
\end{mydef}
If $X$ is a process corresponding to some risky game, then a stopping time may be seen as a strategy which tells us whether we should withdraw at time $t$ basing only on the information accessible at time $t$.
\begin{example}
 A typical example of a stopping time is first $t$, when $X_t$ reaches a fixed barrier, i.e.
 \[\tau = \inf\{t\in[0,T]: X_t \geq b\}\]
\end{example}
Stopping times play important part in financial mathematics. Often we are interested iin finding an optimal strategy for exercising an option. Such strategy is a stopping time.

\begin{mydef}
 We call a stochastic process $X=(X_t)_{t=0}^T$ a \textbf{martingale} with respect to filtration $(\mathcal{F}_t)_{t=0}^T$ if and only if it satisfies following conditions:
 \begin{itemize}
  \item for all $t \in [0,T],\ X_t$ is $\mathcal{F}_t$-measurable,
  \item $E|X_t| < \infty$,
  \item for all $0 \leq s < t \leq T, E[X_t|\mathcal{F}_s] = X_s$  a.s.
 \end{itemize}
\end{mydef}

\begin{example}
 Brownian motion $W$ is a martingale. Indeed, we have
 \begin{equation*}
  E[W_t|\mathcal{F}_s] = E[W_t-W_s|\mathcal{F}_s] + E[W_s|\mathcal{F}_s] = E[W_t-W_s] + W_s = W_s.
 \end{equation*}
 for all $t > s$.
\end{example}
\begin{example}
 Process $(W_t^2-t)_t$ is a martingale. As previously, for any $t > s$
 \begin{equation*}
  \begin{split}
       & E[W_t^2-t|\mathcal{F}_s] = E[(W_t-W_s)^2 + 2W_tW_s - W_s^2|\mathcal{F}_s] -t =\\ 
    =\ & E[(W_t-W_s)^2|\mathcal{F}_s] + E[2W_tW_s|\mathcal{F}_s] - E[W_s^2|\mathcal{F}_s] -t = \\
    =\ & E[(W_t-W_s)^2] + 2W_sE[W_t|\mathcal{F}_s] - W_s^2 -t = \\
    =\ & E[(W_t-W_s)^2] + 2W_s^2 - W_s^2 -t = W_s^2 - s.
  \end{split}
 \end{equation*}
\end{example}


\subsection{Elements of Monte Carlo theory}
\subsection{Types of options}
\subsection{Black-Scholes model}
\subsection{Pricing European options with Monte Carlo method}
\newpage
\section{Least Squares Monte Carlo}

\end{document}
