\documentclass[a4paper,12pt, twoside]{article}
\usepackage[utf8]{inputenc}
\usepackage{amsmath, amsthm, amssymb}
\usepackage[polish, english]{babel}
\usepackage[OT4]{fontenc}
\usepackage[left=2.5cm,right=2.5cm]{geometry}
\usepackage{fancyhdr}
\usepackage[section]{placeins} %powinno utrzymac obrazki w odpowiednich sekcjach
\usepackage{graphicx}

\newtheorem{thm}{Theorem}[section]
\newtheorem{prop}[thm]{Proposition}
\newtheorem{coll}[thm]{Colloraly}

\theoremstyle{definition}
\newtheorem{mydef}{Definition}[section]
\newtheorem{example}{Example}[section]

\theoremstyle{remark}
\newtheorem{remark}{Remark}

% \newcounter{example}
% \newenvironment{example}
%   {\refstepcounter{example} \par\medskip\noindent \textbf{Example~\arabic{example}.}  }
%   {\begin{flushright}$\square$\end{flushright}}
  
\def\Var{{\rm Var}\,}
\def\E{{\mathbb{E}}\,}
\def\P{{\mathbb{P}}\,}
\def\conv{\xrightarrow[n \rightarrow \infty]{}}
\def\limn{\lim\limits_{n \rightarrow \infty} }

%opening
\title{Pricing american and exotic option using Least Squares Monte Carlo approach}
\author{Grzegorz Łoś}

\pagestyle{fancy}
\fancyhf{}
\fancyhead[LE,RO]{\small\bfseries\thepage}
\fancyhead[LO]{\small\bfseries\nouppercase\rightmark}
\fancyhead[RE]{\small\bfseries\nouppercase\leftmark}

%\lhead{\nouppercase{\bfseries \leftmark}}
%\rhead{\nouppercase \rightmark}
\setlength{\headheight}{15pt}

\begin{document}
 
\newpage
\thispagestyle{empty}
\begin{center}
\textbf{\large Uniwersytet Wrocławski\\
Wydział Matematyki i Informatyki\\
Instytut Matematyczny}\\
\textit{\large specjalność: brak}\\
\vspace{4cm}
\textbf{\textit{\large Grzegorz Łoś}\\
\vspace{0.5cm}
{\Large Pricing american and exotic option using Least Squares Monte Carlo approach}}\\
\end{center}
\vspace{3cm}
{\hspace*{6.5cm}\large Praca magisterska\\
\hspace*{6.5cm}\large  napisana pod kierunkiem\\
\hspace*{6.5cm}\large  doktora Pawła Kawy }
\vfill
\begin{center}
{\large Wrocław 2013}\\
\end{center}

\newpage
\thispagestyle{empty}
\vspace*{10cm}
\noindent {\large Oświadczam, że pracę magisterską wykonałem samodzielnie\\ i~zgłaszam ją do oceny.\\[1.5cm]
Data:....................\hfill Podpis autora pracy:.........................\\[1.5cm]
Oświadczam, że praca jest gotowa do oceny przez recenzenta.\\[1.5cm]
Data:.................... \hfill Podpis opiekuna pracy:.........................}

\newpage

\tableofcontents

\newpage

\begin{abstract}
  Thesis presents a technique of option pricing known as Least Squares Monte Carlo or Longstaff-Schwartz model, due to it's discoverers Francis Longstaff and Eduardo Schwartz. In the first chapter some general facts from Stochastic Analysis, Monte Carlo theory and Black-Scholes model are reminded. Next chapters describe LSM algorithm, the implementation, examples of usage for valuation of exotic options.
\end{abstract}

\section{Preliminaries}
We expect from the reader some basic knowledge of stochastic processes and option pricing. Purpose of this section is to recall definitions and facts essential for this work and to establish notation.

\subsection{Elements of stochastic analysis}
\begin{mydef}
 Let $(\Omega, \mathcal{F}, \P)$ be a probability space, $(E, \mathcal{E})$ be a measureable space and $T$ be an arbitrary set. A \textbf{stochastic process} with values in a measurable space $E$, indexed by an arbitrary set $T$, is a family of random variables $X = (X_t)_{t \in T}$, where each $X_t$ is $E$-valued.
 
 For given $\omega \in \Omega$ a \textbf{trajectory} of process $X$ is a function $t \mapsto X_t(\omega)$, with domain $T$ and codomain $E$.
\end{mydef}

\begin{remark}
 Stochastic process may be viewed as a function $X: \Omega \rightarrow \mathbb{E}^T$. Then a trajectory is value of such function, i.e. for given $\omega,\ X(\omega)$ is a trajectory. In the other words, a stochastic process is a random function and a trajectory is it's concrete realization.
\end{remark}

\begin{remark}
 In our applications space $E$ will be equal $\mathbb{R}$ or $\mathbb{R}^n$. 
\end{remark}

\begin{mydef}
 A \textbf{Brownian motion} (or a \textbf{Wiener process}) is a stochastic process $(W_t)_{t \geq 0}$ defined by following conditions:
 \begin{itemize}
  \item $W_0 = 0$ a.s.,
  \item for any $t,\ W_t \sim \mathcal{N}(0,t)$,
  \item increments of $W$ are independent (i.e. for any $t_0 \leq t_1 \leq \ldots \leq t_n$ random variable $W_{t_0}, W_{t_1} - W_{t_0},\ \ldots, W_{t_n} - W_{t_{n-1}}$ are independent,
  \item increments of $W$ are stationary (i.e. for every $0 \leq s < t,\ W_t-W_s$ is equal in distribution to $W_{t-s})$,
  \item trajectories of $W$ are continous a.s.
 \end{itemize}
\end{mydef}
\noindent In this paper $W$ will always denote Brownian motion.

\begin{mydef}
 \textbf{Filtration} $(\mathcal{F}_t)_{t=0}^T$ on probability space $(\Omega, \mathcal{F}, {P})$ is an increasing family of $\sigma$-algebras contained in $\mathcal{F}$, i.e. for all $s<t,\ \mathcal{F}_s \subseteq \mathcal{F}_t \subseteq \mathcal{F}$.
\end{mydef}
\noindent Sometimes $\mathcal{F}_t$ is interpreted as a set of all events observable up to time $t$.

\begin{mydef}
 Process $X=(X_t)_{t=0}^T$ is called $\mathcal{F}_t$-\textbf{adoptable} if and only if for all $t~\in~[0,T],\ X_t$ is $\mathcal{F}_t$-measurable.
\end{mydef}
\noindent Minimal filtration to which $X$ is adoptable if of course filtration generated by $X$, defined as $\mathcal{F}_t^X = \sigma(X_s:\ s \leq t)$.

\begin{mydef}
 A \textbf{stopping time} with respect to filtration $(\mathcal{F}_t)_{t=0}^T$ is a random variable $\tau:\ \Omega \rightarrow [0,T]\cup\{\infty\}$, such that $\{\tau \leq t\} \in \mathcal{F}_t$ for all $t \in [0,T]$.
\end{mydef}
If $X$ is a process corresponding to some risky game, then a stopping time may be seen as a strategy which tells us whether we should withdraw at time $t$, basing only on the information accessible at time $t$.
\begin{example}
 A typical example of a stopping time is first $t$, when $X_t$ reaches a fixed barrier, i.e.
 \[\tau = \inf\{t\in[0,T]: X_t \geq b\}\]
\end{example}
\noindent Stopping times play important role in financial mathematics. Often we are interested in finding an optimal strategy for exercising an option. Such strategy is a stopping time.

\begin{mydef}
 We call a stochastic process $M=(M_t)_{t=0}^T$ a \textbf{martingale} with respect to filtration $(\mathcal{F}_t)_{t=0}^T$ if and only if it satisfies following conditions:
 \begin{itemize}
  \item for all $t \in [0,T],\ M_t$ is $\mathcal{F}_t$-measurable,
  \item $\E|M_t| < \infty$,
  \item for all $0 \leq s < t \leq T, \E[M_t|\mathcal{F}_s] = M_s$  a.s.
 \end{itemize}
\end{mydef}

\begin{example}
 Brownian motion $W$ is a martingale. Indeed, we have
 \begin{equation*}
  \E[W_t|\mathcal{F}_s] = \E[W_t-W_s|\mathcal{F}_s] + \E[W_s|\mathcal{F}_s] = \E[W_t-W_s] + W_s = W_s.
 \end{equation*}
 for all $t > s$.
\end{example}
\begin{example}
 \label{ex:angleWt}
 Process $(W_t^2-t)_t$ is a martingale. As previously, for any $t > s$
 \begin{equation*}
  \begin{split}
       & \E[W_t^2-t|\mathcal{F}_s] = \E[(W_t-W_s)^2 + 2W_tW_s - W_s^2|\mathcal{F}_s] -t =\\ 
    =\ & \E[(W_t-W_s)^2|\mathcal{F}_s] + \E[2W_tW_s|\mathcal{F}_s] - \E[W_s^2|\mathcal{F}_s] -t = \\
    =\ & \E[(W_t-W_s)^2] + 2W_s\E[W_t|\mathcal{F}_s] - W_s^2 -t = \\
    =\ & \E[(W_t-W_s)^2] + 2W_s^2 - W_s^2 -t = W_s^2 - s.
  \end{split}
 \end{equation*}
\end{example}

\noindent Before we move to SDE's and It\^{o}'s lemma, let us do a notation remark. Let $X$ be a square integrable process and $Z$ be a process with respect to which we can integrate ($Z$ may be a Brownian motion, a martingale, or in general even a semimartingale).
\begin{itemize}
 \item $\int\limits_0^t X_s ds$ is a ``standard'' Riemann integral. The integrated function is random, but for fixed $\omega$ we can use Riemann's theory to integrate it.
 \item $\int\limits_0^t X_s dZ_s$ denotes isometric It\^{o} integral.
\end{itemize}

\begin{mydef}
 Let $\mu, \sigma \in C^1,\ \xi$ be $\mathcal{F}_s$-measurable random variable. We say that process $X=(X_t)_{t=0}^T$ solves a \textbf{stochastic differential equation (SDE)}
 \[dX_t = \mu(X_t)dt + \sigma(X_t) dW_t,\ X_0 = \xi\]
 if and only if
 \[X_t = \xi + \int\limits_0^t \mu(X_s)ds + \int\limits_0^t\sigma(X_s) dW_s\]
for all $t \in [0,T)$.
\end{mydef}

\begin{mydef}
 A stochastic process $S$ given by SDE
 \begin{equation}
  dS_t = \mu S_t dt + \sigma S_t dW_t, 
  \label{eq:gbm}
 \end{equation}
where $\mu,\sigma \in \mathbb{R}$ is called a \textbf{geometic Brownian motion}.
\end{mydef}

\noindent Now we can formulate a version of It\^{o}'s lemma, which is commonly used in financial mathematics.
\begin{thm}[It\^{o}'s lemma]
  Let $S$ be a geometic Brownian motion as in (\ref{eq:gbm}), $V:\ \mathbb{R}^2 \rightarrow \mathbb{R},\ V \in C^2$. Then 
  \begin{equation*}
   V(S_t, t) = V(S_0, 0) + \int\limits_0^t \frac{\partial V}{\partial S}(S_r,r)dS_r + \int\limits_0^t \frac{\partial V}{\partial t}(S_r,r)dr + \frac{1}{2}\sigma^2 S_t^2 \int\limits_0^t \frac{\partial^2 V}{\partial S^2}(S_r,r)dr
  \end{equation*}
  or equivalently in SDE form
  \begin{equation}
   \label{eq:ito}
   dV(S_t, t) = \frac{\partial V}{\partial S}(S_t,t)dS_t + \frac{\partial V}{\partial t}(S_t,t)dt + \frac{1}{2}\sigma^2 S_t^2 \frac{\partial^2 V}{\partial S^2}(S_t,t)dt   .
  \end{equation}  
\end{thm}

\noindent In the books on the stochastic processes It\^{o}'s lemma is proven in much greater generality. However, for our purposes, as in many other literature on financial mathematics, formulated theorem will be sufficient. Equation (\ref{eq:ito}) is also called an It\^{o}'s formula.


\subsection{Elements of Monte Carlo theory}
Monte Carlo methods are a class of algorithms designed for estimation of unknown values by simulation. They do not refer to any particular algorithm, they are rather a general ``recipe'' for procedures, which obtain results by simulation.

Suppose we want to estimate an unknown value $I$, which can be written as expected value of some random variable, i.e.
\[ I = \E Y. \]
The idea of Monte Carlo technique is to replicate $Y$ many times, and as estimation of $I$ take an average. So
\[ I \approx \frac{1}{n} \sum\limits_{i=1}^n Y_i, \]
where $n$ is big natural number and $Y_i$ are independent, with the same distribution as $Y$.

This procedure is justified by the strong low of large numbers. Let 
\begin{equation}
 \label{eq:CMC}
 \hat{Y}_n = \frac{1}{n}\sum\limits_{i=1}^n Y_i
\end{equation}
It is called \textbf{crude Monte Carlo} estimator. Of course $\E\hat{Y}_n = \E Y = I$, so $\hat{Y}_n$ is unbiased. Moreover, from law of large numbers $\hat{Y}_n \conv I$ a.s., what implies 
\[ \P(|\hat{Y}_n - I| > b) \conv 0 \hbox{\ \ a.s.,} \]
for any $b > 0$. 

Here appears a natural question, how big should be number $n$ for fixed $b$, so we could tell, that error of our estimation, with known probability $1-\alpha$, is not greater than $b$?
Let $\sigma = \sqrt{\Var{Y}}$, $z_{1-\alpha/2} = \Phi^{-1}(1-\alpha/2)$, where $\Phi$ is cumulative distribution function of normal distribution. Simple calculations give
\begin{align*}
 \P(-b \leq \hat{Y}_n - I \leq b) &= 1 - \alpha\\
 \P(-b \leq \frac{\sum\limits_{i=1}^n Y_i - nI}{n}  \leq b) &= 1 - \alpha\\
 \P(-\frac{b\sqrt{n}}{\sigma} \leq \frac{\sum\limits_{i=1}^n Y_i - nI}{\sqrt{n}\sigma}  \leq \frac{b\sqrt{n}}{\sigma}) &= 1 - \alpha
\end{align*}
From Central Limit Theorem we have
\[ \limn \P(-z_{1-\alpha/2} \leq \frac{\sum\limits_{i=1}^n Y_i - nI}{\sqrt{n}\sigma}  \leq z_{1-\alpha/2}) = 1 - \alpha, \]
hence for large $n$ we have
\[z_{1-\alpha/2} \approx \frac{b\sqrt{n}}{\sigma}.\]
In a typical situation we do not know variation of $Y$ (we don't even know the expected value, yet we are using Monte Carlo method to find it!), so above formula has rather theoretical meaning. However, there is a version of CLT which uses unbiased estimator
\[ \hat{\sigma} = \frac{1}{n-1}\sum\limits_{i=1}^n (Y_i - \hat{Y}_n)^2 \]
instead of $\sigma$, what allows to replace $\sigma$ by $\hat{\sigma}$ in above approximation. We have proven following 
\begin{thm}
 Dependency between number of simulations $n$, error $b$ and confidence level $\alpha$ is given by following formulas
 \begin{equation}
   \label{eq:error}
   b = \frac{\hat{\sigma} z_{1-\alpha/2}}{\sqrt{n}}
 \end{equation}
 \begin{equation}
   \label{eq:sim}
   n = \frac{\hat{\sigma}^2 z_{1-\alpha/2}^2}{b^2}.
 \end{equation}
\end{thm}
\noindent Equation (\ref{eq:error}) tells us how big is an error of estimation when we performed $n$ simulation. Equation (\ref{eq:sim}) inverses situation, it allows us to plan the number of simulations necessary to obtain requested accuracy.

\paragraph{Simulation.} Now, when we know how Monte Carlo methods work, we need to describe how to get random values. Most computational enviroments and programming languages have built-in generator of values from uniform distribution. We will show how to obtain (pseudo)random variables with ubiquitous normal distribution.

The most popular way is to use Box-Muller algorithm
\begin{enumerate}
 \item generate independent random variables $U_1, U_2$ with uniform distribution,
 \item $N_1 := \sqrt{-2\log(U_1)} \cos(2\pi U_2)$,
 \item $N_2 := \sqrt{-2\log(U_1)} \sin(2\pi U_2)$,
 \item return($N_1$, $N_2$).
\end{enumerate}
This method returns values of two independent random variables coming from standard normal distribution. Necessity of calculating sine and cosine may is slowing algorithm down. The second procedure is significantly faster.
\begin{enumerate}
 \item generate independent random variables $U_1, U_2$ with uniform distribution,
 \item $V_1 := 2U_1-1$,
 \item $V_2 := 2U_2-1$,
 \item $W := U_1^2 + U_2^2$,
 \item $N_1 := \sqrt{\frac{-2\log(W)}{W}} V_1$,
 \item $N_2 := \sqrt{\frac{-2\log(W)}{W}} V_2$,
\end{enumerate}

Above methods allows us to generate independent random variables. In the real world however, movements of observed values are often correlated. Following algorithm generetes values from multivariate normal distribution with given mean vector $\mu$ and covariation matrix $\Sigma$. 
\begin{enumerate}
 \item use Cholesky decomposition to obtain matrix $L$ such that $\Sigma = LL^T$,
 \item using previously shown method generate vector $N$ of independent random variables with standard normal distribution.
 \item return $\mu + LN$.
\end{enumerate}


\newpage
\section{Basics of option pricing}
The main field of interest of financial mathematics is pricing contingent claims, which are assets whose payoff depends on stock prices. In case of European options there exists a straightforward forumula, derived by Black and Scholes in 1973, which gives price of the option. Antother way of pricing is using Monte Carlo simulation. Although simulations are time-consuming and for that reason less effective than Black-Scholes formula, they are widely used because of possibility to adjust them to diffrent contingent claims. 

Throughout this thesis we assume we are given a probability space $(\Omega, \mathcal{F}, (\mathcal{F}_t)_{t=0}^T, \P)$. Since it is observed by investors, $\P$ is sometimes called a real measure, in opposite to artificial risk-neutral measure, which will be later on. Elements of $\Omega$ are called market scenarios. Furthermore $\sigma$-algebra $\mathcal{F}_t$ may be seen as set of all events observable up to time $t$.

We consider a market with $d+1$ assets, where each asset $S^{(i)}$ is modelled as a stochastic process adapted to $(\mathcal{F}_t)_{t=0}^T$. The $0$\textsuperscript{th} asset is considered to be a riskless bond, what means that it's value does not depend on market scenario $\omega$ and is given by 
\[S^{(0)}_t = e^{rt},\]
where $r$ is riskless interest rate.

For convenience we also consider a discounted time processes
\[ X^{(i)}_t = \frac{S^{(i)}_t}{S^{(0)}_t}. \]
It allows us to compare asset prices quoted at diffrent times.


\subsection{Types of options}
We start this section with definition of mentioned contingent claim.
\begin{mydef}
 \label{def:cc_eu}
 \textbf{European contingent claim} is a non-negative random variable $V$ on $(\Omega, \mathcal{F}_T, \P)$. \textbf{Derivative} of underlying assets $S^{(0)}, S^{(1)}, \ldots, S^{(d)}$ is a contingent claim which is measurable with respect to $\sigma$-algebra generated by price processes.
\end{mydef}
European contingent claims may be seen as assets yielding a random payoff at \textbf{exercise date} $T$ (also called \textbf{maturity}). Of course the seller of such contingent claim can't take a random amount of monye from the buyer. What should be then the price at time $0$? It is main concern of financial mathematics.

The most important type of derivatives are options.
\begin{mydef}
 An \textbf{European call option} on asset $S^{(i)}$ with exercise date $T$ and \textbf{strike price} $K$ gives it's owner the right, but not the obligation, to \underline{buy} that asset at time $T$ for a fixed price $K$.
 
 An \textbf{European put option} on asset $S^{(i)}$ with exercise date $T$ and \textbf{strike price} $K$ gives it's owner the right, but not the obligation, to \underline{sell} that asset at time $T$ for a fixed price $K$.
\end{mydef}
From the definition
\begin{equation}
 \label{eq:ecall}
 V\textsuperscript{call} = (S^{(i)}_T - K)_+
\end{equation}
\begin{equation}
 \label{eq:eput}
 V\textsuperscript{put} = (K - S^{(i)}_T )_+
\end{equation}
where $(x)_+ = \max(0,x)$.

A popular modification of above vanilla options are \textbf{barrier options}. Their payoff depends not only on the stock price at maturity, but also on historical prices. Barrier options are divided into \textit{knock-in} options, which may be ``turned on'', and \textit{knock-out} options, which may be ``turned off'' in case of reaching some \textbf{barrier}. Let us write down two example definitions
\begin{mydef}
Payoff of \textbf{up-and-in call} option on asset $S^{(i)}$ with exercise date $T$, strike price $K$ and barrier $\bar{B}$, is
\[ C^{\hbox{\scriptsize call}}_{\hbox{\scriptsize u\&i}} = 
\begin{cases}
 (S^{(i)}_T - K)_+    & \hbox{if } \sup\limits_{0 \leq t \leq T} S^{(i)}_t \geq \bar{B}\\
 0                    & \hbox{otherwise.}
\end{cases}
\]
\end{mydef}
Payoff of knock-out option is zeroed when barrier is reached.
\begin{mydef}
Payoff of \textbf{down-and-out put} option on asset $S^{(i)}$ with exercise date $T$, strike price $K$ and barrier $\underline{B}$, is
\[ C^{\hbox{\scriptsize put}}_{\hbox{\scriptsize d\&o}} = 
\begin{cases}
 (K - S^{(i)}_T)_+    & \hbox{if } \inf\limits_{0 \leq t \leq T} S^{(i)}_t \leq \underline{B}\\
 0                    & \hbox{otherwise.}
\end{cases}
\]
\end{mydef}
We have eight types of barrier options, as every one is call or put, up or down, in or out. They are all defined in similar manner. The best way to get a grip on barrier options is possibly through a graphical example. Figure \ref{fig:barrier} discusses payoff from up-and-out call option in three diffrent scenarios. 
\begin{figure}
\centering
 \includegraphics[scale=0.5]{images/barrier.pdf}
\caption{We consider up-and-out call option with strike 110, barrier 140, expiring at time 1. In the red scenario option is in-the-money at the maturity, however in the past barrier was crossed, thus payoff is 0. In the blue scenario stock price ends about the level of 115, barrier wasn't reached, hence our payoff equals 5. In case of green scenario payoff is 0, as it would be for vanilla option, because we ended out-of-the-money }
\label{fig:barrier}
\end{figure}

\FloatBarrier
\subsection{Black-Scholes model}
In this section we will recall famous Black-Scholes model, leading to a straightforward formula for picing European options. Here we are considering a market with only one stock $S$. The assumptions of that model, are following.

\noindent \textbf{The underlying stock price follows a lognormal model.} Moreover drift $\mu$ and volatility $\sigma$ are constant in time. Thus, SDE for option price is described by equation
\begin{equation}
 \label{eq:stockDynamics}
 dS_t = \mu S_t dt + \sigma S_t dW_t. 
\end{equation}

\noindent \textbf{There exists constant risk-free interest rate $r$.} Investors may both, borrow and lend, any amount of cash at rate $r$.

\noindent \textbf{It is possible to buy and sell any amount of stock.} It means that investors can even trade fractional numbers of stock, and short sell unbounded quantity of shares.

\noindent \textbf{All transactions do not incur any additional costs.}

\noindent \textbf{Market does not admit arbitrage opportunities.} A strict definition of arbitrage will be given in the section \textit{Risk-neutral pricing}. For now it is sufficient to know that the market is arbitrage-free when assets yielding identical cash flows have the same price and instruments with a known payoff in the future must be priced as its future payoff discounted at the risk-free rate.

Let us suppose we are constructing a portfolio consisting of short position in one option and long position in $\Delta$ shares. Hence
\begin{equation}
 \label{eq:portfolio}
  \Pi = \Delta S - V 
\end{equation}
We may ask how much the portfolio will change in a short period of time. We have
\[ d\Pi = \Delta dS - dV  \]
Note that we do not have to differentiate $\Delta$, because it does not change in an infinitezimal increment in time. To handle $dV$ we will use It\^{o}s lemma. Thus
\[ d\Pi = \Delta dS - \frac{\partial V}{\partial S}dS - \frac{\partial V}{\partial t}dt - \frac{1}{2}\sigma^2 S^2 \frac{\partial^2 V}{\partial S^2}dt  \]
Risk in increment of portfolio is carriered by changes of stock price. By choosing
\begin{equation}
 \label{eq:delta}
 \Delta = \frac{\partial V}{\partial S}
\end{equation}
we get rid off uncertainity. Now we have
\begin{equation}
  \label{eq:portfolio_inc}
 d\Pi = -(\frac{\partial V}{\partial t} + \frac{1}{2}\sigma^2 S^2 \frac{\partial^2 V}{\partial S^2})dt.
\end{equation}
Increment of $\Pi$ does not depend on any risky asset, hence no-arbitrage assumption induces
\[ d\Pi = r\Pi dt. \]
After substituting (\ref{eq:portfolio}), (\ref{eq:delta}) and (\ref{eq:portfolio_inc}) into above equation, we obtain
\[ -(\frac{\partial V}{\partial t} + \frac{1}{2}\sigma^2 S^2 \frac{\partial^2 V}{\partial S^2})dt = r(\frac{\partial V}{\partial S} S - V)dt, \]
which after simple calculation gives
\begin{equation}
 \label{eq:BSeq}
 \frac{\partial V}{\partial t} + \frac{1}{2}\sigma^2 S^2 \frac{\partial^2 V}{\partial S^2} + r\frac{\partial V}{\partial S} S - rV = 0.
\end{equation}
(\ref{eq:BSeq}) is known as \textbf{Black-Scholes equation}. Note that so far we did not say anything about the final condition of (\ref{eq:BSeq}), i.e. about payoff off the option, thus this is a general equation. For European options it has straightforward solution.

\begin{align*}
V\textsuperscript{call} &= N(d_1)S  - N(d_2)K e^{-r(T-t)},\\
V\textsuperscript{put} &= -N(-d_1)S + N(-d_2)K e^{-r(T-t)},\\
d_1 &= \frac{\ln\left(\frac{S}{K}\right)+\left(r+\frac{\sigma^{2}}{2}\right)(T-t)}{\sigma\sqrt{T-t}}\\
d_2 &= \frac{\ln\left(\frac{S}{K}\right)+\left(r-\frac{\sigma^{2}}{2}\right)(T-t)}{\sigma\sqrt{T-t}} = d_{1}-\sigma\sqrt{T-t}.
\end{align*}
It is called \textbf{Black-Scholes formula}.

\subsection{Risk-neutral pricing}
\subsection{Pricing European options with Monte Carlo method}
\begin{figure}
\centering
 \includegraphics[scale=0.5]{images/trajectories.pdf}
\caption{Generated trajectories}
\end{figure}



\newpage
\section{Least Squares Monte Carlo}
\subsection{The idea of LSM}
\begin{figure}[h]
\centering
 \includegraphics[scale=0.5]{images/LSMidea1.pdf}
\caption{To co chcemy}
\end{figure}
\begin{figure}
\centering
 \includegraphics[scale=0.5]{images/LSMidea2.pdf}
\caption{To co mozemy}
\end{figure}

\begin{thebibliography}{99}
\addcontentsline{toc}{section}{\bfseries References}

\bibitem{follmer}
H. F\"{o}llmer, A. Schied, \emph{Stochastic finance. An introduction in discrete time}, Walter de Gruyter, Berlin, 2nd edition, 2004

\bibitem{latala}
R. Latała, \emph{Wstęp do Analizy Stochastycznej}, Warszawa, 2011

\bibitem{london}
J. London, \emph{Modeling derivatives in C++}, John Wiley \& Sons, Hoboken, 2005

\bibitem{l-sch}
F. Longstaff, E. Schwartz, \emph{Valuing American Options by Simulation: A Simple Least Swuares Approach}, The Review of Financial Studies, Vol.14, No.1, pp.113-147, 2001

\bibitem{wilmott}
P. Wilmott, \emph{On quantitative finance}, John Wiley \& Sons, Chichester, 2nd edition, 2006

\end{thebibliography}


\end{document}
